\chapter{Aug.~28 --- Irreducibility}

\section{Irreducibility}

\begin{definition}
  A topological space $X$ is \emph{reducible}
  if there exists a decomposition
  \[
    X = X_1 \cup X_2
  \]
  with $X_1, X_2 \subsetneq X$ closed.
  Otherwise $X$ is \emph{irreducible}.
\end{definition}

\begin{theorem}\label{thm:irreducibility}
  Let $X \subseteq \Affine^n$ be an
  affine variety. Then the following are equivalent:
  \begin{enumerate}
    \item $X$ is irreducible;
    \item $I(X) \le k[x_1, \dots, x_n]$
      is a prime ideal;
    \item the coordinate ring $A(X)$
      is an integral domain.
  \end{enumerate}
\end{theorem}

\begin{proof}
  $(2 \Leftrightarrow 3)$ Use that
  $A(X) \cong k[x_1, \dots, x_n] / I(X)$.

  $(1 \Leftrightarrow 2)$
  We have
  $X$ is reducible if and only if
  $X = X_1 \cup X_2$ for some
  $X_1, X_2 \subsetneq X$ closed,
  if and only if
  \[
    X = V_X(f) \cup V_X(g)
    = V_X(fg)
  \]
  for some $f, g \in A(X)$ nonzero.
  This is equivalent to $fg = 0$
  for some $f, g \in A(X)$ nonzero, i.e.
  $A(X)$ is not an integral domain.
\end{proof}

\begin{example}
  We have the following:
  \begin{enumerate}
    \item $\Affine_k^n$ is irreducible
      as $A(\Affine_k^n) = k[x_1, \dots, x_n]$,
      which is an integral domain.
    \item A \emph{hypersurface}
      $X \subseteq \Affine_k^n$ is
      an affine variety with $I(X) = (f)$
      for some $f \in k[x_1, \dots, x_n]$.
      Then $A$ is irreducible if and only
      if $(f)$ is prime, if and only if
      $f$ is irreducible.\footnote{Note that any prime ideal is radical.}
  \end{enumerate}
\end{example}

\begin{remark}
  Theorem \ref{thm:irreducibility}
  implies that there is a bijection
  \[
    \{\text{irreducible subvarieties $Y \subseteq X$}\}
    \longleftrightarrow
    \{\text{prime ideals $\p \le A(X)$}\}.
  \]
\end{remark}

\begin{example}
  Let $f \in k[x_1, \dots, x_n]$
  be a nonzero nonunit. Write
  $f = f_1^{a_1} \cdots f_r^{a_r}$
  with $f_i$ irreducible and $a_i \in \Z_{> 0}$.
  Then
  \[
    V(f)
    = V(f_1^{a_1} \cdots f_r^{a_r})
    = V(f_1) \cup \cdots \cup V(f_r).
  \]
  Note $f_i$ irreducible
  implies $f_i$ is prime ($k[x_1, \dots, x_n]$ is a UFD),
  so $(f_i)$ is prime and $V(f_i)$
  is irreducible.
\end{example}

\begin{remark}
  We want a unique decomposition
  into irreducibles for arbitrary
  varieties. For this, we need
  the notion of Noetherianity.
\end{remark}

\begin{definition}
  A topological space $X$ is
  \emph{Noetherian} if there is no
  infinite chain of closed subsets
  \[
    X \supseteq X_0
    \supsetneq X_1 \supsetneq X_2
    \supsetneq \cdots.
  \]
\end{definition}

\begin{example}
  The spaces $\C^n$ and $\R^n$ are
  not Noetherian, e.g.
  take $X = \R^n$ and $X_n = [0, 1 / n]$.
\end{example}

\begin{lemma}
  Any affine variety $X \subseteq \Affine^n_k$
  is Noetherian.
\end{lemma}

\begin{proof}
  An infinite chain of closed subsets
  \[
    X
    \supsetneq X_1 \supsetneq X_2
    \supsetneq \cdots
  \]
  gives an infinite chain of
  (radical) ideals
  \[
    (0) = I_X(X)
    \subsetneq I_X(X_1)
    \subsetneq I_X(X_2)
    \subsetneq \cdots
  \]
  of $A(X)$. As
  $A(X) \cong k[x_1, \dots, x_n] / I(X)$ is
  a Noetherian ring, no such chain
  exists.
\end{proof}

\begin{example}
  For $X = \Affine^n_k$, we have a chain
  \[
    V(x_1) \supsetneq V(x_1, x_2)
    \supsetneq \cdots
    \supsetneq V(x_1, \dots, x_n),
  \]
  but such a chain stops.
\end{example}

\begin{theorem}
  If $X$ is a Noetherian topological
  space, then there exists a decomposition
  \[
    X = X_1 \cup \cdots \cup X_r
  \]
  with each $X_i$ an irreducible closed
  subset and
  $X_i \not\subseteq X_j$ for $i \ne j$.
  Furthermore, this decomposition is unique up to
  reordering.
\end{theorem}

\begin{proof}
  First we show existence. If $X$ is
  not irreducible, then write
  \[
    X = X_1 \cup X_1'
  \]
  with $X_1, X_1' \subsetneq X$ closed.
  If $X_1$ or $X_1'$ is not irreducible
  (say $X_1$), write
  \[
    X_1 = X_2 \cup X_2'
  \]
  with $X_2, X_2' \subsetneq X_1$ closed.
  If this process fails to terminate, then
  we get an infinite chain
  \[
    X_1 \supsetneq X_2 \supsetneq \cdots,
  \]
  contradicting Noetherianity.

  For uniqueness, say
  $X = X_1 \cup \cdots \cup X_r = X_1' \cup \cdots \cup X_s'$
  with $X_i, X_j'$ closed and irreducible.
  Fix $1 \le i \le r$. Since
  $X_i \subseteq \bigcup_{j = 1}^s X_j'$,
  we have
  \[
    X_i = (X_i \cap X_1') \cup \cdots \cup (X_i \cap X_s').
  \]
  As $X_i$ is irreducible, we have
  $X_i = X_i \cap X_j'$ for some $j$,
  so $X_i \subseteq X_j'$. By symmetry,
  $X_j' \subseteq X_k$ for some $k$.
  So $X_i \subseteq X_j' \subseteq X_k$,
  which implies $X_i = X_j' = X_k$.
  Hence $X_i$ is an $X_j'$ and vice versa.
\end{proof}

\begin{remark}
  If $X$ is an affine variety, then
  there exists a bijection
  \[
    \{\text{irreducible components of $X$}\}
    \longleftrightarrow
    \{\text{minimal prime ideals in $A(X)$}\}.
  \]
\end{remark}

\begin{remark}
  The primary decomposition implies that
  an ideal $I \subseteq k[x_1, \dots, x_n]$
  can be written as
  \[
    I = Q_1 \cap \cdots \cap Q_n
  \]
  with $Q_i$ primary (so that
  $P_i := \sqrt{Q_i}$ is prime).\footnote{Recall that $\p$ is \emph{primary} if $fg \in \p$ implies $f \in \p$ or $g^n \in \p$ for some $n > 0$.}
  Then
  \[
    X := V(I)
    = V(Q_1 \cap \cdots \cap Q_n)
    = V(Q_1) \cup \cdots \cup V(Q_n)
    = V(P_1) \cup \cdots \cup V(P_n),
  \]
  though this decomposition is not
  necessarily minimal.
\end{remark}

\begin{example}
  Let $I = (x^2, xy) \le k[x, y]$.
  Then $I = (x) \cap (x^2, xy, y^2)$, and
  \[
    V(I) = V(x) \cup V(x, y) = V(x).
  \]
\end{example}

\section{Dimension}

\begin{example}
  The motivating example is the following:
  Consider
  \[
    \Affine^n_k
    \supseteq V(0)
    \supsetneq V(x_1)
    \supsetneq V(x_1, x_2)
    \supsetneq \cdots
    \supsetneq V(x_1, \dots, x_n).
  \]
  This is a length $n$ chain and there
  are no irreducible
  subspaces
  $V(x_1, \dots, x_i) \supsetneq Y \supsetneq V(x_1, \dots, x_{i + 1})$.
\end{example}

\begin{definition}
  Let $X$ be a nonempty topological space.
  \begin{itemize}
    \item The \emph{dimension} of $X$,
      denoted $\dim X$, is the supremum of
      the $n$ such that there exists a
      chain of irreducible closed subspaces
      \[X \supseteq X_0 \supsetneq X_1 \supsetneq \cdots \supsetneq X_n \ne \varnothing.\]
    \item For $Y \subseteq X$ closed
      and irreducible, the
      \emph{codimension} of $Y$ in $X$,
      denoted $\codim_X Y$, is the supremum
      of the $n$ as above such that
      $X_n = Y$.
  \end{itemize}
\end{definition}

\begin{example}
  We have $\dim \Affine^1_k = 1$,
  as all maximal chains are of the form
  $\Affine^1_k \supsetneq \{p\}$.
\end{example}

\begin{example}
  We clearly have $\dim \Affine^n_k \ge n$.
\end{example}

\begin{remark}[Properties of dimension]
  We have the following:
  \begin{enumerate}
    \item $\dim_X = \sup\{\codim_X\{a\} : a \in X\}$.
    \item If $X$ is a Noetherian
      topological space with irreducible
      decomposition
      \[
        X = X_1 \cup \dots \cup X_r,
      \]
      then $\dim X = \max\{\dim X_1, \dots, \dim X_r\}$.

      Check $\ge$ as an exercise.
      To see $\le$, choose a chain of
      irreducible subspaces
      \[
        X \supseteq Y_n \supsetneq
        \dots \supsetneq Y_0.
      \]
      Then $Y_n = (Y_n \cap X_1) \cup \cdots \cup (Y_n \cap X_r)$,
      so $Y_n \subseteq X_i$ for some
      $i$ (as $Y_n$ is irreducible).
      \[
        X_i \supsetneq Y_n \supsetneq \dots \supsetneq Y_0,
      \]
      hence $\dim X \le \max \dim X_i$.
  \end{enumerate}
\end{remark}
