\documentclass[12pt, letterpaper, oneside]{book}
\usepackage[margin={0.6in, 0.75in}]{geometry}
\usepackage{microtype}
% \usepackage{kpfonts}
\usepackage{amsmath, amssymb, amsthm}
\usepackage{parskip}
\usepackage[many]{tcolorbox}
\usepackage{footnote}
\usepackage{cancel}
\usepackage{titlesec}
\usepackage{pgffor}
\usepackage[shortlabels, inline]{enumitem}
\usepackage{hyperref}
\usepackage{tikz-cd}

\usepackage[overload]{textcase}
\usepackage{graphicx}

\renewcommand{\chaptername}{Lecture}
\newtheorem{axiom}{Axiom}[chapter]
\newtheorem{theorem}{Theorem}[chapter]
\newtheorem{prop}{Proposition}[chapter]
\newtheorem{corollary}{Corollary}[theorem]
\newtheorem{lemma}{Lemma}[chapter]
\newtheorem{conjecture}{Conjecture}[theorem]
\theoremstyle{definition}
\newtheorem{definition}{Definition}[chapter]
\newtheorem{exercise}{Exercise}[chapter]
\newtheorem{example}{Example}[definition]
\newtheorem*{remark}{Remark}

\tcbset{sharp corners, breakable, enhanced, parbox=false}
\newtcolorbox{mybox}[3][]
{
  colframe = #2!150,
  colback  = #2!5,
  coltitle = #2!0!white,  
  title    = {#3},
  #1,
}

\titleformat{\chapter}[display]
    {\normalfont\huge\bfseries}{\chaptertitlename\ \thechapter}{20pt}{\Huge}
\titlespacing*{\chapter}{0pt}{0pt}{40pt}

\newcommand{\R}{\mathbb{R}}
\newcommand{\N}{\mathbb{N}}
\newcommand{\Z}{\mathbb{Z}}
\newcommand{\C}{\mathbb{C}}
\newcommand{\Q}{\mathbb{Q}}
\newcommand{\F}{\mathbb{F}}
\newcommand{\PP}{\mathbb{P}}
\newcommand{\Affine}{\mathbb{A}}
\newcommand{\OO}{\mathcal{O}}
\newcommand{\p}{\mathfrak{p}}
\newcommand{\q}{\mathfrak{q}}
\newcommand{\m}{\mathfrak{m}}
\newcommand{\Mod}[1]{\ {\mathrm{mod}\ #1}}
\newcommand{\Pmod}[1]{\ (\mathrm{mod}\ #1)}

\newcommand{\T}{\mathcal{T}}
\newcommand{\B}{\mathcal{B}}

\DeclareMathOperator{\vol}{vol}
\DeclareMathOperator{\Int}{int}
\DeclareMathOperator{\area}{area}
\DeclareMathOperator{\id}{id}
\DeclareMathOperator{\ord}{ord}
\DeclareMathOperator{\lcm}{lcm}
\DeclareMathOperator{\Gal}{Gal}
\DeclareMathOperator{\re}{Re}
\DeclareMathOperator{\im}{Im}
\DeclareMathOperator{\GL}{GL}
\DeclareMathOperator{\Spec}{Spec}
\DeclareMathOperator{\Tr}{Tr}
\DeclareMathOperator{\codim}{codim}
\DeclareMathOperator{\height}{ht}
\DeclareMathOperator{\trdeg}{tr{.}deg}
\DeclareMathOperator{\Frac}{Frac}
\DeclareMathOperator{\Open}{Open}
\DeclareMathOperator{\Rings}{Rings}
\DeclareMathOperator{\Hom}{Hom}
\DeclareMathOperator{\Char}{char}
\DeclareMathOperator{\Aut}{Aut}
\DeclareMathOperator{\PGL}{PGL}

\title{MATH 6421: Algebraic Geometry I}
\author{Frank Qiang\\Instructor: Harold Blum}
\date{Georgia Institute of Technology\\Fall 2025}

\begin{document}
  \maketitle

  \begingroup
  \let\cleardoublepage\clearpage
  \tableofcontents
  \endgroup

  % \foreach \i in {00, 01, 02, 03, 04, ..., 50} {%
  %   \edef\FileName{lectures/lecture\i.tex}%     The % here are necessary to eliminate any
  %   \IfFileExists{\FileName}{%  spurious spaces that may get inserted
  %      \input{\FileName}%       at these points
  %   }
  % }
  \input{lectures/lecture01.tex}
  \chapter{Aug.~21 --- Hilbert's Nullstellensatz}

\section{Applications of Hilbert's Nullstellensatz}

\begin{corollary}[Weak nullstellensatz]
  If $J \le k[x_1, \dots, x_n]$ is an
  ideal with $J \ne (1)$, then
  $V(J) \ne \varnothing$. Equivalently,
  if $f_1, \dots, f_r \in k[x_1, \dots, x_n]$
  have no common zeros, then there
  exist $g_1, \dots, g_r \in k[x_1, \dots, x_n]$
  such that $\sum_{i = 1}^r f_i g_i = 1$.
\end{corollary}

\begin{proof}
  Assume otherwise that $V(J) = \varnothing$.
  Then
  $I(V(J)) = I(\varnothing) = (1)$,
  so by Hilbert's nullstellensatz,
  we have $\sqrt{J} = (1)$. Then
  $1^n \in J$ for some $n > 0$, so
  $1 \in J$, i.e. $J = (1)$.
\end{proof}

\begin{remark}
  We need
  $k$ to be algebraically closed.
  Note that $(1) \ne (x^2 + 1) \le \R[x]$
  but $V(x^2 + 1) = \varnothing$.
\end{remark}

\begin{corollary}
  There is an inclusion-reversing bijection
  between radical ideals $J \le k[x_1, \dots, x_n]$ and
  affine varieties $X \subseteq \Affine_k^n$
  given by $J \mapsto V(J)$ with inverse $X \mapsto I(X)$.
\end{corollary}

\begin{proof}
  It suffices to show that these maps
  are inverses. For $J \le k[x_1, \dots, x_n]$
  a radical ideal, we have
  \[
    I(V(J)) = \sqrt{J} = J
  \]
  by Hilbert's nullstellensatz. For
  $X \subseteq \Affine_k^n$ an affine
  variety, we clearly have
  $X \subseteq V(I(X)) X$. For the
  reverse inclusion, choose
  an ideal $J \le k[x_1, \dots, x_n]$
  such that $V(J) = X$. Then
  $J \subseteq I(X)$, so we have
  $V(I(X)) \subseteq V(J) = X$.
  Thus we also get $V(I(X)) = X$.
\end{proof}

\begin{remark}
  This implies that maximal ideals in
  $k[x_1, \dots, x_n]$ correspond to
  points in $\Affine_k^n$, since
  maximal ideals correspond to minimal
  varieties under this bijection.
\end{remark}

\begin{corollary}
  If $X_1, X_2$ are affine varieties
  in $\Affine_k^n$, then
  \begin{enumerate}
    \item $I(X_1 \cup X_2) = I(X_1) \cap I(X_2)$;
    \item $I(X_1 \cap X_2) = \sqrt{I(X_1) + I(X_2)}$.
  \end{enumerate}
\end{corollary}

\begin{proof}
  (1) This follows from definitions.

  (2) Write $I(X_1 \cap X_2) = I(V(I(X_1)) \cap V(I(X_2))) = I(V(I(X_1) + I(X_2))) = \sqrt{I(X_1) + I(X_2)}$.
\end{proof}

\begin{example}
  The radical in (2) is necessary. Consider
  $X_1 = V(y)$ and $X_2 = V(y - x^2)$
  in $\Affine_k^2$. Then
  $X_1 \cap X_2 = \{(0, 0)\} \subseteq \Affine_k^2$,
  so $I(X_1 \cap X_2) = (x, y)$.
  However, $I(X_1) + I(X_2) = (y) + (y - x^2) = (y, x^2)$.

  Note that it is sometimes better to
  consider $(y, x^2)$ anyway as
  it tracks multiplicities.
  In particular, we can see the multiplicity
  in the dimension of
  $k[x, y] / (x, y^2) \cong \overline{1} k \oplus \overline{y} k$
  as a $k$-vector space.
\end{example}

\section{Proof of Hilbert's Nullstellensatz}

We will assume the following result
from commutative algebra without proof:
\begin{theorem}[Noether normalization]
  Let $A$ be a finitely generated algebra
  over a field $k$ with $A$ a domain.
  Then there is an injective $k$-algebra
  homomorphism
  $k[z_1, \dots, z_n] \hookrightarrow A$
  that is finite, i.e. $A$ is a
  finitely generated
  $k[z_1, \dots, z_n]$-module.
\end{theorem}

\begin{corollary}
  \label{cor:finite-extension}
  If $K \subseteq L$ is a field extension
  and $L$ is a finitely generated $K$-algebra,
  then $K \subseteq L$ is a finite field
  extension. In particular,
  if in addition $K = \overline{K}$, then
  $K = L$.
\end{corollary}

\begin{proof}
  By Noether normalization,
  there exists a $k$-algebra homomorphism
  $K[z_1, \dots, z_n] \to L$
  that is finite. Then by a result from
  commutative algebra, $L$ is
  integral over $K[z_1, \dots, z_n]$,
  which implies that $K[z_1, \dots, z_n]$
  must also be a field since $L$ is.
  Thus $n = 0$, so $K \subseteq L$ is
  a finite extension.
\end{proof}

\begin{prop}\label{prop:in-maximal-ideal}
  If $(1) \ne J \le R$ is an ideal, then
  $J$ is contained in some maximal ideal.
\end{prop}

\begin{proof}
  Consider the set $P = \{I \le R : J \subseteq I, I \ne (1)\}$
  with the partial order given by inclusion.
  Note that $P \ne \varnothing$ since
  $J \in P$. Furthermore,
  every chain in $P$ has an upper bound
  (for $\{I_\alpha : \alpha \in A\}$ a
  chain $P$, we can take $\bigcup_{\alpha \in A} I_\alpha$, which one can
  check is indeed
  an ideal that lies in $P$; note
  that $1 \notin I_\alpha$ implies
  $1 \notin \bigcup_{\alpha \in A} I_\alpha$).
  So Zorn's lemma implies there is a
  maximal element in $P$, which is a
  maximal ideal.
\end{proof}

\begin{proof}[Proof of Theorem \ref{thm:hilbert-nullstellensatz}]
  We will proceed in the following
  steps:
  \begin{enumerate}
    \item Show that the maximal ideals
      of $k[x_1, \dots, x_n]$ are
      of the form $(x_1 - a_1, \dots, x_n - a_n)$
      for $a_i \in k$.
    \item Prove the weak nullstellensatz:
      If $1 \ne J \le k[x_1, \dots, x_n]$,
      is an ideal, then $V(J) \ne \varnothing$.
    \item Prove the (strong) nullstellensatz:
      $I(V(J)) = \sqrt{J}$
      for $J \le k[x_1, \dots, x_n]$.
  \end{enumerate}
  The most difficult part is the first step
  and is where we need $k$ to be
  algebraically closed.\footnote{The statement is false when $k$ is not algebraically closed: $(x^2 + 1)$ is maximal in $\R[x]$.}

  (1) For $a_1, \dots, a_n \in k$, the
  ideal $(x_1 - a_1, \dots, x_n - a_n)$
  is maximal (the quotient is $k$, which
  is a field). Conversely, fix a maximal
  ideal $\m \in k[x_1, \dots, x_n]$.
  Since
  \[
    k \overset{\phi}{\longrightarrow}
    k[x_1, \dots, x_n] / \m = L
  \]
  is a finitely generated $k$-algebra and
  $k$ is algebraically closed,
  $\phi$ is an isomorphism by
  Corollary \ref{cor:finite-extension}.
  Choose $a_i \in k$ such that
  $\phi(a_i) = \overline{x}_i$, so
  $\overline{x_i - a_i} = 0$ in $L$
  Then $(x_1 - a_1, \dots, x_n - a_n) \subseteq \m$, so they
  must be equal since both the left and
  right hand sides are maximal ideals.

  (2) By Proposition \ref{prop:in-maximal-ideal},
  $J$ is contained in some maximal ideal
  $\m$. By (1), $\m = (x_1 - a_1, \dots, x_n - a_n)$
  for some $a_1, \dots, a_n \in k$.
  Since $J \subseteq \m$, we have
  $V(J) \supseteq V(\m) \supseteq \{(a_1, \dots, a_n)\}$, so
  $J \ne \varnothing$.

  $(3)$ The reverse inclusion follows
  from definitions. For the forward
  inclusion, fix $f \in I(V(J))$, and
  we want to show that $f^n \in J$ for
  some $n > 0$. Add a new variable $y$
  and consider
  \[
    J_1 = (J, fy - 1) \le k[x_1, \dots, x_n, y].
  \]
  Now $V(J_1) = \{(a, b) = (a_1, \dots, a_n, b) \in \Affine_k^{n + 1} : a \in V(J), f(a) b = 1\} = \varnothing$
  since $f$ vanishes on $V(J)$, so
  $f(a) b = 0$ for any $b$. Thus by the
  weak nullstellensatz, $J_1 = (1)$, so
  $1 = \sum_{i = 1}^r g_i f_i + g_0(fy - 1)$
  with $f_1, \dots, f_r \in J$
  and $g_0, \dots, g_r \in k[x_1, \dots, x_n, y]$. Let $N$ be the maximal power
  of $y$ in the $g_i$. Multiplying by
  $f^N$, we get
  \[
    f^N = \sum_{i = 1}^r G_i(x_1, \dots, x_n, fy) f_i + G_0(x_1, \dots, x_n, fy)(fy - 1)
  \]
  with $G_i \in k[x_1, \dots, x_n, fy]$.
  So if we set $fy = 1$, then we have
  \[
    f^N = \sum_{i = 1}^r G_i(x_1, \dots, x_n, 1) f_i + 0 \in J,
  \]
  which gives $f \in \sqrt{J}$. To
  justify this substitution, we can
  consider the quotient
  $k[x_1, \dots, x_n, y] / (fy - 1)$.
  We have a map
  $k[x_1, \dots, x_n] \to k[x_1, \dots, x_n, y] / (fy - 1)$,
  which is injective since $(fy - 1)$
  does not lie in $k[x_1, \dots, x_n]$,
  so an equality in the quotient implies
  an equality in $k[x_1, \dots, x_n]$.
\end{proof}

  \input{lectures/lecture03.tex}
  \input{lectures/lecture04.tex}
  \input{lectures/lecture05.tex}
  \input{lectures/lecture06.tex}
  \input{lectures/lecture07.tex}
  \input{lectures/lecture08.tex}
  \input{lectures/lecture09.tex}
  \chapter{Sept.~18 --- Pre-varieties}

\section{More on Tensor Products}

\begin{prop}
  If $B$ and $C$ are $A$-algebras (i.e.
  there are ring homomorphisms
  $f : A \to B$ and $g : A \to C$ which
  give $a \cdot b := f(a) b$ and
  $a \cdot c = g(a) c$, then
  $B \otimes_A C$ is also an $A$-algebra
  with
  \[
    (b \otimes c) \cdot (b' \otimes c')
    := (bb') \otimes (cc')
  \]
  and ring homomorphism
  $A \to B \otimes_A C$ given by
  $a \mapsto a \otimes 1$ (equivalently,
  $1 \otimes a$).
\end{prop}

\begin{prop}
  $k[x_1, \dots, x_m] \otimes_k k[y_1, \dots, y_n] \cong k[x_1, \dots, x_m, y_1, \dots, y_n]$.
\end{prop}

\begin{prop}
  $(k[x_1, \dots, x_m] / I) \otimes_k (k[y_1, \dots, y_n] / J) \cong k[x_1, \dots, x_m, y_1, \dots, y_n) / \langle I, J \rangle$.
\end{prop}

\begin{proof}
  Set $R = k[x_1, \dots, x_m]$ and
  $S = k[y_1, \dots, y_n]$. We have a
  short exact sequence
  \[
    0 \longrightarrow
    I \longrightarrow R \longrightarrow R / I \longrightarrow 0.
  \]
  Applying the right exact
  functor $\otimes_k (S / J)$ (and
  vice versa with $J$ and $\otimes R$) gives
  an exact sequence
  \begin{center}
    \begin{tikzcd}
      & R \otimes_k J \ar[d]\\
      & R \otimes_k S \ar[d]\\
      I \otimes_k (S / J) \ar[r] & R \otimes_k (S / J) \ar[r] \ar[d] & (R / I) \otimes_k (S / J) \ar[r] & 0 \\
      & 0
    \end{tikzcd}
  \end{center}
  So we have
  \[
    (R / I) \otimes_k (S / J)
    \cong
    \frac{R \otimes_k (S / J)}
    {\im(I \otimes_k (S / J) \to R \otimes_k (S / J))}
    \cong \frac{R \otimes_k S}{I \otimes_k S + R \otimes_k J},
  \]
  which is the desired result since
  $I \otimes_k S + R \otimes_k J = \langle I, J \rangle$
  in $R \otimes_k S$.
\end{proof}

\begin{prop}[Milne]\label{prop:tensor-domain}
  Let $B$ and $C$ be finitely generated
  $k$-algebras with $k = \overline{k}$.
  \begin{enumerate}
    \item If $B$ and $C$ are
      reduced, then so is $B \otimes_k C$.
    \item If $B$ and $C$ are domains,
      then so is $B \otimes_k C$.
  \end{enumerate}
\end{prop}

\begin{remark}
  We need $k = \overline{k}$ in
  Proposition \ref{prop:tensor-domain}.
  Consider the domains
  $\R[x] / (x^2 + 1)$,
  $\R[y] / (y^2 + 1)$. Then
  \[
    \frac{\R[x]}{(x^2 + 1)}
    \otimes_{\R}
    \frac{\R[y]}{(y^2 + 1)}
    \cong
    \frac{\R[x, y]}{(x^2 + 1, y^2 + 1)},
  \]
  which is not a domain since
  $(\overline{x - y})(\overline{x + y}) = \overline{x^2 - y^2} = \overline{-1 - (-1)} = 0$.
\end{remark}

\begin{corollary}
  If $X \subseteq \Affine^m$ and
  $Y \subseteq \Affine^n$ are affine
  varieties, then
  \begin{enumerate}
    \item $I(X \times Y) = \langle I(X), I(Y) \rangle \subseteq k[x_1, \dots, x_m, y_1, \dots, y_n]$.
    \item $A(X \times Y) \cong A(X) \otimes_k A(Y)$.
    \item If $X$ and $Y$ are irreducible,
      then $X \times Y$ is irreducible.
  \end{enumerate}
\end{corollary}

\begin{proof}
  Observe that
  $V(I(X), I(Y)) = X \times Y \subseteq \Affine^{m + n}$, so
  $I(X \times Y) = \sqrt{\langle I(X), I(Y) \rangle}$.
  Now we know that
  $I(X)$ and $I(Y)$ are radical in
  $k[x_1, \dots, x_m]$ and
  $k[y_1, \dots, y_n]$, respectively, so
  \[
    \frac{k[x_1, \dots, x_m]}{I(X)}
    \quad \text{and} \quad
    \frac{k[y_1, \dots, y_n]}{I(Y)}
  \]
  are reduced. By Proposition \ref{prop:tensor-domain},
  we get that
  \[
    \frac{k[x_1, \dots, x_m, y_1, \dots, y_n]}{\langle I(X), I(Y) \rangle}
    \cong
    \frac{k[x_1, \dots, x_m]}{I(X)}
    \otimes_k
    \frac{k[y_1, \dots, y_n]}{I(Y)}
  \]
  is reduced, so
  $\langle I(X), I(Y) \rangle$ is radical.
  Thus $I(X \times Y) = \langle I(X), I(Y) \rangle$, so
  $(1)$ holds.

  Now $(1)$ implies $(2)$, and
  $(3)$ follows since
  $X$ and $Y$ being irreducible implies
  $A(X)$ and $A(Y)$ are domains, which
  implies $A(X \times Y)$ is a domain
  by Proposition \ref{prop:tensor-domain}
  and $(2)$, so
  $X \times Y$ is irreducible.
\end{proof}

\section{Pre-varieties}

\begin{remark}
  We will now head towards defining
  a \emph{variety}, which is roughly
  finitely many affine varieties glued
  together (a \emph{pre-variety}) with a separation
  condition (an algebraic version of
  Hausdorffness).
\end{remark}

\begin{definition}
  A \emph{pre-variety} is a ringed
  space $(X, \OO_X)$ such that
  there exists a finite open cover
  $X = \bigcup_{i = 1}^s U_i$ with
  $(U_i, \OO_X|_{U_i})$ being an
  affine variety for all $i = 1, \dots, s$.
  A \emph{morphism} of pre-varieties
  \[
    f : (X, \OO_X) \longrightarrow (Y, \OO_Y)
  \]
  is a morphism of the ringed spaces.
  We will often just write
  $X$ for $(X, \OO_X)$.
\end{definition}

\begin{remark}
  We call $\varphi \in \OO_X(U)$ with
  $U \subseteq X$ open and
  $\varphi : U \to k$ a \emph{regular function}
  on $U$.
\end{remark}

\begin{example}
  Consider the following:
  \begin{enumerate}
    \item An affine variety $X$ is a
      pre-variety.
      However, we have multiple
      choices for the open cover:
      We can take $X = X$, or
      $X = \bigcup_{i = 1}^s D(f_i)$
      with $f_i \in \OO_X(X)$ and
      $(f_1, \dots, f_s) = (1)$
      in $\OO_X(X)$.
    \item $\PP^n_k = (\Affine^{n + 1} \setminus \{0\}) / k^\times$ is
      a pre-variety. We will see that
      $\PP^1_k = \Affine^1_k \cup \{\mathrm{pt}\}$.
    \item Let $X = V(I) \subseteq \Affine^n$
      be an affine variety and $U \subseteq X$ open.
      Set
      \[
        \OO_U(V) = \{\varphi : V \to k \mid \varphi \text{ is regular}\}.
      \]
      Then $(U, \OO_U)$ is a pre-variety.
      To see this, note that
      $U = \bigcup_{f \in I(X \setminus U)} D(f)$.
      Since $U$ is Noetherian (hence is
      compact), we can find a finite subcover,
      so $U = \bigcup_{i = 1}^s D(f_i)$
      for some $f_i \in A(X)$.
    \item (Gluing) Let
      $X_1$ and $X_2$ be affine varieties,
      and $U_{1,2} \subseteq X_1$,
      $U_{2,1} \subseteq X_2$ open, with
      an isomorphism
      \[
        f : U_{1, 2} \longrightarrow U_{2, 1}.
      \]
      Then we get a pre-variety by
      setting
      $X = (X_1 \sqcup X_2) / {\sim}$,
      where
      $a \sim f(a)$ for all $a \in U_{1, 2}$,
      $f(a) \sim a$ for all
      $a \in U_{2, 1}$, and
      $b \sim b$ for all $b \in X_1 \sqcup X_2$.
      We have quotient maps
      \[
        j_1 : X_1 \longrightarrow X
        \quad \text{and} \quad
        j_2 : X_2 \longrightarrow X.
      \]
      Now $X$ is a topological space
      with the quotient topology, and 
      $j_1, j_2$ are open embeddings
      (i.e. have open images and are
      homeomorphisms onto their images).
      Define a sheaf of rings
      $\OO_X$ on $X$ by
      \[
        \OO_X(U)
        = \{\varphi : U \to k \mid j_1^* \varphi \in \OO_{X_1}(j^{-1}(U)) \text{ and } j_2^* \varphi \in \OO_{X_2}(j_2^{-1}(U))\}.
      \]
      One can check
      $X = j_1(X_1) \cup j_2(X_2)$ and
      $(j(X_i), \OO_X|_{j_i(X_i)}) \cong (X_i, \OO_{X_i})$,
      so $(X, \OO_X)$ is a pre-variety.
  \end{enumerate}
\end{example}

\begin{example}\label{ex:projective-bug-eyed-line}
  Consider $X_1 = \Affine^1_x$
  and $X_2 = \Affine^1_y$, with
  $U_{1, 2} = \Affine^1_x \setminus \{0\}$
  and $U_{2, 1} = \Affine^1_y \setminus \{0\}$. Define
  \begin{align*}
    f : U_{1, 2}
    &\longrightarrow U_{2, 1} \\
    x &\longmapsto 1 / x.
  \end{align*}
  Then we can
  take $\PP^1_k = (X_1 \sqcup X_2) / {\sim}$.
  What are the regular functions
  $\PP^1_k \to k$? We should
  get only the constant functions
  (When $k = \C$,
  $\PP^1_{\C}$ is compact, so a
  holomorphic function $f : \PP^1_{\C} \to \C$
  is bounded. By restricting to $X_1$,
  we get a bounded map
  $f : \C \to \C$, so $f$ is constant
  by Liouville's theorem).

  In general,  let $j_i : X_i \to \PP^1_k$
  be the quotient maps. Fix
  $\varphi \in \OO_{\PP^1}(\PP^1)$.
  Now
  \[
    \varphi|_{X_1} := j_1^* \varphi
    = \sum_{i \ge 0} a_i x^i \quad
    \text{and} \quad
    \varphi|_{X_2} := j_2^* \varphi
    = \sum_{i \ge 0} b_i y^i
  \]
  for some $a_i, b_i \in k$.
  They must agree on the overlap, so
  \[
    \sum_{i \ge 0} a_i x^i
    = \sum_{i \ge 0} b_i (1 / x)^i
  \]
  as functions on $\Affine^1 \setminus \{0\}$.
  Since $\OO_{\Affine^1}(\Affine^1 \setminus \{0\}) = k[x^{\pm 1}]$,
  we have $a_i = b_i = 0$ for
  $i > 0$ and $a_0 = b_0$ (since
  the powers of $x^{\pm 1}$ are $k$-linearly
  independent), so
  $\varphi$ is a constant function.

  If we instead took
  $f : U_{1, 2} \to U_{2, 1}$ to be
  $x \mapsto x$, then
  $X = (X_1 \sqcup X_2) / {\sim}$
  is the ``bug-eyed line'' with
  two points $0, 0'$ at the origin (this is
  the \emph{line with two origins} when
  $k = \R$, which is not Hausdorff.)
  Note that $X \setminus \{0, 0'\} \cong \Affine^1 \setminus \{0\}$.
  In our case, the bad property is
  that there exist two morphisms
  \[
    g_1, g_2 : \Affine^1 \longrightarrow X
  \]
  such that $g_1|_{\Affine^1 \setminus \{0\}} = g_2|_{\Affine^1 \setminus \{0\}}$
  and $g_1 \ne g_2$, i.e.
  ``limits are not unique'' on $X$.
  Note that a similar computation shows
  $\OO_X(X) \cong k[x]$, so in particular,
  $X \ncong \PP^1_k$.
\end{example}

  \input{lectures/lecture11.tex}
  \chapter{Sept.~25 --- Projective Varieties}

\section{Projective Space}

\begin{definition}
  Define \emph{projective $n$-space} over
  $k$ to be
  \[
    \PP^n_k = \PP^n
    = \text{1-dimensional subspaces of $k^{n + 1}$}
    = (k^{n + 1} \setminus \{0\}) / {\sim},
  \]
  where $(x_0, x_1, \dots, x_n) \sim (y_0, y_1, \dots, y_n)$
  if there exists $\lambda \in k^\times$
  such that
  $(x_0, \dots, x_n) = \lambda (y_0, \dots, y_n)$.
  We write $[x_0 : x_1 : \dots : x_n] \in \PP^n_k$
  for the equivalence class of
  $(x_0, x_1, \dots, x_n)$.
\end{definition}

\begin{example}
  For $n = 2$, we have
  $[1 : 0 : 2] = [1 / 2 : 0 : 1] \in \PP^2_k$
  when $\Char k \ne 2$.
\end{example}

\begin{remark}
  For $0 \le i \le n$, define
  $U_i = \{[x_0 : x_1 : \dots : x_n] \in \PP^n_k : x_i \ne 0\}$.
  Then
  \[
    \PP^n_k = \bigcup_{i = 0}^n U_i,
  \]
  and there exist bijective maps
  $f_i : U_i \to \Affine^n$ given by
  \[
    f_i([x_0 : \dots : x_n])
    = (x_0 / x_i, \dots, \widehat{x_i / x_i}, \dots, x_n / x_i),
  \]
  where $\widehat{x_i / x_i}$ means we omit
  $x_i / x_i$.
  For $i = 0$, the inverse is
  $f_0^{-1}(x_1, \dots, x_n) = [1 : x_1 : \dots : x_n]$.
\end{remark}

\begin{remark}
  Another way to think about $\PP^n$ is
  via points at $\infty$. Observe that
  \[
    \PP^n \setminus U_0
    = \{[0 : x_1 : \dots : x_n] \in \PP^n_k : (x_1, \dots, x_n) \in k^n \setminus \{0\}\}
    \cong \PP^{n - 1}.
  \]
  So $\PP^n = \Affine^n \sqcup \PP^{n - 1} = \Affine^n \sqcup \Affine^{n - 1} \cup \PP^{n - 2} = \dots = \Affine^n \sqcup \Affine^{n - 1} \sqcup \dots \sqcup \Affine^0$.
\end{remark}

\begin{remark}
  Why work with $\PP^n$? One motivation
  is analytic (e.g. for $k = \C$):
  \begin{enumerate}
    \item $\PP^n_\C$ is compact
      with the analytic topology:
      There are surjective continuous maps
      \begin{center}
        \begin{tikzcd}
          {\R^{2n + 2} \setminus \{0\}} \ar[r, two heads] & {\C \PP^n} \\
          {S^{2n + 1}} \ar[u, hook, swap] \ar[ru, two heads]
        \end{tikzcd}
      \end{center}
    \item \emph{Chow's theorem}:
      Any closed complex submanifold
      of $\C \PP^n$ is a projective
      variety.
  \end{enumerate}
  Another motivation is the extra data
  at $\infty$:
  \begin{enumerate}
    \item If $\ell_1, \ell_2$ are
      distinct lines in $\Affine^2$, then
      $\#(\ell_1 \cap \ell_2) = 0$ or $1$.
      However, over $\PP^2$,
      $\#(\ell_1 \cap \ell_2) = 1$ always.
    \item \emph{Bezout's theorem}: If
      $C_1, C_2 \subseteq \Affine^2$ are
      two distinct irreducible curves in
      $\Affine^2$, then
      \[
        \#(C_1 \cap C_2)
        \le (\deg C_1)(\deg C_2),
      \]
      counting multiplicities. The
      version over $\PP^2$ always gives
      equality.
  \end{enumerate}
\end{remark}

\section{Graded Rings}

\begin{remark}
  In projective space,
  for $f \in k[x_0, \dots, x_n]$,
  we could try to define
  \[
    V(f) = \{[a_0 : \dots : a_n] : f(a_0, \dots, a_n) = 0\}.
  \]
  But this is bad notation as it is
  not well-defined ($f = 0$ depends on
  the representative in the equivalence
  class). Instead, if $f$ is homogeneous
  of degree $d$, then
  \[
    f(\lambda a_0, \dots, \lambda a_n)
    = \lambda^d f(a_0, \dots, a_n),
  \]
  so $V(f)$ is well-defined in this case,
  when $f$ is homogeneous.
\end{remark}

\begin{definition}
  An \emph{$\N$-graded ring} is a ring
  $R$ with subgroups
  $R_d \subseteq R$ for $d \in \N$ such that
  \[
    R = \bigoplus_{d \in \N} R_d
    \quad\text{and}\quad
    R_d R_e \subseteq R_{d + e}.
  \]
  An element $f \in R$ is
  \emph{homogeneous} if there exists
  $d$ such that $f \in R_d$.
\end{definition}

\begin{example}
  For $S = k[x_0, \dots, x_n]$,
  we can take $S_d = \bigoplus_{a_i \ge 0, \sum a_i = d} k x_0^{a_0} \cdots x_n^{a_n}$.
\end{example}

\begin{definition}
  An ideal $I$ in a graded ring is \emph{homogeneous}
  if it is generated by homogeneous elements.
\end{definition}

\begin{example}
  We can write $(x, y^3 - 3x^2) \subseteq k[x, y]$
  as $(x, y^3)$, so it is homogeneous.
\end{example}

\begin{prop}
  Let $R$ be a graded ring with ideal
  $I$. The following are equivalent:
  \begin{enumerate}
    \item $I$ is homogeneous;
    \item for any $f = \sum_{d \in \N} f_d \in I$
      with $f_d \in R_d$, then
      $f_d \in I$ for all $d$;
    \item $I = \bigoplus_{d \in \N} (I \cap R_d)$.
  \end{enumerate}
\end{prop}

\begin{proof}
  Left as an exercise. The interesting
  implication is $(1 \Rightarrow 2)$.
\end{proof}

\begin{prop}
  Let $I, J$ be homogeneous ideals
  of a graded ring $R$. Then
  \begin{enumerate}
    \item $I + J$, $IJ$, $\sqrt{I}$, and
      $I \cap J$ are all homogeneous;
    \item $R / I$ is a graded ring
      with $R / I = \bigoplus_{d \in \N} R_d / I_d$, where
      $I_d = I \cap R_d$.
  \end{enumerate}
\end{prop}

\begin{proof}
  (1) We prove that $\sqrt{I}$ is
  homogeneous.
  Assume $f \in \sqrt{I}$, and write
  $f = f_0 + f_1 + \dots + f_d$ with
  $f_i \in R_i$ and $f_d \ne 0$. Now
  there exists $n > 0$ such that
  $f^n \in I$, and
  \[
    f^n = f^n_d + \text{lower order terms}.
  \]
  Since $I$ is homogeneous, $f_d^n \in I$,
  so $f_d \in \sqrt{I}$.
  Then $f_0 + \dots + f_{d - 1} \in \sqrt{I}$,
  and we can repeat.

  (2) We can write
  $R / I = (\bigoplus_{d \in \N} R_d) / (\bigoplus_{d \in \N} (I \cap R_d))$.
  As abelian groups, this is
  $R / I \cong \bigoplus_{d \in \N} R_d / I_d$.
  One can check that the multiplication
  also respects the grading, so
  this is an isomorphism of rings.
\end{proof}

\section{Projective Varieties}

\begin{definition}
  For a set $T \subseteq k[x_0, \dots, x_n]$ of
  homogeneous elements, define
  its \emph{vanishing locus}
  \[
    V_p(T) := V(T)
    = \{[x_0 : \dots : x_n] \in \PP^n : f(x) = 0 \text{ for all } f \in T\}
    \subseteq \PP^n.
  \]
  A \emph{projective variety}
  is a subset of this form.
  For a homogeneous ideal $I \le k[x_0, \dots, x_n]$,
  define
  \[
    V(I) = \{x \in \PP^n : f(x) \text{ for all } f \in I \text{ homogeneous}\}.
  \]
  For a subset $X \subseteq \PP^n$, define
  its \emph{ideal}
  \[
    I_p(X) := I(X)
    = ( f \in k[x_0, \dots, x_n] \text{ homogeneous} : f(x) = 0 \text{ for all } [x] \in X ).
  \]
  Note that we need to take the
  ideal generated by these
  elements, otherwise we may
  not get an ideal.
\end{definition}

\begin{remark}
  If $T \subseteq k[x_0, \dots, x_n]$
  is a subset of homogeneous elements,
  then we have $V_p(T) = V_p((T))$.
  So projective varieties can
  equivalently be defined as vanishing
  sets of homogeneous ideals.
\end{remark}

\begin{example}
  Consider $X = V_p(x^2 - yz) \subseteq \PP^{2}_{x : y : z}$.
  Set $H = V(x)$, then there is a bijection
  \begin{align*}
    U = \PP^2 \setminus H
    &\overset{f}{\longrightarrow} \Affine^2 \\
    [1 : y : z]
    &\longmapsto (y, z).
  \end{align*}
  Then $f(X \cap U) = V(1 - yz)$. On the
  other hand, we can see that
  \[
    X \cap H
    = \{[0 : 1 : 0], [0 : 0 : 1]\}
    = \{a, b\}.
  \]
  If we were working with $\C$ with
  the analytic topology, then we can
  take limits on $V(1 - yz)$ and see
  \[
    \lim_{t \to 0} [1 : t : 1 / t]
    = \lim_{t \to 0} [t : t^2 : 1]
    = [0 : 0 : 1] = b.
  \]
  Note that we essentially switched
  charts in order to take this limit.
  Similarly, we have
  \[
    \lim_{t \to \infty}
    [1 : t : 1 / t]
    = \lim_{t \to \infty} [1 / t : 1 : 1 / t^2]
    = [0 : 1 : 0] = a.
  \]
  So we can see $a, b$ as points at
  $\infty$ compactifying the curve
  $V(1 - yz)$.
\end{example}

\begin{example}
  We have the following:
  \begin{enumerate}
    \item $V_p(0) = \PP^n$;
    \item $V_p(1) = \varnothing$;
    \item if $p = [a_0 : \dots : a_n]$
      and
      $J = (a_i x_j - a_j x_i : 0 \le i, j \le n)$,
      then $V(J) = \{0\}$;
    \item $I_0 = (x_0, \dots, x_n)$
      is called the \emph{irrelevant ideal},
      which has
      $V_p(I_0) = \varnothing = V_p(1)$
      but $I_0 = \sqrt{I_0} \subsetneq (1)$.
  \end{enumerate}
\end{example}

  \chapter{Sept.~30 --- Projective Varieties, Part 2}

\section{More on Projective Varieties}

\begin{example}
  Consider $X = V(y^2 z - x^3 - z x^2 - z^3) \subseteq \PP^2$
  and $H_z = V(z)$. Let
  \begin{align*}
    U_z = \PP^2 \setminus H_z
    &\underset{\text{bij}}{\overset{f}{\longrightarrow}} \Affine^2 \\
    {[x:y:1]} &\longmapsto (x, y).
  \end{align*}
  Then $f(X \cap U_z) = V(y^2 - x^3 - x^2 - 1)$,
  and
  \[
    X \cap U_z
    = V(y^2 z - x^3 - z x^2 - z^3, z)
    = V(x^3, z)
    = \{[0 : 1 : 0]\}.
  \]
\end{example}

\begin{example}
  Let $I = (x_0, \dots, x_n)$
  be the irrelevant ideal.
  Then $I$ is radical, but
  \[I_p(V_p(I)) = I_p(\varnothing) = (1) \ne \sqrt{I}.\]
\end{example}

\section{Cones}
\begin{definition}
  A subset $C \subseteq \Affine^{n + 1}$
  is a \emph{cone} if $0 \in C$ and
  $\lambda x \in C$ whenever
  $x \in C$ and $\lambda \in k$.
\end{definition}

\begin{example}
  If $X \subseteq \PP^n$ is a projective
  variety, then we can set $C(X) = \pi^{-1}(X) \{0\}$, where
  \begin{align*}
    \pi :
    \Affine^{n + 1} \setminus \{0\}
    &\longrightarrow \PP^n \\
    x &\longmapsto [x].
  \end{align*}
\end{example}

\begin{prop}
  If $C \subseteq \Affine^{n + 1}$
  is a cone, then $I_a(C) \le k[x_0, \dots, x_n]$
  is homogeneous.
\end{prop}

\begin{proof}
  Fix $f \in I_a(C)$. Then we can
  write $f = \sum_{i = 0}^d f_i$
  with $f_i$ homogeneous of degree $i$.
  We want to show that
  $f_i \in I_a(C)$ for each $i$.
  Fix $x \in C$. For any
  $\lambda \in k$,
  \[
    0 = f(\lambda x)
    = \sum_{i = 0}^d \lambda^i f_i(x).
  \]
  Viewing this as a polynomial in
  $\lambda$ (with $x$ fixed),
  we must have each $f_i(x) = 0$.
  Thus $f_i \in I_a(C)$.
\end{proof}

\section{Projective Nullstellensatz}

\begin{theorem}[Projective Hilbert's Nullstellensatz]
  We have the following:
  \begin{enumerate}
    \item For a projective variety
      $X \subseteq \PP^n$,
      $V_p(I_p(X)) = X$.
    \item For a homogeneous ideal
      $J \le k[x_0, \dots, x_n]$
      with $\sqrt{J} \ne (x_0, \dots, x_n)$,
      $I_p(V_p(J)) = \sqrt{J}$.
  \end{enumerate}
  As a consequence, there is a bijection
  between projective varieties and
  radical homogeneous ideals
  of $k[x_0, \dots, x_n]$
  which are not equal to
  $(x_0, \dots, x_n)$, given by
  $X \mapsto I_p(X)$ with inverse
  $J \mapsto V_p(J)$.
\end{theorem}

\begin{proof}
  (1) This is similar to the affine case.

  (2) Fix
  a homogeneous ideal
  $(1) \ne J \le k[x_0, \dots, x_n]$
  such that $\sqrt{J} \ne (x_0, \dots, x_n)$
  (the theorem is clearly true for
  the unit ideal). Then observe
  that we can write
  \begin{align*}
    I_p(V_p(J))
    &= (f \in k[x_0, \dots, x_n] \text{ homogeneous} : f(x) = 0 \text{ for all } [x] \in V_p(J)) \\
    &= (f \in k[x_0, \dots, x_n] \text{ homogeneous} : f(x) = 0 \text{ for all } x \in V_a(J) \setminus \{0\}) \\
    &= (f \in k[x_0, \dots, x_n] : f(x) = 0 \text{ for all } x \in \overline{V_a(J) \setminus \{0\}}) \\
    &=
    \begin{cases}
      I_a(V_a(J)) & \text{if } V_a(J) \supsetneq \{0\}, \quad\quad \text{(A)} \\
      I_a(\varnothing) & \text{if } V_a(J) = \{0\}, \quad\quad \text{(B)}
    \end{cases}
  \end{align*}
  In Case A, we get that
  $I_p(V_p(J)) = I_a(V_a(J)) = \sqrt{J}$
  by the affine Nullstellensatz.
  In Case B, we have
  $V_a(J) = \{0\}$, so
  $\sqrt{J} = (x_0, \dots, x_n)$,
  which we assumed was not the case.
\end{proof}

\section{The Zariski Topology on \texorpdfstring{$\PP^n$}{Pn}}

\begin{remark}
  We have the following properties
  of $I_p$ and $V_p$:
  \begin{enumerate}
    \item For homogeneous
      ideals $J_i \le k[x_0, \dots, x_n]$
      for $i \in I$, we have
      $V_p(\sum_{i \in I} J_i) = \bigcap_{i \in I} V_p(J_i)$;

      If $I = \{1, 2\}$, then we have
      $V_p(J_1 J_2) = V_p(J_1) \cup V_p(J_2)$.
    \item If $X_1, X_2 \subseteq \PP^n$
      are projective varieties, then
      \[
        I_p(X_1 \cup X_2)
        = I_p(X_1) \cap I_p(X_2)
        \quad \text{and} \quad
        I_p(X_1 \cap X_2)
        = \sqrt{I_p(X_1) + I_p(X_2)},
      \]
      where we assume in the second
      equality that
      $X_1 \cap X_2 \ne \varnothing$.
  \end{enumerate}
  The proofs are similar to the
  affine case.
\end{remark}

\begin{example}
  Let $X_1 = V(x) \subseteq \PP^2$
  and $X_2 = V(y, z) \subseteq \PP^2$.
  Then $I(X_1 \cap X_1) = I(\varnothing) = (1)$,
  but we have
  $I(X_1) + I(X_2) = (x, y, z)$, which
  is already radical.
\end{example}

\begin{definition}
  The \emph{Zariski topology} on
  $\PP^n$ is the topology whose
  closed sets are projective
  varieties $X \subseteq \PP^n$
  (equivalently, the vanishing loci
  of homogeneous ideals).
\end{definition}

\begin{remark}
  This is a topology by the above
  properties of $I_p$ and $V_p$.
  We now want to relate this to
  the topology on our charts.
  Let $H_0 = V(x_0)$ and consider
  the bijection
  \begin{align*}
    \Affine^n
    &\overset{\rho_0}{\longrightarrow} \PP^n \setminus H_0 \\
    (x_1, \dots, x_n)
    &\longmapsto [1 : x_1 : \dots : x_n].
  \end{align*}
  We want to show that $\rho_0$ is a
  homeomorphism. Write
  $\Affine^n \subseteq \PP^n$.
  Consider the ring homomorphism
  \begin{align*}
    k[x_0, \dots, x_n]
    &\overset{\Phi}{\longrightarrow} k[x_1, \dots, x_n] \\
    f(x_0, \dots, x_n)
    &\longmapsto f(1, x_1, \dots, x_n) =: f^i
  \end{align*}
  We call $f^i$ the
  \emph{dehomogenization} of $f$.
\end{remark}

\begin{example}
  Let $f(x) = x_0 x_2^2 - x_1^3 - x_0 x_1^2 - x_0^3$,
  then
  $f^i(x) = x_2^2 - x_1^3 - x_1^2 - 1$.
\end{example}

\begin{definition}
  If $J \le k[x_0, \dots, x_n]$
  is homogeneous, then define
  its \emph{dehomogenization} to be
  \[
    J^i = (f^i : f \in J)
    = \Phi(J).
  \]
\end{definition}

\begin{prop}
  For $J \le k[x_0, \dots, x_n]$
  homogeneous,
  $V_p(J) \cap \Affine^n = V_a(J^i)$.
\end{prop}

\begin{proof}
  The idea is to
  use that for
  $[1 : x_1 : \dots : x_n] \in \PP^n$
  and $f \in k[x_0, \dots, x_n]$
  homogeneous, we have
  $f([1 : x]) = 0$ if and only if
  $f^i(x) = 0$. Fill in the details
  as an exercise.
\end{proof}

\begin{definition}
  If $f \in k[x_1, \dots, x_n]$
  with $\deg f  = d$, then define
  its \emph{homogenization} to be
  \[
    f^h = x_0^d f(x_1 / x_0, \dots, x_n / x_0) \in k[x_0, x_1, \dots, x_n],
  \]
  which is homogeneous of degree $d$.
\end{definition}

\begin{example}
  Let $f = x_2^2 - x_1^3 - x_1^2 - 1$.
  Then we have
  \[
    f^h = x_0^3 ((x_2 / x_0)^2 - (x_1 / x_0)^3 - (x_1 / x_0)^2 - 1)
    = x_0 x_2^2 - x_1^3 - x_0 x_1^2 - x_0^3.
  \]
\end{example}

\begin{remark}
  While $f^h g^h = (fg)^h$, note that
  $(f + g)^h \ne f^h + g^h$ in general.
\end{remark}

\begin{definition}
  For $J \le k[x_1, \dots, x_n]$ an ideal,
  define its \emph{homogenization} to be
  \[
    J^h = (f^h : f \in J).
  \]
\end{definition}

\begin{prop}
  For $J \le k[x_1, \dots, x_n]$ an ideal,
  $V_a(J) = V_p(J^h) \cap \Affine^n$.
\end{prop}

\begin{proof}
  Left as an exercise,
  use that
  $f(a_1, \dots, a_n) = 0$ if and only
  if $f^h(1, a_1, \dots, a_n) = 0$.
\end{proof}

\begin{remark}
  The above results imply that
  $\rho_0 : \Affine^n \to \PP^n \setminus H_0$
  is a homeomorphism.
\end{remark}

\begin{prop}[Projective closure]
  For $J \le k[x_1, \dots, x_n]$
  and $X = V_a(J) \subseteq \Affine^n \subseteq \PP^n$,
  we have
  \[
    \overline{X} = V_p(J^h).
  \]
\end{prop}

\begin{proof}
  See Gathmann.
\end{proof}

\begin{prop}
  If $X = V_a(f) \subseteq \Affine^n$ with $f \in k[x_1, \dots, x_n]$,
  then its projective closure
  in $\PP^n$ is
  \[
    \overline{X} = V(f^h).
  \]
\end{prop}

\begin{proof}
  Left as an exercise, the idea
  is to show that
  $(f)^h = (f^h)$.
\end{proof}

  \chapter{Oct.~9 --- Projective Space as Varieties}

\section{More on the Zariski Topology on \texorpdfstring{$\PP^n$}{Pn}}
\begin{prop}
  For each $0 \le i \le n$,
  the map
  \begin{align*}
    U_i = \PP^n \setminus V(x_i)
    &\overset{h_i}{\longrightarrow} \Affine^n \\
    [x_0 : \cdots : x_n]
    &\longmapsto
    (x_0 / x_i, \dots, \widehat{x_i / x_i}, \dots, x_n / x_i)
  \end{align*}
  is a homeomorphism.
\end{prop}

\begin{proof}
  The main inputs to the proof are
  \begin{itemize}
    \item For $I \le k[x_0, \dots, x_n]$
      homogeneous,
      $h_0(V(I) \cap U_0) = V(I^i)$.
    \item For $J \le k[x_1, \dots, x_n]$,
      $h_0^{-1}(V(J)) = V(J^h)$.
  \end{itemize}
  Fill in the remaining details as
  an exercise.
\end{proof}

\begin{prop}[Projective closure]\label{prop:projective-closure}
  For $J \le k[x_1, \dots, x_n]$
  and $X = V_a(J) \subseteq \Affine^n \subseteq \PP^n$,
  we have
  \[
    \overline{X} = V_p(J^h).
  \]
\end{prop}

\begin{proof}
  See Gathmann.
\end{proof}

\begin{prop}
  If $X = V_a(f) \subseteq \Affine^n$ with $f \in k[x_1, \dots, x_n]$,
  then its projective closure
  in $\PP^n$ is
  \[
    \overline{X} = V_p(f^h).
  \]
\end{prop}

\begin{proof}
  We know that
  $\overline{X} = V_p(\langle f \rangle^h)$
  by Proposition
  \ref{prop:projective-closure}.
  Now
  \[
    \langle f \rangle^h
    = \langle (fg)^h : g \in k[x_1, \dots, x_n] \rangle
    = \langle f^h g^h : g \in k[x_1, \dots, x_n] \rangle
    = \langle f^h \rangle,
  \]
  which implies the desired result.
\end{proof}

\begin{example}[Twisted cubic]
  Take $X = \im(\Affine^1 \to \Affine^3 : t \mapsto (t, t^2, t^3))$.
  Note that $X \cong \Affine^1$, and
  \[
    I_a(X)
    = (x^2 - y, x^3 - z)
    = (x^2 - y, x^3 - z, xy - z).
  \]
  Then one can check
  that
  $\overline{X} \subseteq \PP^3_{w : x : y : z}$
  is given by
  $\overline{X} = V_p(x^2 - yw, x^3 - zw^2, xy - zw)$.
  However, one can also check that
  $\overline{X}$ cannot be
  cut out by $2$ equations.
  For example,
  \[
    V_p(x^2 - yw, x^3 - zw^2)
    = \overline{X} \cup V(w, x).
  \]
\end{example}

\section{Projective Space as Varieties}

\begin{remark}
  Our goal now is to show that
  projective varieties are varieties.
  The first step is to define a sheaf
  of regular functions on $\PP^n$.
\end{remark}

\begin{definition}
  Let $U$ be an open set of a
  projective variety $X \subseteq \PP^n$.
  A function $\varphi : U \to k$
  is \emph{regular} if for every $p \in U$,
  there exists $d \in \N$,
  $f, g \in k[x_0, \dots, x_n]$
  homogeneous of degree $d$, and
  $U_p \subseteq U$ open such that
  \[
    \varphi(x) = \frac{f(x)}{g(x)}
    \quad \text{for all } x \in U_p.
  \]
\end{definition}

\begin{remark}
  If $X \subseteq \PP^n$ is a projective
  variety, then
  \[
    \OO_X(U)
    = \{\varphi : U \to k \mid \varphi \text{ is regular}\}
  \]
  is a sheaf of rings on $X$.
  Again this is because the regular
  condition can be checked locally.
\end{remark}

\begin{prop}
  If $X \subseteq \PP^n$ is a projective
  variety, then
  $(X, \OO_X)$ is a pre-variety.
\end{prop}

\begin{proof}
  Let $X_i = X \cap (\PP^n \setminus V(x_i))$.
  It suffices to show
  $(X_i, \OO_X|_{X_i})$ is an affine
  variety for each $0 \le i \le n$.
  For simplicity, assume $i = 0$.
  Let $J = I(X) \le k[x_0, \dots, x_n]$
  and $Z_0 = V(J^i) \subseteq \Affine^n$.
  We have seen before that we have
  a homeomorphism
  \begin{align*}
    X_0 &\overset{F}{\longrightarrow} Z_0 \\
    [x_0 : \cdots : x_n]
        &\longmapsto (x_1 / x_0, \dots, x_n / x_0).
  \end{align*}
  We claim that $F$ induces an
  isomorphism of ringed spaces
  $(X_0, \OO_X|_{X_0}) \cong (Z_0, \OO_{Z_0})$.
  To see this, we need to check
  that regular functions pull back to
  regular functions via $F$ and
  $F^{-1}$. A regular function on an
  open set of $X_0$ is locally of
  the form
  \[
    \frac{f(x_0, \dots, x_n)}{g(x_0, \dots, x_n)}
  \]
  with $f, g$ homogeneous of the
  same degree. Now
  \[
    (F^{-1})^*
    \left(\frac{f(x_0, \dots, x_n)}{g(x_0, \dots, x_n)}\right)
    = \frac{f(1, x_1, \dots, x_n)}{g(1, x_1, \dots, x_n)},
  \]
  which is a fraction of polynomials
  and hence regular on $Z_0$.
  So $F^{-1}$ pulls regular functions
  back to regular functions. Conversely,
  a regular function on
  $Z_0$ is locally given by
  \[
    \frac{q(x_1, \dots, x_n)}{r(x_1, \dots, x_n)},
  \]
  and its pullback via $F$ is
  \[
    F^*\left(\frac{q(x_1, \dots, x_n)}{r(x_1, \dots, x_n)}\right)
    = \frac{q(x_1 / x_0, \dots, x_n / x_0)}{r(x_1 / x_0, \dots, x_n / x_0)}
    = \frac{x_0^d q(x_1 / x_0, \dots, x_n / x_0)}{x_0^d r(x_1 / x_0, \dots, x_n / x_0)},
  \]
  where $d = \max\{\deg q, \deg r\}$.
  This is regular on $X_0$, so
  $F$ also pulls regular functions
  back to regular functions.
  So we get an isomorphism of
  ringed spaces, as desired.
\end{proof}

\begin{example}
  $\PP^n$ is a pre-variety, and
  $\PP^n \setminus V(x_i) =: U_i \cong \Affine^n$
  as pre-varieties.
\end{example}

\begin{definition}
  A \emph{morphism}
  of projective varieties
  is a morphism of
  the underlying pre-varieties.
\end{definition}

\begin{remark}
  For a projective variety $X$, it will
  be convenient to work with
  ``global coordinates,'' i.e.
  \[
    S(X)
    := k[x_0, \dots, x_n] / I_p(X).
  \]
  This is called the
  \emph{homogeneous coordinate ring}.
  Note the following:
  \begin{enumerate}
    \item For $f \in S(X)$
      homogeneous, $f$ is not
      necessarily a well-defined
      function on $X$. But
      \[
        V(f) = \{[x] \in X : f(x) = 0\}
      \]
      is still well-defined.
    \item A relative version of the
      projective Nullstellensatz holds:
      There is a bijection
      \begin{align*}
        \left\{
          \substack{\displaystyle\text{projective subvarieties} \\ \displaystyle Y \subseteq X}
        \right\}
        &\longleftrightarrow
        \left\{
          \substack{\displaystyle\text{radical homogeneous ideals in $S(X)$} \\ \displaystyle \text{not equal to $(\overline{x}_1, \dots, \overline{x}_n)$}}
        \right\} \\
        Y &\longmapsto I(Y) \\
        V(J) &\mathrel{\reflectbox{\ensuremath{\longmapsto}}} J
      \end{align*}
  \end{enumerate}
  where $I(Y) = \langle f \in S(X) : f \text{ homogeneous and } f(y) = 0 \text{ for all } y \in Y \rangle$.
\end{remark}

\begin{lemma}
  If $X \subseteq \PP^n$ and
  $f_0, \dots, f_m \in S(X)$ are homogeneous
  of the same degree, then
  \begin{align*}
    U = X \setminus V(f_0, \dots, f_m)
    &\overset{f}{\longrightarrow} \PP^m \\
    [x_0 : \cdots : x_n]
    &\longmapsto [f_0(x) : \cdots : f_m(x)]
  \end{align*}
  is a morphism.
\end{lemma}

\begin{proof}
  To see that $f$ is well-defined,
  note that for $[a_0 : \cdots : a_n] \in X \setminus V(f_0, \dots, f_m)$,
  we have
  \[
    (f_0(\lambda a), \dots, f_m(\lambda a))
    = \lambda^d (f_0(a), \dots, f_m(a))
  \]
  with $d = \deg f_i$.
  So $[f_0(a) : \cdots : f_m(a)] \in \PP^m$
  is well-defined. To see that $f$ is a
  morphism, we check locally on
  $\PP^m$. Let
  $V_i = \PP^m \setminus V(x_i)$
  and $U_i = f^{-1}(V_i)$.
  Then
  \begin{align*}
    U_i
    &\overset{f|_{U_i}}{\longrightarrow} V_i \cong \Affine^m \\
    a &\longmapsto
    \left(
      \frac{f_0(a)}{f_i(a)},
      \dots,
      \widehat{\frac{f_i(a)}{f_i(a)}},
      \dots,
      \frac{f_m(a)}{f_i(a)}
    \right).
  \end{align*}
  Since each $f_j / f_i$ is regular,
  $f|_{U_i}$ is a morphism. So
  $f$ is a morphism.
\end{proof}

\begin{example}
  Define a map
  \begin{align*}
    \PP^1_{s : t}
    &\overset{f}{\longrightarrow}
    \PP^3_{x : y : z} \\
    [s : t]
    &\longmapsto
    [s^3 : s^2 t : s t^2 : t^3].
  \end{align*}
  Then $S(\PP^1) = k[s, t]$ and
  $f(\PP^1)$ is the projective
  twisted cubic in $\PP^3$.
\end{example}

\begin{example}
  Let $A \in \GL_{n + 1}(k)$.
  Then
  \begin{align*}
    f_A : \PP^n
    &\longrightarrow \PP^n \\
    [x] &\longmapsto [Ax]
  \end{align*}
  is an isomorphism with inverse
  $f_{A^{-1}}$. We will see later
  that we have a surjective group
  homomorphism
  \begin{align*}
    \GL_{n + 1}(k)
    &\longrightarrow \Aut(\PP^n) \\
    A &\longmapsto f_A
  \end{align*}
  with kernel $k^\times I$.
  So we get
  $\Aut(\PP^n) \cong \GL_{n + 1}(k) / k^\times I =: \PGL_{n + 1}(k)$.
\end{example}

\begin{example}[Conics]
  Let $f \in k[x, y, z]$ be homogeneous
  of degree $2$, and write
  \[
    f =
    (x, y, z) B (x, y, z)^T
  \]
  with $B$ a symmetric $3 \times 3$
  matrix. We want to characterize
  $X = V(f)$.
  Choose
  $A \in \GL_3(k)$ such that
  \[
    B' = A B A^T =
    \begin{pmatrix}
      1 & 0 & 0 \\
      0 & 1 & 0 \\
      0 & 0 & 1
    \end{pmatrix},\
    \begin{pmatrix}
      1 & 0 & 0 \\
      0 & 1 & 0 \\
      0 & 0 & 0
    \end{pmatrix},
    \text{ or }
    \begin{pmatrix}
      1 & 0 & 0 \\
      0 & 0 & 0 \\
      0 & 0 & 0
    \end{pmatrix}.
  \]
  Then $f' = (x, y, z) B' (x, y, z)^T$
  has $f' = x^2 + y^2 + z^2$,
  $x^2 + y^2$, or $x^2$. Now
  $A$ induces an isomorphism
  $h_{A^{-1}} : \PP^2 \to \PP^2$
  and $g := h_{A^{-1}}|_{X} : X \to h_{A^{-1}}(X) = V(f')$,
  so any projective conic is isomorphic
  to
  \[
    V(x^2 + y^2 + z^2), \quad
    V(x^2 + y^2), \quad
    \text{or} \quad
    V(x^2).
  \]
\end{example}

\begin{example}[Projections]
  Let $a = [1 : 0 : \cdots : 0]$
  and define
  \begin{align*}
    \PP^n \setminus \{a\}
    &\overset{f}{\longrightarrow}
    \PP^{n - 1} \\
    [x_0 : \cdots : x_n]
    &\longmapsto
    [x_1 : \cdots : x_n].
  \end{align*}
  Geometrically, if we fix
  $[b] \in \PP^n \setminus \{a\}$
  and set
  \[
    \ell_{a, b}
    = \{
      [s : tb_1 : \cdots : tb_n]
      : (s, t) \in k^2 \setminus \{0\}
    \}
    = \text{the line through $a$ and $b$},
  \]
  then $\ell_{a, b} \cap V(x_0) = [0 : b_1 : \cdots : b_n] = [0 : f(b)]$.
\end{example}

  \chapter{Oct.~14 --- Projective Space as Varieties, Part 2}

\section{Example of Projective Morphism}
\begin{example}[Projections, continued]
  Let $H \subseteq \PP^n$ be a hyperplane
  and $p \notin H$. Then we can define
  \begin{align*}
    \PP^n \setminus \{p\}
    &\overset{\pi}{\longrightarrow} H \cong \PP^{n - 1} \\
    q &\longmapsto
    \text{intersection point of $H$ and $\overline{pq}$}.
  \end{align*}
  For example, when $n = 2$,
  $p = [1 : 0 : 0] \in \PP^2_{x_0 : x_1 : x_2}$,
  and $H = V(x_0)$, then we have
  \begin{align*}
    \PP^2 \setminus \{p\}
    &\longrightarrow \PP^{1} \\
    [x_0 : x_1 : x_2]
    &\longmapsto [x_1 : x_2].
  \end{align*}
  Note that this does not extend to a
  morphism $\PP^2 \to \PP^1$. But if
  we let $X = V(x_0 x_1 - x_2^2) \subseteq \PP^2$, then
  the restriction of the above morphism
  to $X$:
  \begin{align*}
    X \setminus \{p\}
    &\longrightarrow \PP^{1} \\
    [x_0 : x_1 : x_2]
    &\longmapsto [x_1 : x_2].
  \end{align*}
  does extend to a morphism
  \begin{align*}
    X
    &\longrightarrow \PP^{1} \\
    [x_0 : x_1 : x_2]
    &\longmapsto
    \begin{cases}
      [x_1 : x_2] & \text{if } [x] \ne [1 : 0 : 0], \\
      [x_2 : x_0] & \text{if } [x] \ne [1 : 1 : 0].
    \end{cases}
  \end{align*}
\end{example}

\section{The Segre Embedding}

\begin{remark}
  We now want to show that
  projective varieties are varieties, and
  understand and analogue of compactness
  in algebraic geometry.
  To do this, we will need to understand
  products.
\end{remark}

\begin{definition}
  Fix $m, n \ge 0$. The
  \emph{Segre embedding} is
  the map $\Sigma : \PP_{x_i}^m \times \PP_{y_i}^n \to \PP^N_{z_{i, j}}$
  given by
  \[
    \Sigma([x_0 : \cdots : x_m], [y_0 : \cdots : y_n])
    = [x_i y_j : 0 \le i \le m, 0 \le j \le n],
  \]
  where $N = (m + 1)(n + 1) - 1$.
\end{definition}

\begin{prop}
  Let $\Sigma$ be the Segre embedding.
  Then
  \begin{enumerate}
    \item $X = \Sigma(\PP^m \times \PP^n) = V(z_{i, j} z_{k, \ell} - z_{i, \ell} z_{k, j} : 0 \le i, k \le m,\, 0 \le j, \ell \le n)$.
    \item The map
      $\PP^m \times \PP^n \to X$ is an
      isomorphism, i.e. $\Sigma$ is a
      closed embedding.
  \end{enumerate}
\end{prop}

\begin{proof}
  (0) First one can check that
  $\Sigma$ is a morphism.
  To do this, restrict to charts.

  (1) Fix $[a_{i ,j}] \in \PP^N$.
  Then $[a_{i, j}] \in \im \Sigma$
  if and only if the matrix $(a_{i, j})$
  has rank $1$, which occurs if and only
  if all $2 \times 2$ minors of
  $(a_{i, j})$ vanish,
  which happens if and only if
  \[
    a_{i, j} a_{k, \ell} - a_{i, \ell} a_{k, j} = 0
  \]
  for all $i, j, k, \ell$ for which the
  above equation makes sense.

  (2) We define a morphism
  $X \to \PP^m \times \PP^n$ that will
  be inverse to $\Sigma$. Set
  $U_{i, j} = X \cap \{z_{i, j} \ne 0\}$.
  Define
  \begin{align*}
    U_{i, j}
    &\overset{h_{i, j}}{\longrightarrow}
    \PP^m \times \PP^n \\
    [z_{i, j}]
    &\longmapsto
    ([z_{0, j} : \cdots : z_{m, j}], [z_{i, 0} : \cdots : z_{i, n}]).
  \end{align*}
  Using the definition of $X$
  (as the set of rank $1$ matrices up
  to scaling), these glue to give a
  morphism $X \to \PP^m \times \PP^n$
  that is inverse to $\Sigma$.
\end{proof}

\begin{example}
  Let $m = n = 1$. Then the Segre
  embedding is given by
  \begin{align*}
    \PP^1 \times \PP^1
    &\longrightarrow \PP^3_{x : y : z : w} \\
    ([a_0 : a_1], [b_0 : b_1])
    &\longmapsto
    \begin{bmatrix}
      a_0 b_0 & a_0 b_1 \\
      a_1 b_0 & a_1 b_1
    \end{bmatrix}.
  \end{align*}
  Then $\im \Sigma = V(xw - yz)$. Observe
  that the images of $\{a\} \times \PP^1$
  and $\PP^1 \times \{b\}$ in
  $\Sigma(\PP^1 \times \PP^1)$ are two
  families of lines, where the lines within
  each families do not intersect.
\end{example}

\begin{remark}
  The following are consequences of the
  Segre embedding.
  \begin{enumerate}
    \item We can study products of
      projective varieties.

      \begin{definition}[Redefinition of projective variety]
        A \emph{projective variety}
        is a (pre-)variety $X$ such that
        there exists a closed embedding
        $X \hookrightarrow \PP^n$ for some $n \ge 0$.
      \end{definition}

      Now using the Segre embedding,
      we get that
      $\PP^m \times \PP^n$ is a projective
      variety. Moreover, if
      $X \subseteq \PP^m$ and $Y \subseteq \PP^n$
      are projective varieties, then
      so is $X \times Y$.
    \item We can show that $\PP^n$ is
      separated.

      \begin{lemma}\label{lem:pn-separated}
        $\Delta_{\PP^n}$ is closed in
        $\PP^n \times \PP^n$.
      \end{lemma}

      \begin{proof}
        Observe that
        \[
          \Delta_{\PP^n}
          = \{([x_0 : \cdots : x_n], [y_0 : \cdots : y_n]) \in \PP^n \times \PP^n :
            x_i y_j - x_j y_i
            = 0 \text{ for all }
            0 \le i, j \le n
          \}.
        \]
        It suffices to show that
        $\Delta_{\PP^n}$ is closed in
        $\PP^n \times \PP^n$.
        There are two ways to see this.
        The first is to use the Segre
        embedding $\PP^n_{x_i} \times \PP^n_{y_i} \underset{\text{cl}}{\overset{\Sigma}{\hookrightarrow}} \PP^{N}_{z_{i, j}}$
        Then we can write
        \[
          \Sigma(\Delta_{\PP^n})
          = \Sigma(\PP^n \times \PP^n)
          \cap V(z_{i, j} - z_{j, i} : 0 \le i, j \le n),
        \]
        which is closed in $\PP^N$, so
        $\Delta_{\PP^n}$ is closed in
        $\PP^n \times \PP^n$.
        Alternatively, one can just compute
        $\Delta_{\PP^n}$ directly on the
        affine charts. One can fill in the
        details of this method as an
        exercise.
      \end{proof}
  \end{enumerate}
\end{remark}

\begin{prop}
  Projective varieties are varieties.
\end{prop}

\begin{proof}
  We have already seen that they are
  pre-varieties, so
  it suffices to show that they are separated.
  By Lemma \ref{lem:pn-separated},
  $\PP^n$ is separated. Thus
  any closed sub-prevariety of $\PP^n$ is also
  separated.
\end{proof}

\section{Completeness}

\begin{remark}
  We now want an analogue of compactness
  in algebraic geometry. One issue
  is that all varieties are
  compact to begin with, but
  $\Affine^n$ has points missing in some
  sense.
\end{remark}

\begin{example}\label{ex:missing-point}
  Consider the projection map
  \begin{align*}
    \Affine^1 \times \Affine^1
    &\overset{\mathrm{pr}_2}{\longrightarrow} \Affine^1 \\
    (x, t) &\longmapsto t.
  \end{align*}
  Then $X = V(xt - 1) \subseteq \Affine^1 \times \Affine^1$
  is closed, but
  $\mathrm{pr}_2(X) = \Affine^1 \setminus \{0\}$.
  If we instead viewed this over $\PP^1$:
  \begin{align*}
    \PP^1_{[x : y]} \times \Affine^1_t
    &\overset{\mathrm{pr}_2}{\longrightarrow} \Affine^1_t \\
    ([x : y], t) &\longmapsto t
  \end{align*}
  with $\overline{X} = V(xt - y)$,
  then $\mathrm{pr}_2(\overline{X}) = \Affine^1$
  as there is a point
  $([1 : 0], 0)$ at infinity in
  $\overline{X}$.
  In other words, ``compactifying''
  $\Affine^1$ to $\PP^1$ gives the
  desired missing point.
\end{example}

\begin{definition}
  A morphism $f : X \to Y$ is
  \emph{closed} if $f(Z)$ is
  closed in $Y$ for all closed sets $Z \subseteq X$.
\end{definition}

\begin{definition}
  A variety $X$ is \emph{complete}
  if the projection
  \[
    \mathrm{pr}_2
    : X \times Y \longrightarrow Y
  \]
  is closed for all varieties $Y$.
\end{definition}

\begin{remark}
  The same definition for
  topological spaces
  gives the usual notion of
  compactness.
\end{remark}

\begin{example}
  Example \ref{ex:missing-point} shows that
  $\Affine^1$ is not complete.
  Similar examples show that
  $\Affine^n$ is not complete for any
  $n \ge 1$.
\end{example}

\begin{prop}
  $\PP^n$ is complete.
\end{prop}

\begin{proof}
  The steps to show this are the following:
  \begin{enumerate}
    \item For any $m, n \ge 0$,
      the projection
      $\mathrm{pr}_2 : \PP^n \times \PP^m \to \PP^m$
      is closed.

      See Gathmann for a proof of this
      fact.
    \item If $Y$ is an affine variety,
      then
      $\mathrm{pr}_2 : \PP^n \times Y \to Y$
      is closed.

      To see this, write
      $Y = V(I) \subseteq \Affine^m \subseteq \PP^m$
      and consider the diagram
      \begin{center}
      \begin{tikzcd}
        \PP^n \times \PP^m
        \ar[r, "\mathrm{pr}_2"] &\PP^m \\
        \PP^n \times \Affine^n \ar[u, hook, "\text{op}"] \ar[r, "\mathrm{pr}_2"] &\Affine^n \ar[u, hook]\\
        \PP^n \times Y \ar[u, hook, "\text{cl}"] \ar[r, "\mathrm{pr}_2"] &\ar[u, hook] Y
      \end{tikzcd}
      \end{center}
      Since the top row is closed, so
      is the bottom row.
  \end{enumerate}
  Finally, we complete the proof.
  If $Y$ is a variety, then it admits an
  open affine cover
  $Y = \bigcup_{i = 1}^r U_i$. Now
  $\mathrm{pr}_2 : \PP^n \times Y \to Y$
  is closed when restricted to
  $\mathrm{pr}_2^{-1}(U_i)$. Since
  closedness of a map can be checked on
  an open cover of the target,
  we see that $\PP^n \times Y \to Y$
  is closed.
\end{proof}

\begin{remark}
  The same definition and
  arguments for completeness work if
  $Y$ is replaced by a pre-variety.
\end{remark}

\begin{exercise}
  Show that if $X$ is a complete variety,
  then so is any closed subvariety of
  $X$.
\end{exercise}

\begin{corollary}
  Any projective variety is complete.
\end{corollary}

\end{document}
