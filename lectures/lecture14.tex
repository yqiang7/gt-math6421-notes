\chapter{Oct.~9 --- Projective Space as Varieties}

\section{More on the Zariski Topology on \texorpdfstring{$\PP^n$}{Pn}}
\begin{prop}
  For each $0 \le i \le n$,
  the map
  \begin{align*}
    U_i = \PP^n \setminus V(x_i)
    &\overset{h_i}{\longrightarrow} \Affine^n \\
    [x_0 : \cdots : x_n]
    &\longmapsto
    (x_0 / x_i, \dots, \widehat{x_i / x_i}, \dots, x_n / x_i)
  \end{align*}
  is a homeomorphism.
\end{prop}

\begin{proof}
  The main inputs to the proof are
  \begin{itemize}
    \item For $I \le k[x_0, \dots, x_n]$
      homogeneous,
      $h_0(V(I) \cap U_0) = V(I^i)$.
    \item For $J \le k[x_1, \dots, x_n]$,
      $h_0^{-1}(V(J)) = V(J^h)$.
  \end{itemize}
  Fill in the remaining details as
  an exercise.
\end{proof}

\begin{prop}[Projective closure]\label{prop:projective-closure}
  For $J \le k[x_1, \dots, x_n]$
  and $X = V_a(J) \subseteq \Affine^n \subseteq \PP^n$,
  we have
  \[
    \overline{X} = V_p(J^h).
  \]
\end{prop}

\begin{proof}
  See Gathmann.
\end{proof}

\begin{prop}
  If $X = V_a(f) \subseteq \Affine^n$ with $f \in k[x_1, \dots, x_n]$,
  then its projective closure
  in $\PP^n$ is
  \[
    \overline{X} = V_p(f^h).
  \]
\end{prop}

\begin{proof}
  We know that
  $\overline{X} = V_p(\langle f \rangle^h)$
  by Proposition
  \ref{prop:projective-closure}.
  Now
  \[
    \langle f \rangle^h
    = \langle (fg)^h : g \in k[x_1, \dots, x_n] \rangle
    = \langle f^h g^h : g \in k[x_1, \dots, x_n] \rangle
    = \langle f^h \rangle,
  \]
  which implies the desired result.
\end{proof}

\begin{example}[Twisted cubic]
  Take $X = \im(\Affine^1 \to \Affine^3 : t \mapsto (t, t^2, t^3))$.
  Note that $X \cong \Affine^1$, and
  \[
    I_a(X)
    = (x^2 - y, x^3 - z)
    = (x^2 - y, x^3 - z, xy - z).
  \]
  Then one can check
  that
  $\overline{X} \subseteq \PP^3_{w : x : y : z}$
  is given by
  $\overline{X} = V_p(x^2 - yw, x^3 - zw^2, xy - zw)$.
  However, one can also check that
  $\overline{X}$ cannot be
  cut out by $2$ equations.
  For example,
  \[
    V_p(x^2 - yw, x^3 - zw^2)
    = \overline{X} \cup V(w, x).
  \]
\end{example}

\section{Projective Space as Varieties}

\begin{remark}
  Our goal now is to show that
  projective varieties are varieties.
  The first step is to define a sheaf
  of regular functions on $\PP^n$.
\end{remark}

\begin{definition}
  Let $U$ be an open set of a
  projective variety $X \subseteq \PP^n$.
  A function $\varphi : U \to k$
  is \emph{regular} if for every $p \in U$,
  there exists $d \in \N$,
  $f, g \in k[x_0, \dots, x_n]$
  homogeneous of degree $d$, and
  $U_p \subseteq U$ open such that
  \[
    \varphi(x) = \frac{f(x)}{g(x)}
    \quad \text{for all } x \in U_p.
  \]
\end{definition}

\begin{remark}
  If $X \subseteq \PP^n$ is a projective
  variety, then
  \[
    \OO_X(U)
    = \{\varphi : U \to k \mid \varphi \text{ is regular}\}
  \]
  is a sheaf of rings on $X$.
  Again this is because the regular
  condition can be checked locally.
\end{remark}

\begin{prop}
  If $X \subseteq \PP^n$ is a projective
  variety, then
  $(X, \OO_X)$ is a pre-variety.
\end{prop}

\begin{proof}
  Let $X_i = X \cap (\PP^n \setminus V(x_i))$.
  It suffices to show
  $(X_i, \OO_X|_{X_i})$ is an affine
  variety for each $0 \le i \le n$.
  For simplicity, assume $i = 0$.
  Let $J = I(X) \le k[x_0, \dots, x_n]$
  and $Z_0 = V(J^i) \subseteq \Affine^n$.
  We have seen before that we have
  a homeomorphism
  \begin{align*}
    X_0 &\overset{F}{\longrightarrow} Z_0 \\
    [x_0 : \cdots : x_n]
        &\longmapsto (x_1 / x_0, \dots, x_n / x_0).
  \end{align*}
  We claim that $F$ induces an
  isomorphism of ringed spaces
  $(X_0, \OO_X|_{X_0}) \cong (Z_0, \OO_{Z_0})$.
  To see this, we need to check
  that regular functions pull back to
  regular functions via $F$ and
  $F^{-1}$. A regular function on an
  open set of $X_0$ is locally of
  the form
  \[
    \frac{f(x_0, \dots, x_n)}{g(x_0, \dots, x_n)}
  \]
  with $f, g$ homogeneous of the
  same degree. Now
  \[
    (F^{-1})^*
    \left(\frac{f(x_0, \dots, x_n)}{g(x_0, \dots, x_n)}\right)
    = \frac{f(1, x_1, \dots, x_n)}{g(1, x_1, \dots, x_n)},
  \]
  which is a fraction of polynomials
  and hence regular on $Z_0$.
  So $F^{-1}$ pulls regular functions
  back to regular functions. Conversely,
  a regular function on
  $Z_0$ is locally given by
  \[
    \frac{q(x_1, \dots, x_n)}{r(x_1, \dots, x_n)},
  \]
  and its pullback via $F$ is
  \[
    F^*\left(\frac{q(x_1, \dots, x_n)}{r(x_1, \dots, x_n)}\right)
    = \frac{q(x_1 / x_0, \dots, x_n / x_0)}{r(x_1 / x_0, \dots, x_n / x_0)}
    = \frac{x_0^d q(x_1 / x_0, \dots, x_n / x_0)}{x_0^d r(x_1 / x_0, \dots, x_n / x_0)},
  \]
  where $d = \max\{\deg q, \deg r\}$.
  This is regular on $X_0$, so
  $F$ also pulls regular functions
  back to regular functions.
  So we get an isomorphism of
  ringed spaces, as desired.
\end{proof}

\begin{example}
  $\PP^n$ is a pre-variety, and
  $\PP^n \setminus V(x_i) =: U_i \cong \Affine^n$
  as pre-varieties.
\end{example}

\begin{definition}
  A \emph{morphism}
  of projective varieties
  is a morphism of
  the underlying pre-varieties.
\end{definition}

\begin{remark}
  For a projective variety $X$, it will
  be convenient to work with
  ``global coordinates,'' i.e.
  \[
    S(X)
    := k[x_0, \dots, x_n] / I_p(X).
  \]
  This is called the
  \emph{homogeneous coordinate ring}.
  Note the following:
  \begin{enumerate}
    \item For $f \in S(X)$
      homogeneous, $f$ is not
      necessarily a well-defined
      function on $X$. But
      \[
        V(f) = \{[x] \in X : f(x) = 0\}
      \]
      is still well-defined.
    \item A relative version of the
      projective Nullstellensatz holds:
      There is a bijection
      \begin{align*}
        \left\{
          \substack{\displaystyle\text{projective subvarieties} \\ \displaystyle Y \subseteq X}
        \right\}
        &\longleftrightarrow
        \left\{
          \substack{\displaystyle\text{radical homogeneous ideals in $S(X)$} \\ \displaystyle \text{not equal to $(\overline{x}_1, \dots, \overline{x}_n)$}}
        \right\} \\
        Y &\longmapsto I(Y) \\
        V(J) &\mathrel{\reflectbox{\ensuremath{\longmapsto}}} J
      \end{align*}
  \end{enumerate}
  where $I(Y) = \langle f \in S(X) : f \text{ homogeneous and } f(y) = 0 \text{ for all } y \in Y \rangle$.
\end{remark}

\begin{lemma}
  If $X \subseteq \PP^n$ and
  $f_0, \dots, f_m \in S(X)$ are homogeneous
  of the same degree, then
  \begin{align*}
    U = X \setminus V(f_0, \dots, f_m)
    &\overset{f}{\longrightarrow} \PP^m \\
    [x_0 : \cdots : x_n]
    &\longmapsto [f_0(x) : \cdots : f_m(x)]
  \end{align*}
  is a morphism.
\end{lemma}

\begin{proof}
  To see that $f$ is well-defined,
  note that for $[a_0 : \cdots : a_n] \in X \setminus V(f_0, \dots, f_m)$,
  we have
  \[
    (f_0(\lambda a), \dots, f_m(\lambda a))
    = \lambda^d (f_0(a), \dots, f_m(a))
  \]
  with $d = \deg f_i$.
  So $[f_0(a) : \cdots : f_m(a)] \in \PP^m$
  is well-defined. To see that $f$ is a
  morphism, we check locally on
  $\PP^m$. Let
  $V_i = \PP^m \setminus V(x_i)$
  and $U_i = f^{-1}(V_i)$.
  Then
  \begin{align*}
    U_i
    &\overset{f|_{U_i}}{\longrightarrow} V_i \cong \Affine^m \\
    a &\longmapsto
    \left(
      \frac{f_0(a)}{f_i(a)},
      \dots,
      \widehat{\frac{f_i(a)}{f_i(a)}},
      \dots,
      \frac{f_m(a)}{f_i(a)}
    \right).
  \end{align*}
  Since each $f_j / f_i$ is regular,
  $f|_{U_i}$ is a morphism. So
  $f$ is a morphism.
\end{proof}

\begin{example}
  Define a map
  \begin{align*}
    \PP^1_{s : t}
    &\overset{f}{\longrightarrow}
    \PP^3_{x : y : z} \\
    [s : t]
    &\longmapsto
    [s^3 : s^2 t : s t^2 : t^3].
  \end{align*}
  Then $S(\PP^1) = k[s, t]$ and
  $f(\PP^1)$ is the projective
  twisted cubic in $\PP^3$.
\end{example}

\begin{example}
  Let $A \in \GL_{n + 1}(k)$.
  Then
  \begin{align*}
    f_A : \PP^n
    &\longrightarrow \PP^n \\
    [x] &\longmapsto [Ax]
  \end{align*}
  is an isomorphism with inverse
  $f_{A^{-1}}$. We will see later
  that we have a surjective group
  homomorphism
  \begin{align*}
    \GL_{n + 1}(k)
    &\longrightarrow \Aut(\PP^n) \\
    A &\longmapsto f_A
  \end{align*}
  with kernel $k^\times I$.
  So we get
  $\Aut(\PP^n) \cong \GL_{n + 1}(k) / k^\times I =: \PGL_{n + 1}(k)$.
\end{example}

\begin{example}[Conics]
  Let $f \in k[x, y, z]$ be homogeneous
  of degree $2$, and write
  \[
    f =
    (x, y, z) B (x, y, z)^T
  \]
  with $B$ a symmetric $3 \times 3$
  matrix. We want to characterize
  $X = V(f)$.
  Choose
  $A \in \GL_3(k)$ such that
  \[
    B' = A B A^T =
    \begin{pmatrix}
      1 & 0 & 0 \\
      0 & 1 & 0 \\
      0 & 0 & 1
    \end{pmatrix},\
    \begin{pmatrix}
      1 & 0 & 0 \\
      0 & 1 & 0 \\
      0 & 0 & 0
    \end{pmatrix},
    \text{ or }
    \begin{pmatrix}
      1 & 0 & 0 \\
      0 & 0 & 0 \\
      0 & 0 & 0
    \end{pmatrix}.
  \]
  Then $f' = (x, y, z) B' (x, y, z)^T$
  has $f' = x^2 + y^2 + z^2$,
  $x^2 + y^2$, or $x^2$. Now
  $A$ induces an isomorphism
  $h_{A^{-1}} : \PP^2 \to \PP^2$
  and $g := h_{A^{-1}}|_{X} : X \to h_{A^{-1}}(X) = V(f')$,
  so any projective conic is isomorphic
  to
  \[
    V(x^2 + y^2 + z^2), \quad
    V(x^2 + y^2), \quad
    \text{or} \quad
    V(x^2).
  \]
\end{example}

\begin{example}[Projections]
  Let $a = [1 : 0 : \cdots : 0]$
  and define
  \begin{align*}
    \PP^n \setminus \{a\}
    &\overset{f}{\longrightarrow}
    \PP^{n - 1} \\
    [x_0 : \cdots : x_n]
    &\longmapsto
    [x_1 : \cdots : x_n].
  \end{align*}
  Geometrically, if we fix
  $[b] \in \PP^n \setminus \{a\}$
  and set
  \[
    \ell_{a, b}
    = \{
      [s : tb_1 : \cdots : tb_n]
      : (s, t) \in k^2 \setminus \{0\}
    \}
    = \text{the line through $a$ and $b$},
  \]
  then $\ell_{a, b} \cap V(x_0) = [0 : b_1 : \cdots : b_n] = [0 : f(b)]$.
\end{example}
