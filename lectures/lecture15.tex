\chapter{Oct.~14 --- Projective Space as Varieties, Part 2}

\section{Example of Projective Morphism}
\begin{example}[Projections, continued]
  Let $H \subseteq \PP^n$ be a hyperplane
  and $p \notin H$. Then we can define
  \begin{align*}
    \PP^n \setminus \{p\}
    &\overset{\pi}{\longrightarrow} H \cong \PP^{n - 1} \\
    q &\longmapsto
    \text{intersection point of $H$ and $\overline{pq}$}.
  \end{align*}
  For example, when $n = 2$,
  $p = [1 : 0 : 0] \in \PP^2_{x_0 : x_1 : x_2}$,
  and $H = V(x_0)$, then we have
  \begin{align*}
    \PP^2 \setminus \{p\}
    &\longrightarrow \PP^{1} \\
    [x_0 : x_1 : x_2]
    &\longmapsto [x_1 : x_2].
  \end{align*}
  Note that this does not extend to a
  morphism $\PP^2 \to \PP^1$. But if
  we let $X = V(x_0 x_1 - x_2^2) \subseteq \PP^2$, then
  the restriction of the above morphism
  to $X$:
  \begin{align*}
    X \setminus \{p\}
    &\longrightarrow \PP^{1} \\
    [x_0 : x_1 : x_2]
    &\longmapsto [x_1 : x_2].
  \end{align*}
  does extend to a morphism
  \begin{align*}
    X
    &\longrightarrow \PP^{1} \\
    [x_0 : x_1 : x_2]
    &\longmapsto
    \begin{cases}
      [x_1 : x_2] & \text{if } [x] \ne [1 : 0 : 0], \\
      [x_2 : x_0] & \text{if } [x] \ne [1 : 1 : 0].
    \end{cases}
  \end{align*}
\end{example}

\section{The Segre Embedding}

\begin{remark}
  We now want to show that
  projective varieties are varieties, and
  understand and analogue of compactness
  in algebraic geometry.
  To do this, we will need to understand
  products.
\end{remark}

\begin{definition}
  Fix $m, n \ge 0$. The
  \emph{Segre embedding} is
  the map $\Sigma : \PP_{x_i}^m \times \PP_{y_i}^n \to \PP^N_{z_{i, j}}$
  given by
  \[
    \Sigma([x_0 : \cdots : x_m], [y_0 : \cdots : y_n])
    = [x_i y_j : 0 \le i \le m, 0 \le j \le n],
  \]
  where $N = (m + 1)(n + 1) - 1$.
\end{definition}

\begin{prop}
  Let $\Sigma$ be the Segre embedding.
  Then
  \begin{enumerate}
    \item $X = \Sigma(\PP^m \times \PP^n) = V(z_{i, j} z_{k, \ell} - z_{i, \ell} z_{k, j} : 0 \le i, k \le m,\, 0 \le j, \ell \le n)$.
    \item The map
      $\PP^m \times \PP^n \to X$ is an
      isomorphism, i.e. $\Sigma$ is a
      closed embedding.
  \end{enumerate}
\end{prop}

\begin{proof}
  (0) First one can check that
  $\Sigma$ is a morphism.
  To do this, restrict to charts.

  (1) Fix $[a_{i ,j}] \in \PP^N$.
  Then $[a_{i, j}] \in \im \Sigma$
  if and only if the matrix $(a_{i, j})$
  has rank $1$, which occurs if and only
  if all $2 \times 2$ minors of
  $(a_{i, j})$ vanish,
  which happens if and only if
  \[
    a_{i, j} a_{k, \ell} - a_{i, \ell} a_{k, j} = 0
  \]
  for all $i, j, k, \ell$ for which the
  above equation makes sense.

  (2) We define a morphism
  $X \to \PP^m \times \PP^n$ that will
  be inverse to $\Sigma$. Set
  $U_{i, j} = X \cap \{z_{i, j} \ne 0\}$.
  Define
  \begin{align*}
    U_{i, j}
    &\overset{h_{i, j}}{\longrightarrow}
    \PP^m \times \PP^n \\
    [z_{i, j}]
    &\longmapsto
    ([z_{0, j} : \cdots : z_{m, j}], [z_{i, 0} : \cdots : z_{i, n}]).
  \end{align*}
  Using the definition of $X$
  (as the set of rank $1$ matrices up
  to scaling), these glue to give a
  morphism $X \to \PP^m \times \PP^n$
  that is inverse to $\Sigma$.
\end{proof}

\begin{example}
  Let $m = n = 1$. Then the Segre
  embedding is given by
  \begin{align*}
    \PP^1 \times \PP^1
    &\longrightarrow \PP^3_{x : y : z : w} \\
    ([a_0 : a_1], [b_0 : b_1])
    &\longmapsto
    \begin{bmatrix}
      a_0 b_0 & a_0 b_1 \\
      a_1 b_0 & a_1 b_1
    \end{bmatrix}.
  \end{align*}
  Then $\im \Sigma = V(xw - yz)$. Observe
  that the images of $\{a\} \times \PP^1$
  and $\PP^1 \times \{b\}$ in
  $\Sigma(\PP^1 \times \PP^1)$ are two
  families of lines, where the lines within
  each families do not intersect.
\end{example}

\begin{remark}
  The following are consequences of the
  Segre embedding.
  \begin{enumerate}
    \item We can study products of
      projective varieties.

      \begin{definition}[Redefinition of projective variety]
        A \emph{projective variety}
        is a (pre-)variety $X$ such that
        there exists a closed embedding
        $X \hookrightarrow \PP^n$ for some $n \ge 0$.
      \end{definition}

      Now using the Segre embedding,
      we get that
      $\PP^m \times \PP^n$ is a projective
      variety. Moreover, if
      $X \subseteq \PP^m$ and $Y \subseteq \PP^n$
      are projective varieties, then
      so is $X \times Y$.
    \item We can show that $\PP^n$ is
      separated.

      \begin{lemma}\label{lem:pn-separated}
        $\Delta_{\PP^n}$ is closed in
        $\PP^n \times \PP^n$.
      \end{lemma}

      \begin{proof}
        Observe that
        \[
          \Delta_{\PP^n}
          = \{([x_0 : \cdots : x_n], [y_0 : \cdots : y_n]) \in \PP^n \times \PP^n :
            x_i y_j - x_j y_i
            = 0 \text{ for all }
            0 \le i, j \le n
          \}.
        \]
        It suffices to show that
        $\Delta_{\PP^n}$ is closed in
        $\PP^n \times \PP^n$.
        There are two ways to see this.
        The first is to use the Segre
        embedding $\PP^n_{x_i} \times \PP^n_{y_i} \underset{\text{cl}}{\overset{\Sigma}{\hookrightarrow}} \PP^{N}_{z_{i, j}}$
        Then we can write
        \[
          \Sigma(\Delta_{\PP^n})
          = \Sigma(\PP^n \times \PP^n)
          \cap V(z_{i, j} - z_{j, i} : 0 \le i, j \le n),
        \]
        which is closed in $\PP^N$, so
        $\Delta_{\PP^n}$ is closed in
        $\PP^n \times \PP^n$.
        Alternatively, one can just compute
        $\Delta_{\PP^n}$ directly on the
        affine charts. One can fill in the
        details of this method as an
        exercise.
      \end{proof}
  \end{enumerate}
\end{remark}

\begin{prop}
  Projective varieties are varieties.
\end{prop}

\begin{proof}
  We have already seen that they are
  pre-varieties, so
  it suffices to show that they are separated.
  By Lemma \ref{lem:pn-separated},
  $\PP^n$ is separated. Thus
  any closed sub-prevariety of $\PP^n$ is also
  separated.
\end{proof}

\section{Completeness}

\begin{remark}
  We now want an analogue of compactness
  in algebraic geometry. One issue
  is that all varieties are
  compact to begin with, but
  $\Affine^n$ has points missing in some
  sense.
\end{remark}

\begin{example}\label{ex:missing-point}
  Consider the projection map
  \begin{align*}
    \Affine^1 \times \Affine^1
    &\overset{\mathrm{pr}_2}{\longrightarrow} \Affine^1 \\
    (x, t) &\longmapsto t.
  \end{align*}
  Then $X = V(xt - 1) \subseteq \Affine^1 \times \Affine^1$
  is closed, but
  $\mathrm{pr}_2(X) = \Affine^1 \setminus \{0\}$.
  If we instead viewed this over $\PP^1$:
  \begin{align*}
    \PP^1_{[x : y]} \times \Affine^1_t
    &\overset{\mathrm{pr}_2}{\longrightarrow} \Affine^1_t \\
    ([x : y], t) &\longmapsto t
  \end{align*}
  with $\overline{X} = V(xt - y)$,
  then $\mathrm{pr}_2(\overline{X}) = \Affine^1$
  as there is a point
  $([1 : 0], 0)$ at infinity in
  $\overline{X}$.
  In other words, ``compactifying''
  $\Affine^1$ to $\PP^1$ gives the
  desired missing point.
\end{example}

\begin{definition}
  A morphism $f : X \to Y$ is
  \emph{closed} if $f(Z)$ is
  closed in $Y$ for all closed sets $Z \subseteq X$.
\end{definition}

\begin{definition}
  A variety $X$ is \emph{complete}
  if the projection
  \[
    \mathrm{pr}_2
    : X \times Y \longrightarrow Y
  \]
  is closed for all varieties $Y$.
\end{definition}

\begin{remark}
  The same definition for
  topological spaces
  gives the usual notion of
  compactness.
\end{remark}

\begin{example}
  Example \ref{ex:missing-point} shows that
  $\Affine^1$ is not complete.
  Similar examples show that
  $\Affine^n$ is not complete for any
  $n \ge 1$.
\end{example}

\begin{prop}
  $\PP^n$ is complete.
\end{prop}

\begin{proof}
  The steps to show this are the following:
  \begin{enumerate}
    \item For any $m, n \ge 0$,
      the projection
      $\mathrm{pr}_2 : \PP^n \times \PP^m \to \PP^m$
      is closed.

      See Gathmann for a proof of this
      fact.
    \item If $Y$ is an affine variety,
      then
      $\mathrm{pr}_2 : \PP^n \times Y \to Y$
      is closed.

      To see this, write
      $Y = V(I) \subseteq \Affine^m \subseteq \PP^m$
      and consider the diagram
      \begin{center}
      \begin{tikzcd}
        \PP^n \times \PP^m
        \ar[r, "\mathrm{pr}_2"] &\PP^m \\
        \PP^n \times \Affine^n \ar[u, hook, "\text{op}"] \ar[r, "\mathrm{pr}_2"] &\Affine^n \ar[u, hook]\\
        \PP^n \times Y \ar[u, hook, "\text{cl}"] \ar[r, "\mathrm{pr}_2"] &\ar[u, hook] Y
      \end{tikzcd}
      \end{center}
      Since the top row is closed, so
      is the bottom row.
  \end{enumerate}
  Finally, we complete the proof.
  If $Y$ is a variety, then it admits an
  open affine cover
  $Y = \bigcup_{i = 1}^r U_i$. Now
  $\mathrm{pr}_2 : \PP^n \times Y \to Y$
  is closed when restricted to
  $\mathrm{pr}_2^{-1}(U_i)$. Since
  closedness of a map can be checked on
  an open cover of the target,
  we see that $\PP^n \times Y \to Y$
  is closed.
\end{proof}

\begin{remark}
  The same definition and
  arguments for completeness work if
  $Y$ is replaced by a pre-variety.
\end{remark}

\begin{exercise}
  Show that if $X$ is a complete variety,
  then so is any closed subvariety of
  $X$.
\end{exercise}

\begin{corollary}
  Any projective variety is complete.
\end{corollary}
