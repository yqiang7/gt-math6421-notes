\chapter{Sept.~11 --- Morphisms}

\section{Morphisms of Open Sets}

\begin{remark}
  Recall that a continuous
  map $f : \R^m \to \R^n$ is \emph{smooth} if
  it satisfies either of the following equivalent conditions:
  \begin{enumerate}
    \item there exist smooth functions
      $f_1, \dots, f_n : \R^m \to \R$ such
      that $f(x) = (f_1(x), \dots, f_n(x))$;
    \item for each
      open set $U \subseteq \R^n$ and
      smooth $\varphi : U \to \R$,
      the function $f^* \varphi := \varphi \circ f : \R^m \to \R$
      is smooth.
  \end{enumerate}
  The implication $(1 \Rightarrow 2)$ follows
  by the chain rule. To see $(2 \Rightarrow 1)$,
  take $y_i : \R^n \to \R$
  defined by $f_i := f^* y_i$.
  We want a similar definition
  in algebraic geometry.
\end{remark}

\begin{definition}
  Let $X$ and $Y$ be open sets of affine
  varieties. A \emph{morphism}
  $f : X \to Y$ is a continuous map
  such that every $U \subseteq Y$ open
  and $\varphi \in \OO_Y(U)$, the map
  \begin{center}
    \begin{tikzcd}
      f^{-1}(U) \arrow[r, "f"] \arrow[rr, bend left=40, "f^* \varphi"]& U \arrow[r, "\varphi"] & k
    \end{tikzcd}
  \end{center}
  satisfies $f^* \varphi \in \OO_X(f^{-1}(U))$.
  A morphism is an \emph{isomorphism}
  if it has a two-sided inverse (equivalently,
  $f$ is a bijection and $f^{-1}$
  is a morphism).
\end{definition}

\begin{remark}
  We have the following properties of
  morphisms:
  \begin{enumerate}
    \item (Composition) If $f : X \to Y$
      and $g : Y \to Z$ are morphisms of
      open sets of affine varieties, then
      so is $g \circ f : X \to Z$.
    \item (Local on target) If $X \to Y$
      is a map of open sets of affine
      varieties such that there exists an
      open cover $\{U_i\}_{i \in I}$ of
      $Y$ with $f|_{f^{-1}(U_i)} : f^{-1}(U_i) \to U_i$
      a morphism for all $i \in I$, then
      $f$ is a morphism.
  \end{enumerate}
\end{remark}

\begin{prop}
  Let $X \subseteq \Affine^m$ and
  $Y \subseteq \Affine^n$ be affine
  varieties. Let $U \subseteq X$ and
  $V \subseteq Y$ be open sets. A map
  $f : U \to V$ is a morphism if and only
  if there exist
  $\varphi_1, \dots, \varphi_n \in \OO_X(U)$
  such that
  \[f(x) = (\varphi_1(x), \dots, \varphi_n(x)).\]
\end{prop}

\begin{proof}
  $(\Rightarrow)$ Let $U \subseteq \Affine^m_{x_i}$
  and $V \subseteq \Affine^n_{y_i}$.
  By the definition of a morphism,
  $y_i : V \to k$ satisfies
  \[
    \varphi_i := f^* y_i \in \OO_X(U),
  \]
  so we can write
  $f(x) = (\varphi_1(x), \dots, \varphi_n(x))$.

  $(\Leftarrow)$
  Assume there exist $\varphi_1, \dots, \varphi_n \in \OO_X(U)$
  such that
  $f(x) = (\varphi_1(x), \dots, \varphi_n(x))$.

  We first show that $f$ is continuous.
  Let $Z \subseteq V$ be a closed set.
  So we can write
  $Z = V(g_r, \dots, g_r)$ for some
  $g_1, \dots, g_r \in A(\Affine^n) \cong k[y_1, \dots, y_n]$.
  Now we have
  \begin{align*}
    f^{-1}(Z)
    = \{x \in U : f(x) \in Z\}
    &= \{
      x \in U : g_i(f(x)) = 0 \text{ for } i = 1, \dots, r
    \} \\
    &= \{
      x \in U : (f^* g_i)(x) = 0 \text{ for } i = 1, \dots, r
    \}.
  \end{align*}
  Note that $f^* g_i = g_i(\varphi_1, \dots, \varphi_n)$, which
  is regular since a composition of
  a polynomial with fractions of
  polynomials is again a fraction of
  polynomials. So
  $f^{-1}(Z)$ is closed in $U$.

  Now to show that $f$ is a morphism,
  it suffices to show that for
  any $W \subseteq Y$ open and
  $\varphi \in \OO_Y(W)$, we have
  $f^* \varphi \in \OO_X(f^{-1}(W))$.
  The proof of this is similar to before.
\end{proof}

\begin{example}
  We have the following:
  \begin{enumerate}
    \item Morphisms $\Affine^m \to \Affine^n$
      are of the form
      \[
        x \longmapsto (f_1(x), \dots, f_n(x))
      \]
      with $f_1, \dots, f_n \in \OO_{\Affine^m}(\Affine^m) = k[x_1, \dots, x_m]$.
    \item Write $\Affine^1_t$ to mean
      $\Affine^1$ with variable $t$.
      Then we can define
      $\Affine_t^1 \to V(y - x^2) \subseteq \Affine^2_{x,y}$
      by $t \mapsto (t, t^2)$.
      We can get an inverse
      $V(y - x^2) \to \Affine_t^1$ by
      $(x, y) \mapsto x$, so
      $\Affine_t^1$ and $V(y - x^2)$
      are isomorphic.
    \item Consider the map
      $g : \Affine^1_t \to V(x^2 - y^3) \subseteq \Affine^2_{x,y}$
      given by $t \mapsto (t^3, t^2)$.
      This map is bijective, but it is
      not an isomorphism. To see this,
      we can show that
      $(g^{-1})^* \varphi$ is not regular
      for some regular function $\varphi$
      on $\Affine_1^t$.
      For instance, we can take
      $\varphi = t$, so that
      \[
        (g^{-1})^*(t)
        =
        (x, y) \mapsto
        \begin{cases}
          x / y & \text{if } y \neq 0,\\
          0 & \text{otherwise},
        \end{cases}
      \]
      which we can see is not regular.
  \end{enumerate}
\end{example}

\section{Relation to Coordinate Rings}

\begin{remark}
  Let $X \subseteq \Affine^m$ and
  $Y \subseteq \Affine^n$ be affine
  varieties. Then a morphism
  $f : X \to Y$ of affine
  varieties induces a $k$-algebra
  morphism (called the \emph{pullback} of
  $f$)
  \begin{align*}
    f^* : A(Y)
    &\longrightarrow A(X) \\
    \varphi &\longmapsto f^* = \varphi \circ f
  \end{align*}
  with the properties
  $(g \circ f)^* = f^* \circ g^*$
  and $(\id_X)^* = \id_{A(X)}$, i.e.
  $X \mapsto A(X)$ is a
  contravariant functor.
\end{remark}

\begin{prop}
  The following map is
  a bijection:
  \begin{align*}
    \Hom_{\mathrm{aff,var}}(X, Y)
    &\overset{\Phi}{\longrightarrow}
    \Hom_{k\text{-}\mathrm{alg}}(A(Y), A(X)) \\
    f &\longmapsto f^*
  \end{align*}
\end{prop}

\begin{proof}
  Note that
  $A(X) \cong k[x_1, \dots, x_m]/I(X)$
  and $A(Y) \cong k[y_1, \dots, y_n]/I(Y)$.
  Given a morphism
  \begin{align*}
    f : X &\longrightarrow Y \\
    x &\longmapsto (\varphi_1(x), \dots, \varphi_n(x)),
  \end{align*}
  we can define
  $f^* \overline{y}_i = \varphi_i$.
  Conversely, given a
  $k$-algebra homomorphism
  $\phi : A(Y) \to A(X)$, we can set
  $\varphi_i = \phi(\overline{y}_i)$.
  Now consider the morphism defined by
  \begin{align*}
    f : X
    &\longrightarrow \Affine^n_{y_i} \\
    x &\longmapsto (\varphi_1(x), \dots, \varphi_n(x)).
  \end{align*}
  We claim that $f(X) \subseteq Y$.
  To see this, fix $x \in X$. If
  $h \in I(Y)$, then
  \[
    h(f(x))
    = h(\varphi_1(x), \dots, \varphi_n(x))
    = \phi(h)(x)
    = 0(x) = 0,
  \]
  so $f(X) \subseteq Y$. Thus we
  get a morphism $f : X \to Y$ by
  $x \mapsto (\varphi_1(x), \dots, \varphi_n(x))$ with
  $f^* y_i = \varphi_i$. One can check
  that this gives a map
  $\Psi : \Hom_{k\text{-}\mathrm{alg}}(A(Y), A(X)) \to \Hom_{\mathrm{aff,var}}(X, Y)$
  which is inverse to $\Phi$.
\end{proof}

\begin{example}
  We have the following:
  \begin{enumerate}
    \item Recall the morphism
      $g : \Affine^1_t \to V(y - x^2) \subseteq \Affine_{x, y}^2$
      given by $t \mapsto (t, t^2)$.
      The pullback is given by
      \begin{align*}
        g^* : \frac{k[x, y]}{(y - x^2)}
        &\longmapsto k[t] \\
        x &\longmapsto t \\
        y &\longmapsto t^2.
      \end{align*}
      Note that $g^*$ is an isomorphism
      of $k$-algebras, so
      $g$ is an isomorphism of affine varieties.
      This gives an alternative way of
      seeing this without writing down
      an inverse to $g$.
    \item Recall the morphism
      $h : \Affine^1_t \to V(x^2 - y^3) \subseteq \Affine_{x, y}^2$
      given by $t \mapsto (t^3, t^2)$.
      The pullback is
      \begin{align*}
        h^* : \frac{k[x, y]}{(x^2 - y^3)}
        &\longmapsto k[t] \\
        x &\longmapsto t^3 \\
        y &\longmapsto t^2.
      \end{align*}
      Note that $t \notin \im h^*$,
      so $h^*$ is not an isomorphism,
      so $h$ is not an isomorphism.
  \end{enumerate}
\end{example}

\begin{remark}
  There is a one-to-one correspondence
  between affine varieties (up to isomorphism)
  and finitely generated reduced
  $k$-algebras (up to isomorphism).

  To see this, observe that if $X \subseteq \Affine^n$
  is an affine variety, then
  $A(X) \cong k[x_1, \dots, x_n]/I(X)$. This
  is finitely generated, and reduced since
  $I(X)$ is radical. Conversely, let
  $A$ be a reduced finitely generated
  $k$-algebra. Then
  $A \cong k[y_1, \dots, y_m] / I$ since
  $A$ is finitely generated, and
  $I$ is radical since $A$ is reduced.
  Thus by Hilbert's nullstellensatz,
  $Y = V(I)$ satisfies
  $I(Y) = I(V(I)) = I$, so
  $A \cong A(Y)$.

  In more abstract language, this means
  that there is an equivalence of
  categories
  \[
    \mathrm{AffVar}
    \longleftrightarrow
    \mathrm{RedFGAlg}_k^\mathrm{op}.
  \]
\end{remark}
