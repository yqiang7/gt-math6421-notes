\chapter{Aug.~26 --- The Zariski Topology}

\section{Polynomial Functions and Subvarieties}

\begin{remark}
  Recall that a polynomial
  $f \in k[x_1, \dots, x_n]$
  gives a function
  $\Affine_k^n \to k$ by
  $a \mapsto f(a)$.
\end{remark}

\begin{prop}
  If $f, g \in k[x_1, \dots, x_n]$
  give the same function
  $\Affine_k^n \to k$, then
  $f = g$ in $k[x_1, \dots, x_n]$.
\end{prop}

\begin{proof}
  Assume $f = g$ as polynomial functions.
  Then $V(f - g) = \Affine_k^n$, so
  $\sqrt{(f - g)} = I(\Affine_k^n) = (0)$
  by Hilbert's nullstellensatz (note that
  we can also prove $I(\Affine_k^n) = (0)$
  directly, it is enough to have $k$ be an infinite field for this part). Thus
  $f - g = 0$, so $f = g$ in $k[x_1, \dots, x_n]$.
\end{proof}

\begin{remark}
  In the above proposition, we need
  $k$ to be an infinite field (e.g.
  if $k = \overline{k}$): Otherwise,
  there are only finitely many functions
  $\Affine_k^n \to k$, but infinitely
  many polynomials in $k[x_1, \dots, x_n]$.
\end{remark}

\begin{remark}
  The set of polynomials functions
  $\Affine_k^n \to k$ form a ring, and
  the above proposition implies that this
  ring is isomorphic to $k[x_1, \dots, x_n]$.
\end{remark}

\begin{definition}
  A \emph{polynomial function} on an
  affine variety $X \subseteq \Affine_k^n$
  is a function $\varphi : X \to k$ such
  that there exists $f \in k[x_1, \dots, x_n]$
  with $\varphi(a) = f(a)$ for
  every $a \in X$.
\end{definition}

\begin{definition}
  The \emph{coordinate ring} of $X$ is
  $A(X) = \{f : X \to k \mid f \text{ is a polynomial function}\}$,
  which is a ring under pointwise addition
  and multiplication.
\end{definition}

\begin{remark}
  Observe that there exists a surjective
  ring homomorphism
  \begin{align*}
    k[x_1, \dots, x_n] &\longrightarrow A(X) \\
    f &\longmapsto (a \mapsto f(a))
  \end{align*}
  with kernel $I(X)$. Thus we have
  $A(X) \cong k[x_1, \dots, x_n] / I(X)$.
\end{remark}

\begin{remark}
  We can now replace $\Affine_k^n$ and
  $k[x_1, \dots, x_n]$
  by $X$ and $A(X)$ to study
  \emph{subvarieties} of $X$.
\end{remark}

\begin{definition}
  Let $X \subseteq \Affine_k^n$
  be an affine variety.
  If $S \subseteq A(X)$
  is a subset, then define
  \[
    V_X(S) = \{a \in X : f(a) = 0 \text{ for all } f \in S\}.
  \]
  A subset of $X$ of this form is called
  an \emph{affine subvariety} of $X$.
  (Equivalently, these are the same as
  an affine variety $Y \subseteq \Affine_k^n$
  such that $Y \subseteq X$.) For $Y \subseteq X$ a
  subvariety, define
  \[
    I_X(Y) = \{f \in A(X) : f(a) = 0 \text{ for all } a \in Y\}.
  \]
\end{definition}

\begin{prop}
  There is a bijective correspondence
  between radical ideals in $A(X)$ and
  affine subvarieties of $X$
  given by $J \mapsto V_X(J)$ and
  $Y \mapsto I_X(Y)$.
\end{prop}

\begin{proof}
  See Homework 2.
\end{proof}

\section{The Zariski Topology}

\begin{definition}
  The \emph{Zariski topology} on
  $\Affine_k^n$ is the topology with
  closed sets $V(I) \subseteq \Affine_k^n$,
  where $I$ is an ideal in
  $k[x_1, \dots, x_n]$. (Equivalently,
  the closed sets are the affine varieties
  in $\Affine_k^n$.)
\end{definition}

\begin{remark}
  Note the following:
  \begin{enumerate}
    \item On $\Affine_k^1$, the closed
      sets are of the form: $\varnothing$,
      $\Affine_k^1$, or finite collections
      of points.
    \item When $k = \C$, then
      $X \subseteq \Affine_\C^n$ being
      Zariski closed implies that
      $X$ is closed in the analytic
      topology on $\Affine_\C^n$.
      In particular, the Zariski topology
      is coarser than the analytic topology.
    \item On $\Affine_k^2$, the closed
      sets are of the form:
      $\varnothing$, $\Affine_k^2$,
      finite collections of points, plane
      curves, and their finite unions.
  \end{enumerate}
\end{remark}

\begin{prop}
  The Zariski topology on $\Affine_k^n$
  is indeed a topology.
\end{prop}

\begin{proof}
  First note that $\varnothing = V((1))$ and
  $\Affine_k^n = V((0))$ are closed.
  For arbitrary intersections, note
  that $\bigcap_\alpha V(I_\alpha) = V(\sum_\alpha I_\alpha)$, and
  for finite unions, note that
  $\bigcup_{i = 1}^r V(I_i) = V(I_1 \cdots I_r)$.
\end{proof}

\begin{example}
  The Zariski topology on
  $\Affine_k^{n + m}$ is in general
  \emph{not} the product topology
  of the Zariski topologies on
  $\Affine_k^n$ and $\Affine_k^m$.
  Consider $V(y - x^2) \subseteq \Affine_k^2$,
  which is a closed set in the Zariski
  topology, but the only closed
  sets in $\Affine_k^1$ are either
  $\varnothing$, $\Affine_k^1$,
  or finite.
\end{example}

\begin{definition}
  If $X \subseteq \Affine_k^n$ is
  an affine variety, then we can define the
  \emph{Zariski topology} on $X$ in
  the following two equivalent ways:
  \begin{enumerate}
    \item take the subspace topology
      from the Zariski topology on
      $\Affine_k^n$;
    \item take the closed
      sets of $X$ to be of the form
      $V_X(I)$ for some ideal $I \le A(X)$.
  \end{enumerate}
  This is because an affine subvariety
  of $X$ is precisely the
  intersection of $X$ with an affine
  variety in $\Affine_k^n$.
\end{definition}

\begin{remark}
  Our goal now is to relate properties
  of the Zariski topology on $X$ to the
  ring $A(X)$,
  and then to the ideal $I(X) \le k[x_1, \dots, x_n]$.
\end{remark}

\begin{definition}
  A topological space $X$ is \emph{reducible}
  if we can write $X = X_1 \cup X_2$
  for some closed sets $X_1, X_2 \subsetneq X$.
  Otherwise, $X$ is called \emph{irreducible}.
\end{definition}

\begin{example}
  The plane curve
  $X = V(y^2 - x^2y) = V(y) \cup V(y - x^2)$
  is reducible.
\end{example}

\begin{remark}
  Note the following:
  \begin{enumerate}
    \item A disconnected topological space
      is reducible.
    \item Many topologies are reducible,
      e.g. $\C^n$, $\R^n$ with the analytic
      topology.
    \item If $X$ is irreducible and
      $U \subseteq X$ is a nonempty open
      set, then $\overline{U} = X$ (we have
      $\overline{U} \cup (X \setminus U) = X$).
  \end{enumerate}
\end{remark}
