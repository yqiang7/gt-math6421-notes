\chapter{Oct.~30 --- The Blow-Up, Part 2}

\section{The Blow-Up, Continued}

\begin{example}
  Let $\pi : B_{[1 : 0 : 0]} \Affine^2 \to \Affine^2$
  be the blow-up. Then
  $\pi^{-1}(U_0) \cong B_0 \Affine^2$ and
  $\pi^{-1}(U_i) \cong U_i$ for $i = 1, 2$.
  Note that $\pi$ is an isomorphism
  over $\Affine^2 \setminus \{[1 : 0 : 0]\}$
  and $\pi^{-1}([1 : 0 : 0]) = \PP^1$.
\end{example}

\begin{definition}[Blow-up for projective varieties]
  Let $X \subseteq \PP^n$ be a projective
  variety and $f_1, \dots, f_r \in S(X)$
  homogeneous of the same degree.
  Let $U = X \setminus V(f_1, \dots, f_r)$
  and define
  \begin{align*}
    f : U &\longrightarrow \PP^{r - 1} \\
    x &\longmapsto [f_1(x) : \dots : f_r(x)].
  \end{align*}
  Then the \emph{blow-up} of
  $X$ along $f_1, \dots, f_r$ is
  \[
    B_{f_1, \dots, f_r} X
    = \text{closure of } \Gamma_f \text{ in } X \times \PP^{r - 1}.
  \]
\end{definition}

\begin{example}
  We can compute the blow-up locally, e.g.
  for $U_0 = \{[x] \in X : x_0 \ne 0\}$,
  we have
  \[
    B_{f_1, \dots, f_r} X|_{\pi^{-1}(U_0)}
    \cong B_{f_1^i, \dots, f_r^i} U_0.
  \]
\end{example}

\begin{example}
  Consider the projection
  $\PP^2 \dashrightarrow \PP^1$
  given by $[x, y] \mapsto x$. We have
  \[
    \begin{tikzcd}
      & {B_{x, y} \PP^2 \subseteq \PP^2 \times \PP^1} \ar[dl, "\pi", swap] \ar[dr] \\
      \PP^2 \ar[rr, dashed, mapsto, "{[x, y, z] \mapsto [x : y]}", swap] & & \PP^1
    \end{tikzcd}
  \]
  Note that this resolves the indeterminacy.

  What is $B_{x, y} \PP^2$ isomorphic to?
  The answer is $B_{[0 : 0 : 1]} \PP^2$.
  For example, on charts:
  \begin{enumerate}
    \item on $\Affine^2_{x, y} \hookrightarrow \PP^2$
      given by $(x, y) \mapsto [x : y : z]$,
      we have $(x^i, y^i) = (x, y) = I(0)$;
    \item on $\Affine^2_{x, z} \hookrightarrow \PP^2$,
      we have $(x^i, y^i) = (x, 1) = (1)$;
    \item on $\Affine^2_{y, z} \hookrightarrow \PP^2$,
      we have $(x^i, y^i) = (1, y) = (1)$.
  \end{enumerate}
\end{example}

\begin{theorem}[Castelnuovo]
  If $X \dashrightarrow X'$ is a birational
  map between two smooth complex surfaces,
  then there exists a factorization
  \pagebreak
  \[
    \begin{tikzcd}[column sep=small]
      & Y \ar[dl, "f_1", swap] \ar[dr, "g_1"] \\
      \cdots \ar[d, "f_m", swap] & & \cdots \ar[d, "g_n"] \\
      X & & X'
    \end{tikzcd}
  \]
  with $f_i, g_i$ point blow-ups.
\end{theorem}

\section{Applications to Singularities}

\begin{example}
  Let $C = V(x^2 - y^3)$. Then we have
  $B_0 C \subseteq B_0 \Affine^2$ and
  \[
    \begin{tikzcd}
      B_0 C \ar[r, hook] \ar[d] & B_0 \Affine^2 \ar[d] \\
      C \ar[r, hook] & \Affine^2
    \end{tikzcd}
  \]
  We claim that there is an isomorphism
  \begin{align*}
    g : \Affine^1 &\overset{\cong}{\longrightarrow}
    B_0 C \\
    t &\longmapsto ((t^3, t^2), [t : 1]).
  \end{align*}
  To see this, note that
  $\im f = B_0 C$ (check this locally
  on charts), and $g$ has inverse given by
  \begin{align*}
    B_0 C &\longrightarrow \Affine^1 \\
    ((x, y), [s : t]) &\longmapsto t / s.
  \end{align*}
\end{example}

\begin{theorem}[Hironaka, Fields Medal result]
  If $X$ is an irreducible variety over
  a characteristic zero field, then
  there exists a sequence of blow-ups
  \[
    X_r \longrightarrow \cdots \longrightarrow X_1 \longrightarrow X
  \]
  along subvarieties such that
  $X_r$ is smooth.
\end{theorem}

\begin{definition}
  The \emph{tangent cone} of a variety
  $X$ at a point $a$ is
  \[
    C_a X = \text{cone over } \pi^{-1}(a) \subseteq \PP^{n - 1}
  \]
  with $\pi : B_a X \to X$ the blow-up.
\end{definition}

\begin{example}
  For $0 \in \Affine^n$, we have
  $\pi^{-1}(0) = \PP^{n - 1}$, so
  $C_0 \Affine^n = \Affine^n$.
\end{example}

\begin{example}
  The tangent cones for
  $X_1 = V(y - x^2)$ and $X_2 = V(y^2 - x^3)$
  are both the $x$-axis, whereas the
  tangent cone for
  $X_3 = V(y^2 - x^2 - x^3)$ is a union
  of two lines $V(y - x)$ and $V(y + x)$.
\end{example}

\begin{exercise}
  If $0 \in X \subseteq \Affine^n$ is a
  variety and $I(X) = (f)$, then show that
  $C_0(f) = V(f^{\mathrm{init}})$,
  where $f^{\mathrm{init}} := f_d$
  if we write
  $f = f_d + f_{d + 1} + \cdots$ with
  $f_i$ homogeneous of degree $i$ and
  $f_d \ne 0$.
\end{exercise}
