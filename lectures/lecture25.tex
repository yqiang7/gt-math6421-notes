\chapter{Nov.~25 --- Hilbert Polynomials}

\section{Hilbert Polynomials}

\begin{remark}
  Let $R = k[x_0, \ldots, x_n]$.
  Geometrically, the functions
  $0 \ne f \in I(X)_d$ correspond to
  degree $d$ hypersurfaces
  $X \subseteq V(f) \subseteq \PP^n$
  vanishing on $X$.
\end{remark}

\begin{theorem}\label{thm:hilbert-polynomial}
  Let $I \le R$ be a homogeneous ideal.
  Then there exists a unique polynomial
  $p_I(d) \in \Q[d]$ such that
  \begin{enumerate}
    \item $h_I(d) = p_I(d)$ for $d \gg 0$;
    \item $m := \deg(p_I) = \dim V_p(I)$;
    \item $p_I(d) = \frac{\mathrm{integer}}{m!} d^{m - 1} + \mathrm{lower\ order\ terms}$.
  \end{enumerate}
\end{theorem}

\begin{definition}
  For $X \subseteq \PP^n$ projective, set
  $p_X = p_{I(X)}$. Call
  $p_X$ the \emph{Hilbert polynomial} of $X$.
\end{definition}

\begin{definition}
  A \emph{numerical polynomial} is a
  polynomial $P \in \Q[d]$ such that
  $P(d) \in \Z$ for all
  $d \gg 0$. (We will see that it is
  equivalent to ask $P(d) \in \Z$
  for all $d \in \Z$.)
\end{definition}

\begin{example}
  The following are numerical polynomials:
  \begin{itemize}
    \item A constant polynomial which
      is integer-valued.
    \item $P(d) \in \Z[d]$.
    \item $P_n(d) = \binom{d}{n} = \frac{1}{n!} d(d - 1) \dots (d - n + 1)$.
      Also observe that
      \[
        P_n(d + 1)  - P_n(d)
        = \binom{d + 1}{n} - \binom{d}{n}
        = \binom{d}{n - 1}
        = P_{n - 1}(d).
      \]
  \end{itemize}
\end{example}

\begin{lemma}
  We have the following:
  \begin{enumerate}
    \item If $P$ is a numerical polynomial,
      then there exist $c_i \in \Z$ such
      that
      \[
        P(d)
        = c_0 \binom{d}{n}
        + c_1 \binom{d}{n - 1}
        + \dots + c_n.
      \]
    \item If $f : \N \to \N$ is a
      function such that
      $\Delta f(d) = f(d + 1) - f(d)$
      agrees with a numerical polynomial
      for $d \gg 0$, then $f$ agrees with
      a numerical polynomial for
      $d \gg 0$.
  \end{enumerate}
\end{lemma}

\begin{proof}
  (a) We induct on $\deg P $. The result is
  trivial when $\deg P = 0$. Thus assume
  by induction there exist $c_i \in \Z$
  such that
  \pagebreak
  \[
    \Delta P(d)
    = P(d + 1) - P(d)
    = c_0 \binom{d}{n - 1}
    + c_1 \binom{d}{n - 2}
    + \dots + c_{n - 1}.
  \]
  Set $Q(d) = c_0 \binom{d}{n} + c_1 \binom{d}{n - 1} + \dots + c_{n - 1} \binom{d}{1}$.
  As $\Delta \binom{d}{i} = \binom{d}{i - 1}$, we have
  $\Delta P = \Delta Q$, so
  $\Delta(P - Q) = 0$.
  So $P - Q$ is constant.
  As $P - Q$ is numerical, we have
  $c_n = P - Q \in \Z$. So
  $P = Q + c_n$.

  (2) See Hartshorne.
\end{proof}

\begin{remark}
  Now to prove Theorem \ref{thm:hilbert-polynomial},
  we want to show that
  $\Delta h_I(d) = h_I(d + 1) - h_I(d)$
  agrees with a numerical polynomial
  for $d \gg 0$.
\end{remark}

\begin{lemma}\label{lem:key-lemma}
  Let $I \le R$ be a homogeneous ideal
  and $0 \ne f \in R_e$. Assume there exists
  $d_0$ such that
  \begin{enumerate}
    \item[$\mathrm{(*)}$] If $g \in R_d$ with
      $d \ge d_0$
      and $fg \in I$, then $g \in I$,
  \end{enumerate}
  then $h_{I + (f)}(d) = h_I(d) - h_I(d - e)$
  for $d \gg 0$.
\end{lemma}

\begin{proof}
  Consider the short exact sequence
  \begin{center}
    \begin{tikzcd}
      R / I \arrow[r, " \cdot f "] &
      R / I \arrow[r] &
      R / (I + (f)) \arrow[r] & 0
    \end{tikzcd}
  \end{center}
  For $d - e \ge d_0$,
  taking degree $d$ parts, we get that
  \begin{center}
    \begin{tikzcd}
      0 \ar[r] &
      {[R / I]_{d - e}} \arrow[r, " \cdot f "] &
      {[R / I]_d} \arrow[r] &
      {[R / (I + (f))]_d} \arrow[r] & 0
    \end{tikzcd}
  \end{center}
  This gives the equality for the Hilbert
  functions when $d - e \ge d_0$.
\end{proof}

\begin{remark}
  When does $(*)$ hold? There are
  essentially two cases:
  \begin{enumerate}
    \item $X \subseteq \PP^n$ a
      projective variety with $I = I(X)$.

      We can write $X = X_1 \cup \dots \cup X_r$
      for the irreducible decomposition of
      $X$. So
      \[
        I = I(X) = I(X_1) \cap \dots \cap I(X_r).
      \]
      If $f$ does not vanish on any $X_i$,
      then $fg \in I$ implies that
      $fg$ vanishes on each $X_i$, so
      $g$ vanishes on each $X_i$. So in 
      particular, $g \in I$.
    \item $I \le R$ an arbitrary homogeneous
      ideal.

      By commutative algebra there exists a
      \emph{primary decomposition}
      \[
        I = I_1 \cap \dots \cap I_r,
      \]
      which each $I_j$ is \emph{primary}
      (i.e. $gh \in I_j$ implies
      $g \in I_j$ or  $h \in \sqrt{I_j}$).
      If we choose
      $V_p(I_j) \subsetneq V_p(f)$, then
      $(*)$ holds: If $gf \in I$,
      then $gf \in I_j$ for each $j$.
      By assumption $f \notin \sqrt{I_j}$
      (by the Nullstellensatz), so
      $g \in I_j$ for every $j$.
  \end{enumerate}
  In conclusion, we can always find $f \in R_1$
  such that $(*)$ holds.
\end{remark}

\begin{proof}[Proof of Theorem \ref{thm:hilbert-polynomial}]
  Fix $I \le R$ a homogeneous ideal.
  We want to show that $h_I(d)$ agrees
  with the monomial polynomial of
  $\deg \dim V_p(I)$ for $d \gg 0$.
  We induct
  on $\dim V_p(I)$. If $V_p(I) \ne \varnothing$,
  then
  \pagebreak
  \[
    \sqrt{I} \supseteq (x_0, \ldots, x_n),
  \]
  so $h_{I}(d) = 0$ for
  $d \gg 0$, which gives the result.
  Now assume $\dim V_p(I) \ge 0$.
  Choose $f \in R_1$ such that $(*)$ holds.
  Then Lemma \ref{lem:key-lemma} gives
  \[
    h_I(d) - h_I(d - 1)
    = h_{I + (f)}(d).
  \]
  As $\dim V_p(I + (f)) < \dim V_p(I)$,
  by induction we get that $h_{I + (f)}(d)$
  agrees with the numerical polynomial of
  degree $\dim V_p - 1$, so
  $h_I(d)$ agrees with a numerical polynomial
  of dimension $\deg \dim V_p(I)$.
\end{proof}

\section{Degree}
\begin{definition}
  For $I \le R$ a homogeneous ideal
  with $m = \dim V_p(I)$, the \emph{degree}
  of $I$ is the integer $\deg I \in \Z$
  such that
  \[
    h_I(d)
    = \frac{\deg I}{m!} d^m + \mathrm{lower\ order\ terms}.
  \]
  For a projective variety $X \subseteq \PP^n$,
  set $\deg X = \deg I(X)$, i.e.
  $h_X(d) = \frac{\deg X}{m!} d^m + \dots$,
  where $m = \dim X$.
\end{definition}

\begin{example}
  Consider the following:
  \begin{enumerate}
    \item $X = \PP^n$. Then
      $\deg \PP^n = 1$
      as $h_{\PP^n}(d) = \frac{1}{n!} d^n + \dots$.
    \item If $X \subseteq \PP^n$ is a
      hypersurface of degree $e$, then
      $\deg X = e$.
    \item If $X = \{p_1, \dots, p_r\} \subseteq \PP^n$, then
      $\deg X = r$.

      Also, if $\dim V_p(I) = 0$, then
      $P_I \ge P_{\sqrt{I}}
      = \# V_p(I)$.
      In particular, $\deg I \ge \# V_p(I)$.\footnote{For an example where these are
  consider $I = (x^2) \le k[x, y]$.
  But $\deg I = 2 > 1 = \#V_p(I)$.}
  \item Let $X \subseteq \PP^n$ be a linear
    space. After a linear change of
    coordinates, we may assume
    \[
      X = V(x_{r + 1}, \dots, x_n)
    \]
    for some $0 \le r \le n$. So
    $S(X) = k[x_0, \dots, x_n] / (x_{r + 1}, \dots, x_n) = k[x_0, \dots, x_r]
    = S(\PP^r)$.
  \end{enumerate}
  Now we have that
  \[
    h_X(d) = \binom{r + d}{r}
    + \frac{1}{r!} d^r + \mathrm{lower\ order\ terms},
  \]
  so we see that $\deg X = 1$.
\end{example}

\begin{exercise}
  If a projective variety $X \subseteq \PP^n$ 
  has $\deg X = 1$, then $X$ is a linear
  variety.
\end{exercise}

\begin{prop}
  If $X, Y \subseteq \PP^n$ are projective
  varieties of dimension $m$ with no
  common irreducible components, then
  \[
    \deg(X \cup Y) = \deg X + \deg Y.
  \]
\end{prop}

\begin{proof}
  We know $h_{X \cup Y}(d) = h_{I(X)}(d) + h_{I(Y)} - h_{I(X) + I(Y)}(d)$., so
  \[
    P_{X \cup Y}
    = P_X + P_Y - p_{I(X) + I(Y)}.
  \]
  Now $\deg P_X = \deg P_Y = m$
  and $\deg P_{I(X) + I(Y)} = \dim V_P(I(X) + I(Y)) = \dim (X \cap Y) < m$, so
  \[
    p_{X \cup Y}
    = \frac{\deg X}{m!} d^m
    + \frac{\deg Y}{m!} d^m
    + \mathrm{lower\ order\ terms}.
  \]
  Thus we get that
  $\deg(X \cup Y) = \deg X + \deg Y$.
\end{proof}
