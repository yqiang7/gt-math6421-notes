\chapter{Nov.~18 --- Lines on Hypersurfaces}

\section{Lines on Hypersurfaces in \texorpdfstring{$\PP^3$}{P3}}

\begin{remark}
  Let $X \subseteq \PP^3_{x : y : z : w}$ be a
  smooth hypersurface of degree $d$.
  How many lines are on $X$?
\end{remark}

\begin{example}
  Let $d = 1$. After a linear
  change of coordinates,
  we can assume $X = V(x)$.
  This contains infinitely many lines,
  as any two points are connected by
  a line in $X$.
\end{example}

\begin{example}
  Let $d = 2$. After a linear
  change of coordinates,
  $X = V(x^2 + y^2 + z^2 + w^2)$, which
  shows that all smooth degree $2$
  hypersurfaces are isomorphic.
  In particular, we may assume
  \[
    X = V(xw - yz) \subseteq \PP^3,
  \]
  which is the image of the
  Segre embedding
  $\PP^1 \times \PP^1 \to \PP^3$
  which sends
  $([a : b], [c : d]) \mapsto [ac : ad : bc : bd]$.
  Note that $\im([a : b] \times \PP^1)$
  and $\im(\PP^1 \times [c : d])$
  are lines. So
  $X$ contains infinitely many lines.
\end{example}

\begin{example}
  Let $d = 3$. To understand the
  moduli of cubics, we roughly consider
  $\PP / {\PGL_4}$, where
  $\PP$ is the projective space of
  $k[x, y, z, w]_3$. Then the
  expected dimension is
  \[
    \dim \PP - \dim \PGL_4
    = \left[\binom{3 + 3}{3} - 1\right]
    - [4^2 - 1]
    = 19 - 15 = 4.
  \]
  Thus we expect that there is not
  just one smooth cubic in $\PP^3$
  up to a linear change of coordinates.

  First consider the Fermat cubic
  $X = V(x^3 + y^3 + z^3 + w^3) \subseteq \PP^3$,
  and assume $\Char k \ne 3$.
  We compute the lines on $X$. A line
  $L \subseteq \PP^3$ corresponds to a
  point $L \in G(2, 4)$. After
  permuting the coordinates,
  \[
    L = \text{row span of }
    \begin{pmatrix}
      1 & 0 & a_2 & a_3 \\
      0 & 1 & b_2 & b_3
    \end{pmatrix}
  \]
  with $a_i, b_i \in k$. Thus we
  can write
  \[
    L =
    \{[s : t : sa_2 + tb_2 : s a_3 + tb_3] : [s : t] \in \PP^1\}.
  \]
  Now $L \subseteq X$ if and only if
  \[
    s^3 + t^3 + (s a_2 + t b_2)^3
    + (s a_3 + t b_3)^3 = 0
  \]
  for all $s, t \in k$. This
  happens if and only if the
  coefficients of the above expression
  (as a polynomial in $s, t$)
  vanish,
  which happens if and only if
  the following equations are satisfied:
  \[
    \begin{cases}
      A : a_2^3 + a_3^3 = -1, \\
      B : b_2^3 + b_3^3 = -1, \\
      C : a_2^2 b_2 = -a_3^2 b_3, \\
      D : a_2 b_2^2 = -a_3 b_3^2.
    \end{cases}
  \]
  Note that $a_2, a_3, b_2, b_3$
  cannot all be nonzero, since
  otherwise $C^2 / D$ gives
  $a_2^3 = -a_3^3$, contradicting
  $A$.
  Assume $a_2 = 0$. Then we get
  \[
    \begin{cases}
      a_3^3 = -1, \\
      b_3 = 0, \\
      b_2^3 = -1,
    \end{cases}
  \]
  i.e. $a_2 = 0, a_3 = -\omega^j, b_2 = -\omega^k, b_3 = 0$
  with $\omega$ a primitive 3rd root
  of unity and $0 \le j, k \le 2$. Thus
  \[
    L =
    \begin{pmatrix}
      1 & 0 & 0 & -\omega^j \\
      0 & 1 & -\omega^k & 0
    \end{pmatrix},
  \]
  which gives $9$ lines
  by varying $j, k$.
  Permuting the coordinates gives $18$
  more:
  \[
    \begin{pmatrix}
      1 & 0 & -\omega^j & 0 \\
      0 & 1 & 0 & -\omega^k
    \end{pmatrix} \quad \text{and} \quad
    \begin{pmatrix}
      1 & -\omega^j & 0 & 0 \\
      0 & 0 & 1 & -\omega^k
    \end{pmatrix}.
  \]
  Thus the Fermat cubic
  $V(x^3 + y^3 + z^3 + w^3) \subseteq \PP^3$
  contains exactly $27$ lines.
\end{example}

\begin{remark}
  A similar computation shows that
  $V(x^d + y^d + z^d + w^d) \subseteq \PP^3$
  contains $3d^2$ lines when
  $d \ge 3$ and $\Char k \nmid d$.
\end{remark}

\begin{theorem}\label{thm:lines-on-cubic-surface}
  We have the following:
  \begin{enumerate}
    \item Every smooth cubic surface
      in $\PP^3$ contains exactly
      $27$ lines.
    \item A general smooth hypersurface
      of degree $d > 3$ in $\PP^3$
      contains no lines.
  \end{enumerate}
\end{theorem}

\begin{remark}
  Let $N = \binom{3 + d}{d} - 1$ and
  $\PP = \PP^N$ be the projective
  space of $k[x_0, x_1, x_2, x_3]_d$.
  We call this the ``parameter space
  of hypersurfaces of degree $d$
  in $\PP^3$.''
  Define the \emph{incidence correspondence}
  \[
    \Gamma
    = \{([F], L) \in \PP \times G(2, 4) : L \in V(F)\}
    \subseteq \PP \times G(2, 4).
  \]
  Let $p : \Gamma \to \PP$ and
  $q : \Gamma \to G(2, 4)$
  be the projections. Then observe that:
  \begin{enumerate}
    \item $p^{-1}([F])$ is the
      set of lines contained in $V(F)$.
    \item $p(\Gamma)$ is the set of
      hypersurfaces containing a line.
  \end{enumerate}
\end{remark}

\begin{lemma}
  $\Gamma$ is closed in $\PP \times G(2, 4)$.
\end{lemma}

\begin{proof}
  On the chart $\Affine^4 \cong U_{1, 2} \subseteq G(2, 4)$,
  the points of $U_{1, 2}$ correspond to
  \[
    L =
    \begin{pmatrix}
      1 & 0 & a_2 & a_3 \\
      0 & 1 & b_2 & b_3
    \end{pmatrix}.
  \]
  Now $L \subseteq V(F)$ if and only if
  $F(s, t, s a_2 + t b_2, s a_3 + t b_3) = 0$
  for all $s, t \in k$. As before,
  this corresponds to the condition
  that the coefficients of
  $F(s, t, s a_2 + t b_2, s a_3 + t b_3)$
  as a polynomial in $s, t$ vanish.
  Write $F = \sum c_I x^I$, then
  this is the vanishing locus of a
  polynomial in $a_i, b_i, c_I$.
  So $\Gamma \cap (\PP \times U_{1, 2})$
  is closed in $\PP \times U_{1, 2}$.
  So $\Gamma$ is closed in
  $\PP \times G(2, 4)$ as it is
  closed in every chart.
\end{proof}

\begin{remark}
  Which of the maps
  $p : \Gamma \to \PP$ and
  $q : \Gamma \to G(2, 4)$
  is easier to analyze? Note
  that $\PGL_4$ acts on
  all three of these spaces, and
  the action of $\PGL_4$ is transitive
  on $G(2, 4)$ but not on $\PP$.
\end{remark}

\begin{lemma}
  For any $L \in G(2, 4)$, we have
  $q^{-1}(L) \cong \PP^\nu$
  where $\nu = \dim \PP - (d + 1)$.
\end{lemma}

\begin{proof}
  After a linear change of
  coordinates, we may assume
  \[
    L =
    \begin{pmatrix}
      1 & 0 & 0 & 0 \\
      0 & 1 & 0 & 0
    \end{pmatrix}
    \in G(2, 4),
  \]
  i.e. $L = \{[s : t : 0 : 0] : [s : t] \in \PP^1\}$.
  Now if we write $F = \sum c_I x^I$,
  we have
  $([F], L) \in q^{-1}(L)$
  if and only if $L \subseteq V(F)$,
  which happens if and only if
  the coefficients of $F$ corresponding
  to the monomials $x_0^j x_1^{d - j}$
  all vanish, i.e. that
  \[
    c_{d, 0, 0, 0}
    = c_{d - 1, 1, 0, 0}
    = \dots
    = c_{0, d, 0, 0} = 0.
  \]
  So $q^{-1}(L) \cong \PP^\nu$
  with $\nu = \dim \PP - (d + 1)$
  (where $(d + 1)$ is the number of
  zero coefficients above).
\end{proof}

\begin{remark}
  The next steps are the following,
  which will follow from dimension
  theory:
  \begin{enumerate}
    \item $\dim \Gamma = \dim q^{-1}(L) + \dim G(2, 4) = \dim \PP - (d + 1) + 4$.
    \item $\dim p(\Gamma) \le \dim \Gamma = \dim \PP - (d + 1) + 4$.
  \end{enumerate}
  So once we show this,
  we get that
  $p(\Gamma) \subsetneq \PP$
  for $d > 3$
  (i.e. when $(d + 1) - 4 > 0$).
\end{remark}

\section{More Dimension Theory}

\begin{theorem}
  Let $f : X \to Y$ be a surjective
  morphism between irreducible
  varieties with $n = \dim X$ and
  $m = \dim Y$. Then
  \begin{enumerate}
    \item $n \ge m$.
    \item $\dim f^{-1}(y) \ge n - m$ for $y \in Y$.
    \item There is an open set
      $\varnothing \ne U \subseteq Y$
      such that equality holds in
      (2) for all $y \in U$.
  \end{enumerate}
\end{theorem}

\begin{proof}
  (1) Since $f : X \to Y$ is surjective,
  there is an inclusion
  $f^* : K(Y) \to K(X)$. Now
  \[
    \dim Y = \trdeg_k K(Y)
    \le \trdeg_k K(X) = \dim X.
  \]
  (2) TODO: Fill in proof (from
  Lecture 24).

  (3) See Milne.
\end{proof}

\begin{prop}
  Let $f : X \to Y$ be a surjective
  morphism of varieties with $Y$
  irreducible and $f^{-1}(y)$
  irreducible of constant dimension for
  all $y \in Y$. Then $X$ is
  irreducible and $\dim X = \dim Y + \dim f^{-1}(y)$.
\end{prop}

\begin{proof}
  See Milne.
\end{proof}
