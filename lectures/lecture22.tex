\chapter{Nov.~13 --- Singularities}

\section{Tangent Spaces, Continued}
\begin{remark}
  Recall that the tangent space to a
  variety $X$ at $a \in X$ is
  \[
    T_a X =
    \Hom_k(\m_a / \m_a^2, k),
  \]
  where $\OO_{X, a} \supseteq \m_a = \{\varphi \in \OO_{X, a} : \varphi(a) = 0\}$.
  Note the following:
  \begin{enumerate}
    \item We have
      $\OO_{X, a} / \m_a \cong k$:
      We can define a map
      \begin{align*}
        \psi : \OO_{X, a} &\longrightarrow k \\
        \varphi &\longmapsto \varphi(a).
      \end{align*}
      This is surjective and has
      kernel $\m_a$, so
      $k \cong \OO_{X, a} / {\ker \psi} = \OO_{X, a} / \m_a$.
    \item Since $\m_a \cdot (\m_a / \m_a^2) = 0$,
      we have that
      $\m_a / \m_a^2$ is a module over
      $\OO_{X, a} / \m_a \cong k$.
  \end{enumerate}
\end{remark}

\begin{remark}
  How do we compute the tangent space?
  We can do the following:
  \begin{enumerate}
    \item Shrink $X$ so that it is
      affine: $a \in X \subseteq \Affine^n$
      where $X$ is closed.
    \item Translate so that  $a = 0$.
    \item Use $T_0 X = V(f_1 : f \in I(X))$.
  \end{enumerate}
\end{remark}

\begin{example}
  In $\Affine^2$, we have the following:
  \begin{center}
    \begin{tabular}{c|cc}
      $X$ & $V(y - x^2)$ & $V(y^2 - x^2 - x^3)$ \\
      $T_0 X$ & $V(y)$ & $V(0)$
    \end{tabular}
  \end{center}
\end{example}

\begin{remark}
  Since
  $C_a X \subseteq T_a X$, we have
  \[
    \dim T_a X \ge \dim C_a X
    = \codim_X \{a\}.
  \]
  Note that this is equal to
  $\dim X$ when $X$ is irreducible.
\end{remark}

\section{Singularities}

\begin{definition}
  Let $X$ be a variety and $a \in X$.
  \begin{enumerate}
    \item $X$ is \emph{smooth} (or
      \emph{non-singular}, \emph{regular})
      at $a$ if
      \[\dim T_a X = \codim_X \{a\}.\]
      Otherwise we say that $X$ is singular at $a$.
      \pagebreak
    \item $X$ is \emph{smooth}
      (or \emph{non-singular}, \emph{regular})
      if it is smooth at every point.
  \end{enumerate}
\end{definition}

\begin{example}
  We have the following:
  \begin{enumerate}
    \item For $\Affine^n$, we have
      $T_0 \Affine^n = \Affine^n$, so
      $\Affine^n$ is smooth at $0$.
      By translation, $\Affine^n$ is smooth.
    \item Let $C = V(y - x^2)$. Then
      $\dim T_0 C = 1 = \dim C$, so
      $C$ is smooth at $0$.
    \item Let $0 \in X \subseteq \Affine^n$
      be a hypersurface with $I(X) = (f)$.
      Then $T_0 X = V(f_1)$, so
      \begin{align*}
        \text{$X$ is singular at $0$}
        &\iff \dim T_0 X > n - 1 \\
        &\iff \dim T_0 X = n \\
        &\iff T_0 X = \Affine^n \\
        &\iff f_1 = 0.
      \end{align*}
    \item $D = V(x^2 - y^2 - y^3)$
      is singular at $0$.
  \end{enumerate}
\end{example}

\begin{remark}[Relation to commutative algebra]
  A local ring $(R, \m)$ is
  \emph{regular} if it is Noetherian and
  \[
    \dim R = \dim_{R / \m} \m / \m^2.
  \]
  We will use the following
  result from commutative
  algebra result without proof:
  Regular local rings are always integral
  domains.

  In algebraic geometry, note that
  for a variety $X$ and $a \in X$,
  \begin{align*}
    \dim T_a X &= \dim_k \m_a / \m_a^2 \\
    \codim_X \{a\} &= \dim \OO_{X, a}.
  \end{align*}
  So if $X$ is non-singular at $a$, then
  $\OO_{X, a}$ is regular, so $\OO_{X, a}$
  is a domain. Thus $X$ is irreducible
  at $a$.
\end{remark}

\begin{prop}[Jacobi criterion]\label{prop:jacobi}
  Let $a \in X \subseteq \Affine^n$ be a
  point on an affine variety with
  $I(X) = (f_1, \dots, f_r)$.
  Then $X$ is smooth at $a$ if and only if
  the Jacobian
  \[
    (\partial f_i / \partial x_j (a))_{i, j}
    \text{ has rank }
    \ge n - \codim_X \{a\}.
  \]
  Moreover, this happens if and only if
  the rank is equal to
  $n - \codim_X \{a\}$.
  This implies that the rank is always at
  most $n - \codim_X \{a\}$ regardless
  of smoothness.
\end{prop}

\begin{remark}
  Assume that $X$ is irreducible. Then
  \[
    \Sing X
    := \{a \in X : X \text{ is singular at } a\}
    = V_X(\text{$(n - \dim X)$-dimensional minors of $\Jac(f)$})
  \]
  where $I(X) = (f_1, \dots, f_r)$.
\end{remark}

\begin{example}
  Let $X$ be an irreducible hypersurface
  in $\Affine^n$ with $I(X) = (f)$. Then
  \[
    \Sing X =
    V_X(\partial f / \partial x_1, \dots, \partial f / \partial x_n)
    = V(f, \partial f / \partial x_1, \dots, \partial f / \partial x_n).
  \]
\end{example}

\begin{example}
  Let $C = V(y^2 - x^3)$, then
  \[
    \Sing C = V(y^2 - x^3, -3x^2, 2y)
    = \{0\}.
  \]
\end{example}

\begin{example}[Whitney umbrella]
  Let $X = V(x^2 - y^2 z) \subseteq \Affine^3$. Then
  \[
    \Sing X
    = V(x^2 - y^2 z, 2x, -2yz, -y^2)
    = V(x, y),
  \]
  which is a line in $\Affine^3$.
\end{example}

\begin{proof}[Proof of Proposition \ref{prop:jacobi}]
  By translating, we may assume
  $a = 0$. Now $I(X) = (f_1, \dots, f_r)$,
  so
  \[
    [f_i]_1
    = \sum_{j = 1}^n \frac{\partial f_i}{\partial x_j}(0) x_j.
  \]
  As $T_0 X = V([f_1]_1, \dots, [f_r]_1)$, we have
  $T_0 X = \ker J$ with
  $J = (\partial f_i / \partial x_j(0))_{i, j}$.
  So
  \[
    \dim T_0 X = n - \rank J.
  \]
  As $\dim T_0 X \ge \codim_X \{a\}$,
  we get $\codim_X\{a\} \le n - \rank J$,
  i.e. $\rank J \le n - \codim_X \{a\}$.
  Furthermore, equality holds if and only if
  $\dim T_0 X = \codim_X \{a\}$, which
  happens if and only if
  $X$ is smooth at $a$.
\end{proof}

\begin{corollary}[Generic smoothness]
  The smooth locus of a variety is a
  dense open set.
\end{corollary}

\begin{proof}
  Since the statement is local, we may
  assume $X$ is affine and irreducible
  (it suffices to show that the smooth
  locus is nonempty and open in each
  irreducible component, away from the
  intersections with other irreducible
  components).
  Thus it suffices to show that the smooth
  locus is open and nonempty.

  Openness follows from the Jacobi
  criterion, since
  \[
    \Sing X
    = V_X(\text{$(r \times r)$}-\text{minors of $\Jac(f)$})
  \]
  with $I(X) = (f_1, \dots, f_r)$ and
  $r = n - \dim X$.
  So $\Smooth(X)$ is open.

  To see that it is nonempty, we first
  assume that $X \subseteq \Affine^n$
  is a hypersurface with
  $I(X) = (f)$. Now
  \[
    \Sing X
    = V(f, \partial f / \partial x_1, \dots, \partial f / \partial x_n)
    \subsetneq X.
  \]
  This is because $f$ is irreducible
  (as $X$ is irreducible) and
  $\deg \partial f / \partial x_i < \deg f$,
  so
  \[
    \sqrt{(f)}
    = (f)
    \subsetneq (f, \partial f / \partial x_1, \dots, \partial f / \partial x_n),
  \]
  which implies that
  $X \supsetneq \Sing X$. So
  $\Smooth(X) \ne \varnothing$. In general,
  any irreducible variety $X$ is birational
  to a hypersurface, so
  $\Smooth(X) \ne \varnothing$
  in general as well.\footnote{Let $H$ be the hypersurface. The birational map $\varphi : X \dashrightarrow H$ is defined on some open set $U \subseteq H$, so $\Smooth(X)$ being open dense implies $U \cap \Smooth(H) \ne \varnothing$, so $\Smooth(X) \supseteq \Smooth(X) \cap \varphi^{-1}(U) = \varphi^{-1}(U \cap \Smooth(H)) \ne \varnothing$.}
\end{proof}

\begin{exercise}[Projective Jacobi criterion]
  Let $X \subseteq \PP^n$ be a projective
  variety and
  \[I(X) = (f_1, \dots, f_r) \le k[x_0, \dots, x_n], \quad f_i \text{ homogeneous}.\]
  Then $X$ is smooth at $a$ if and only if
  $(\partial f_i / \partial x_j (a))_{i, j}$ has
  rank $\ge n - \codim_X \{a\}$.
  Moreover, the inequality
  $\le n - \codim_X \{a\}$ always holds.
\end{exercise}

\begin{example}[Fermat hypersurface]
  Let $F = x_0^d + \dots + x_n^d$ with
  $n \ge 2$ and $\Char k \nmid d$. Then
  for $X = V(F) \subseteq \PP^n$, we have
  (using a relaxed version of the
  projective Jacobi criterion: if
  we only assume $X = V(f_1, \dots, f_r)$,
  then we only get
  $\rank(\partial f_i / \partial x_j (a)) \ge n - \codim_X \{a\}$
  implies smoothness at $a$)
  \[
    \Sing X
    = V_X(d x_0^{d - 1}, \dots, d x_n^{d - 1})
    = V(F, d x_0^{d - 1}, \dots, d x_n^{d - 1})
    = \varnothing.
  \]
\end{example}

\begin{exercise}
  Fix $n \ge 2$ and $d \ge 1$
  ($\Char k \nmid d$).
  Let $N = \dim k[x_0, \dots, x_n]_d - 1 = \binom{n + d}{d} - 1$.
  View
  \[
    \PP^N_{a_I}
    = \text{parameter space of degree $d$ hypersurfaces in $\PP^n$},
  \]
  where $I = (i_0, \dots, i_n) \in \N^{n + 1}$
  with $\sum i_j = d$, i.e.
  $p \in [a_I] \in \PP^N$
  corresponds to
  \[
    F_p = \sum a_I x^I
    \in k[x_0, \dots, x_n]_d.
  \]
  Then there exists some nonempty open set
  $U \subseteq \PP^N$ such that
  $V(F_p) \subseteq \PP^n$ is smooth
  for every $p \in U$. Colloquially,
  one says that a ``general'' hypersurface
  of degree $d$ in $\PP^n$ is smooth.

  Hint: Consider the \emph{incidence
  correspondence} (write $F = \sum c_I x^I$)
  \[
    \Gamma
    = \left\{([F], [a]) : a \in V(F)\right\}
    = \left\{([c_I], [a]) : \sum c_I a^I = 0\right\}
    \subseteq \PP^N \times \PP^n
  \]
  and the projection maps
  $p : \Gamma \to \PP^N$,
  $q : \Gamma \to \PP^n$.
  Then for $[F] \in \PP^N$, the set
  $p^{-1}(\{F\}) = [F] \times V(F)$,
  and $q^{-1}([a])$ corresponds to
  the hypersurfaces containing
  $[a]$.
\end{exercise}
