\chapter{Sept.~18 --- Pre-varieties}

\section{More on Tensor Products}

\begin{prop}
  If $B$ and $C$ are $A$-algebras (i.e.
  there are ring homomorphisms
  $f : A \to B$ and $g : A \to C$ which
  give $a \cdot b := f(a) b$ and
  $a \cdot c = g(a) c$, then
  $B \otimes_A C$ is also an $A$-algebra
  with
  \[
    (b \otimes c) \cdot (b' \otimes c')
    := (bb') \otimes (cc')
  \]
  and ring homomorphism
  $A \to B \otimes_A C$ given by
  $a \mapsto a \otimes 1$ (equivalently,
  $1 \otimes a$).
\end{prop}

\begin{prop}
  $k[x_1, \dots, x_m] \otimes_k k[y_1, \dots, y_n] \cong k[x_1, \dots, x_m, y_1, \dots, y_n]$.
\end{prop}

\begin{prop}
  $(k[x_1, \dots, x_m] / I) \otimes_k (k[y_1, \dots, y_n] / J) \cong k[x_1, \dots, x_m, y_1, \dots, y_n) / \langle I, J \rangle$.
\end{prop}

\begin{proof}
  Set $R = k[x_1, \dots, x_m]$ and
  $S = k[y_1, \dots, y_n]$. We have a
  short exact sequence
  \[
    0 \longrightarrow
    I \longrightarrow R \longrightarrow R / I \longrightarrow 0.
  \]
  Applying the right exact
  functor $\otimes_k (S / J)$ (and
  vice versa with $J$ and $\otimes R$) gives
  an exact sequence
  \begin{center}
    \begin{tikzcd}
      & R \otimes_k J \ar[d]\\
      & R \otimes_k S \ar[d]\\
      I \otimes_k (S / J) \ar[r] & R \otimes_k (S / J) \ar[r] \ar[d] & (R / I) \otimes_k (S / J) \ar[r] & 0 \\
      & 0
    \end{tikzcd}
  \end{center}
  So we have
  \[
    (R / I) \otimes_k (S / J)
    \cong
    \frac{R \otimes_k (S / J)}
    {\im(I \otimes_k (S / J) \to R \otimes_k (S / J))}
    \cong \frac{R \otimes_k S}{I \otimes_k S + R \otimes_k J},
  \]
  which is the desired result since
  $I \otimes_k S + R \otimes_k J = \langle I, J \rangle$
  in $R \otimes_k S$.
\end{proof}

\begin{prop}[Milne]\label{prop:tensor-domain}
  Let $B$ and $C$ be finitely generated
  $k$-algebras with $k = \overline{k}$.
  \begin{enumerate}
    \item If $B$ and $C$ are
      reduced, then so is $B \otimes_k C$.
    \item If $B$ and $C$ are domains,
      then so is $B \otimes_k C$.
  \end{enumerate}
\end{prop}

\begin{remark}
  We need $k = \overline{k}$ in
  Proposition \ref{prop:tensor-domain}.
  Consider the domains
  $\R[x] / (x^2 + 1)$,
  $\R[y] / (y^2 + 1)$. Then
  \[
    \frac{\R[x]}{(x^2 + 1)}
    \otimes_{\R}
    \frac{\R[y]}{(y^2 + 1)}
    \cong
    \frac{\R[x, y]}{(x^2 + 1, y^2 + 1)},
  \]
  which is not a domain since
  $(\overline{x - y})(\overline{x + y}) = \overline{x^2 - y^2} = \overline{-1 - (-1)} = 0$.
\end{remark}

\begin{corollary}
  If $X \subseteq \Affine^m$ and
  $Y \subseteq \Affine^n$ are affine
  varieties, then
  \begin{enumerate}
    \item $I(X \times Y) = \langle I(X), I(Y) \rangle \subseteq k[x_1, \dots, x_m, y_1, \dots, y_n]$.
    \item $A(X \times Y) \cong A(X) \otimes_k A(Y)$.
    \item If $X$ and $Y$ are irreducible,
      then $X \times Y$ is irreducible.
  \end{enumerate}
\end{corollary}

\begin{proof}
  Observe that
  $V(I(X), I(Y)) = X \times Y \subseteq \Affine^{m + n}$, so
  $I(X \times Y) = \sqrt{\langle I(X), I(Y) \rangle}$.
  Now we know that
  $I(X)$ and $I(Y)$ are radical in
  $k[x_1, \dots, x_m]$ and
  $k[y_1, \dots, y_n]$, respectively, so
  \[
    \frac{k[x_1, \dots, x_m]}{I(X)}
    \quad \text{and} \quad
    \frac{k[y_1, \dots, y_n]}{I(Y)}
  \]
  are reduced. By Proposition \ref{prop:tensor-domain},
  we get that
  \[
    \frac{k[x_1, \dots, x_m, y_1, \dots, y_n]}{\langle I(X), I(Y) \rangle}
    \cong
    \frac{k[x_1, \dots, x_m]}{I(X)}
    \otimes_k
    \frac{k[y_1, \dots, y_n]}{I(Y)}
  \]
  is reduced, so
  $\langle I(X), I(Y) \rangle$ is radical.
  Thus $I(X \times Y) = \langle I(X), I(Y) \rangle$, so
  $(1)$ holds.

  Now $(1)$ implies $(2)$, and
  $(3)$ follows since
  $X$ and $Y$ being irreducible implies
  $A(X)$ and $A(Y)$ are domains, which
  implies $A(X \times Y)$ is a domain
  by Proposition \ref{prop:tensor-domain}
  and $(2)$, so
  $X \times Y$ is irreducible.
\end{proof}

\section{Pre-varieties}

\begin{remark}
  We will now head towards defining
  a \emph{variety}, which is roughly
  finitely many affine varieties glued
  together (a \emph{pre-variety}) with a separation
  condition (an algebraic version of
  Hausdorff).
\end{remark}

\begin{definition}
  A \emph{pre-variety} is a ringed
  space $(X, \OO_X)$ such that
  there exists a finite open cover
  $X = \bigcup_{i = 1}^s U_i$ with
  $(U_i, \OO_X|_{U_i})$ being an
  affine variety for all $i = 1, \dots, s$.
  A \emph{morphism} of pre-varieties
  \[
    f : (X, \OO_X) \longrightarrow (Y, \OO_Y)
  \]
  is a morphism of the ringed spaces.
  We will often just write
  $X$ for $(X, \OO_X)$.
\end{definition}

\begin{remark}
  We call $\varphi \in \OO_X(U)$ with
  $U \subseteq X$ open and
  $\varphi : U \to k$ a \emph{regular function}
  on $U$.
\end{remark}

\begin{example}
  Consider the following:
  \begin{enumerate}
    \item An affine variety $X$ is a
      pre-variety.
      However, we have multiple
      choices for the open cover:
      We can take $X = X$, or
      $X = \bigcup_{i = 1}^s D(f_i)$
      with $f_i \in \OO_X(X)$ and
      $(f_1, \dots, f_s) = (1)$
      in $\OO_X(X)$.
    \item $\PP^n_k = (\Affine^{n + 1} \setminus \{0\}) / k^*$ is
      a pre-variety. We will see that
      $\PP^1_k = \Affine^1_k \cup \{\mathrm{pt}\}$.
    \item Let $X = V(I) \subseteq \Affine^n$
      be an affine variety and $U \subseteq X$ open.
      Set
      \[
        \OO_U(V) = \{\varphi : V \to k \mid \varphi \text{ is regular}\}.
      \]
      Then $(U, \OO_U)$ is a pre-variety.
      To see this, note that
      $U = \bigcup_{f \in I(X \setminus U)} D(f)$.
      Since $U$ is Noetherian (hence is
      compact), we can find a finite subcover,
      so $U = \bigcup_{i = 1}^s D(f_i)$
      for some $f_i \in A(X)$.
    \item (Gluing) Let
      $X_1$ and $X_2$ be affine varieties,
      and $U_{1,2} \subseteq X_1$,
      $U_{2,1} \subseteq X_2$ open, with
      an isomorphism
      \[
        f : U_{1, 2} \longrightarrow U_{2, 1}.
      \]
      Then we get a pre-variety by
      setting
      $X = (X_1 \sqcup X_2) / {\sim}$,
      where
      $a \sim f(a)$ for all $a \in U_{1, 2}$,
      $f(a) \sim a$ for all
      $a \in U_{2, 1}$, and
      $b \sim b$ for all $b \in X_1 \sqcup X_2$.
      We have quotient maps
      \[
        j_1 : X_1 \longrightarrow X
        \quad \text{and} \quad
        j_2 : X_2 \longrightarrow X.
      \]
      Now $X$ is a topological space
      with the quotient topology, and 
      $j_1, j_2$ are open embeddings
      (i.e. have open images and are
      homeomorphisms onto their images).
      Define a sheaf of rings
      $\OO_X$ on $X$ by
      \[
        \OO_X(U)
        = \{\varphi : U \to k \mid j_1^* \varphi \in \OO_{X_1}(j^{-1}(U)) \text{ and } j_2^* \varphi \in \OO_{X_2}(j_2^{-1}(U))\}.
      \]
      One can check
      $X = j_1(X_1) \cup j_2(X_2)$ and
      $(j(X_i), \OO_X|_{j_i(X_i)}) \cong (X_i, \OO_{X_i})$,
      so $(X, \OO_X)$ is a pre-variety.
  \end{enumerate}
\end{example}

\begin{example}
  Consider $X_1 = \Affine^1_x$
  and $X_2 = \Affine^1_y$, with
  $U_{1, 2} = \Affine^1_x \setminus \{0\}$
  and $U_{2, 1} = \Affine^1_y \setminus \{0\}$. Define
  \begin{align*}
    f : U_{1, 2}
    &\longrightarrow U_{2, 1} \\
    x &\longmapsto 1 / x.
  \end{align*}
  Then we can
  take $\PP^1_k = (X_1 \sqcup X_2) / {\sim}$.
  What are the regular functions
  $\PP^1_k \to k$? We should
  get only the constant functions
  (When $k = \C$,
  $\PP^1_{\C}$ is compact, so a
  holomorphic function $f : \PP^1_{\C} \to \C$
  is bounded. By restricting to $X_1$,
  we get a bounded map
  $f : \C \to \C$, so $f$ is constant
  by Liouville's theorem).

  In general,  let $j_i : X_i \to \PP^1_k$
  be the quotient maps. Fix
  $\varphi \in \OO_{\PP^1}(\PP^1)$.
  Now
  \[
    \varphi|_{X_1} := j_1^* \varphi
    = \sum_{i \ge 0} a_i x^i \quad
    \text{and} \quad
    \varphi|_{X_2} := j_2^* \varphi
    = \sum_{i \ge 0} b_i y^i
  \]
  for some $a_i, b_i \in k$.
  They must agree on the overlap, so
  \[
    \sum_{i \ge 0} a_i x^i
    = \sum_{i \ge 0} b_i (1 / x)^i
  \]
  as functions on $\Affine^1 \setminus \{0\}$.
  Since $\OO_{\Affine^1}(\Affine^1 \setminus \{0\}) = k[x^{\pm 1}]$,
  we have $a_i = b_i = 0$ for
  $i > 0$ and $a_0 = b_0$ (since
  the powers of $x^{\pm 1}$ are $k$-linearly
  independent), so
  $\varphi$ is a constant function.

  If we instead took
  $f : U_{1, 2} \to U_{2, 1}$ to be
  $x \mapsto x$, then
  $X = (X_1 \sqcup X_2) / {\sim}$
  is the ``bug-eyed line'' with
  two points $0, 0'$ at the origin (this is
  the \emph{line with two origins} when
  $k = \R$, which is not Hausdorff.)
  Note that $X \setminus \{0, 0'\} \cong \Affine^1 \setminus \{0\}$.
  In our case, the bad property is
  that there exist two morphisms
  \[
    g_1, g_2 : \Affine^1 \longrightarrow X
  \]
  such that $g_1|_{\Affine^1 \setminus \{0\}} = g_2|_{\Affine^1 \setminus \{0\}}$
  and $g_1 \ne g_2$, i.e.
  ``limits are not unique'' on $X$.
  Note that a similar computation shows
  $\OO_X(X) \cong k[x]$, so in particular,
  $X \ncong \PP^1_k$.
\end{example}

\begin{example}[General gluing procedure]
  Let $I$ be a finite index set,
  $(X_i, \OO_{X_i})$ affine varieties,
  $U_{i, j} \subseteq X_i$ open sets, and
  $f_{i, j} : U_{i, j} \to U_{j, i}$
  isomorphisms
  for each $i, j \in I$. With some
  compatibility conditions, we can also
  glue the $X_i$ together to get
  a pre-variety.
\end{example}
