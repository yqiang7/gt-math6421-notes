\chapter{Hilbert Functions}

\section{Lines on Hypersurfaces in \texorpdfstring{$\PP^3$}{P3}, Continued}

\begin{proof}[Proof of Theorem \ref{thm:lines-on-cubic-surface}]
  Recall that we let
  $\PP = \PP^N$ be the projective
  space of $k[x_0, x_1, x_2, x_3]_d$,
  which is the ``parameter space of
  degree $d$ hypersurfaces in $\PP^3$.''
  Here $N = \binom{3 + d}{d}$.
  We defined the incidence
  correspondence
  \[
    \Gamma = \{([F], L) \in \PP \times G(2, 4) : L \subseteq V(F)\}
  \]
  and the projections
  $p : \Gamma \to \PP$ and
  $q : \Gamma \to G(2, 4)$.
  Then we want to show:
  \begin{enumerate}
    \item For $d = 3$, that
      $p : \Gamma \to \PP$ is surjective
      and has finite fibers
      over a dense open subset of
      $\PP$.
    \item For $d > 3$, $p(\Gamma)$
      is closed with
      $p(\Gamma) \subsetneq \PP$
      (this is the locus of $\PP$ which
      parametrizes hypersurfaces with
      lines).
  \end{enumerate}
  Last time we showed
  $q^{-1}(L) \cong \PP^\nu$ with
  $\nu = \dim \PP - (d + 1)$ for all
  $\Gamma \in G(2, 4)$.
  So $\Gamma$ is irreducible and
  \[
    \dim \Gamma = \dim \PP - (d + 1)
    + \dim G(2, 4)
    = \dim \PP - (d + 1) + 4.
  \]
  Thus if $d > 3$, then
  $\dim \Gamma < \dim \PP$. Thus
  $p : \Gamma \to \PP$ is
  \emph{not} surjective. So
  $p(\Gamma) \subsetneq \PP$
  is a strict closed subset.
  If $d = 3$, then
  $\dim \Gamma = \dim \PP$.
  We have already computed that
  \[
    \# p^{-1}(\text{Fermat cubic})
    = 27,
  \]
  so $\dim p^{-1}(\text{Fermat cubic}) = 0$.
  So we must have $p : \Gamma \to \PP$
  surjective (otherwise $\dim p(\Gamma) < \dim \PP = \dim \Gamma$, so
  $\Gamma \to p(\Gamma)$ has all fibers of
  positive dimension). Now there exists an
  open $\varnothing \ne U \subseteq \PP$
  with $\dim p^{-1}(y) = \dim \Gamma - \dim \PP = 0$
  for all $y \in U$. This completes
  the proof.
\end{proof}

\begin{remark}
  The above proves that a general
  smooth cubic surface contains
  exactly $27$ lines. To conclude that
  \emph{all} smooth cubic surfaces
  contain $27$ lines, one can take the
  following approaches:
  \begin{enumerate}
    \item Show that $p : \Gamma \to \PP$
      is ``smooth'' over the locus of
      smooth cubics, which implies
      that all fibers have the same
      dimension. This is done in
      Gathmann.
    \item Show that any smooth
      cubic surface is isomorphic to
      \[
        X = B_{p_1, \dots, p_6}
        \PP^2
      \]
      with $p_1, \dots, p_6 \in \PP^2$
      in general position (i.e. no
      $3$ points lie on a line and no
      $6$ points lie on a conic). Then
      we obtain
      $27$ lines as follows:
      \begin{itemize}
        \item $6$ from
          $\pi^{-1}(p_i)$ for $i = 1, \dots, 6$.
        \item $15 = \binom{6}{2}$
          from $\widetilde{L}_{i, j}$
          (the strict transform of $L_{i, j}$),
          where $L_{i, j}$
          is the line through $p_i$ and
          $p_j$.
        \item $6 = \binom{6}{5}$
          from $\widetilde{Q}_i$,
          where $Q_i$ is the (unique)
          conic through
          $\{p_1, \dots, p_6\} \setminus \{p_i\}$.
      \end{itemize}
  \end{enumerate}
\end{remark}

\section{Hilbert Functions}

\begin{definition}
  For a hypersurface $X \subseteq \PP^n$ 
  with $I(X) = (f)$, we define
  the \emph{degree} of $X$ to be
  \[
    \deg X = \deg f.
  \]
\end{definition}

\begin{theorem}[Bezout's theorem]
  If $C, D \subseteq \PP^2$ are
  distinct irreducible curves, then
  \[
    \#(C \cap D)
    \le (\deg C)(\deg D),
  \]
  and equality holds when we count
  multiplicity.
\end{theorem}

\begin{remark}
  We want a version of degree for
  any projective variety.
\end{remark}

\begin{definition}
  Let $I \le k[x_0, \dots, x_n]$ be a
  homogeneous ideal. Then the
  \emph{Hilbert function} of
  $I$ is
  \begin{align*}
    h_I : \N
    &\longrightarrow \N \\
    d &\longmapsto
    \dim [k[x_0, \dots, x_n] / I]_d
    = \dim k[x_0, \dots, x_n]_d - \dim I_d.
  \end{align*}
  Similarly, if $X \subseteq \PP^n$
  is a projective variety, we define
  its \emph{Hilbert function} to be
  \begin{align*}
    h_X : \N
    &\longrightarrow \N \\
    d &\longmapsto
    \dim S(X)_d.
  \end{align*}
\end{definition}

\begin{remark}
  Since $S(X) = k[x_0, \dots, x_n] / I(X)$,
  we have $h_X = h_{I(X)}$.
\end{remark}

\begin{example}
  For $\PP^n$, we can compute that
  \begin{align*}
    h_{\PP^n}(d)
    = \dim k[x_0, \dots, x_n]_d
    &= \binom{n + d}{d}
    = \frac{(n + d)!}{n! d!} \\
    &= \frac{(d + 1) \dots (d + n)}{n!}
    = \frac{1}{n!} d^n
    + \text{lower order terms}.
  \end{align*}
\end{example}

\begin{example}
  For $X = \{[1 : 0 : \dots : 0]\} \subseteq \PP^n$, we have
  $I(X) = (x_1, \dots, x_n)$, so
  $S(X) \cong k[x_0]$ and
  \[
    h_X(d) = \dim k[x_0]_d
    = \dim k x_0^d
    = 1.
  \]
\end{example}

\begin{example}
  Let $I = (x^2) \subseteq k[x, y]$.
  View $I$ as cutting out
  $[0 : 1]$ with multiplicity $2$.
  For $d \ge 1$,
  \[
    [k[x, y] / I]_d
    = k \overline{y^d}
    \oplus k \overline{x y^{d - 1}}
  \]
  since $\overline{x^2} = 0$
  in the quotient.
  Thus we see that
  \[
    h_I(d) =
    \begin{cases}
      1 & \text{if } d = 0, \\
      2 & \text{if } d \ge 1.
    \end{cases}
  \]
\end{example}

\begin{example}
  Let $X \subseteq \PP^n$ be a
  hypersurface of degree $e$, i.e.
  $I(X) = (f)$ with $\deg f = e$. Then
  \[
    I(X)_d =
    \begin{cases}
      0 & \text{if } d < e, \\
      f \cdot k[x_0, \dots, x_n]_{d - e} & \text{if } d \ge e.
    \end{cases}
  \]
  So for $d \ge e$, we have
  \begin{align*}
    h_X(d)
    &= \dim k[x_0, \dots, x_n]_d
    - \dim I(X)_d
    = \dim k[x_0, \dots, x_n]_d
    - \dim k[x_0, \dots, x_n]_{d - e} \\
    &= \binom{n + d}{d}
    - \binom{n + d - e}{d - e}
    = \frac{d^n}{n!} - \frac{(d - e)^n}{n!}
    + O(d^{n - 2})
    = \frac{ed^{n - 1}}{(n - 1)!}
    + \text{lower order terms}.
  \end{align*}
  Note that $e = \deg X$ and
  $n - 1 = \dim X$.
\end{example}

\begin{remark}
  We have seen the following:
  \begin{enumerate}
    \item $\displaystyle h_{\PP^n}(d) = \frac{1}{n!} d^n + \text{lower order terms}$.
    \item $\displaystyle h_{\text{hyp.~of deg.~$e$}}(d) = \frac{e}{(n - 1)!} d^{n - 1} + \text{lower order terms}$.
    \item $h_{\mathrm{pt}}(d) = 1$
      for all $d \ge 1$.
  \end{enumerate}
  In these cases, $h_X$ agrees
  with a polynomial for $d \gg 0$, and
  the degree of the polynomial is
  $\dim X$.
\end{remark}

\begin{prop}\label{prop:hilbert-function-additivity}
  For two homogeneous ideals
  $I, J \le k[x_0, \dots, x_n]$, we have
  \[
    h_{I \cap J} + h_{I + J}
    = h_I + h_J.
  \]
\end{prop}

\begin{proof}
  Use that $R = k[x_0, \dots, x_n]$
  satisfies the following
  short exact sequence of $R$-modules:
  \begin{center}
    \begin{tikzcd}[row sep=0pt]
      0 \ar[r] & R / (I \cap J) \ar[r] & R / I \times R / J \ar[r] & R / (I + J) \ar[r] & 0 \\
               & \overline{f} \ar[r, mapsto] & (\overline{f}, \overline{f}) \\
               & & (\overline{g}, \overline{h}) \ar[r, mapsto] & \overline{g - h}
    \end{tikzcd}
  \end{center}
  Check this as an exercise. Then
  restricting to degree $d$
  components gives the result.
\end{proof}

\begin{remark}
  If $X, Y \subseteq \PP^n$ are
  distinct projective varieties, then
  \[I(X \cup Y) = I(X) \cap I(Y)
  \quad \text{and} \quad \sqrt{I(X) + I(Y)} \supseteq (x_0, \dots, x_n)\]
  So by Proposition \ref{prop:hilbert-function-additivity},
  \[
    h_{X \cup Y}(d)
    = h_{I(X) \cap I(Y)}(d)
    = h_{I(X)}(d) + h_{I(Y)}(d)
    - h_{I(X) + I(Y)}(d)
    = h_X(d) + h_Y(d)
  \]
  for $d \gg 0$, since
  $h_{I(X) + I(Y)}(d) = 0$ for
  $d \gg 0$ by Lemma \ref{lem:hilbert-function-vanishing}.
\end{remark}

\begin{example}
  Let $X = \{p_1, \dots, p_r\} \subseteq \PP^n$.
  Then
  \[
    h_X(d) = \sum_{i = 1}^r h_{\{p_i\}}(d)
    = r, \quad \text{for } d \gg 0.
  \]
\end{example}

\begin{lemma}\label{lem:hilbert-function-vanishing}
  If $I \le k[x_0, \dots, x_n]$ is a
  homogeneous ideal with
  $\sqrt{I} \supseteq (x_0, \dots, x_n)$,
  then $h_I(d) = 0$ for $d \gg 0$.
\end{lemma}

\begin{proof}
  Since $\sqrt{I} \supseteq (x_0, \dots, x_n)$,
  there exists $k > 0$ such that
  $x_i^k \in I$ for every $i$.
  So if $d \ge k(n + 1)$, then
  $I_d = k[x_0, \dots, x_n]_d$.
  So $h_I(d) = \dim k[x_0, \dots, x_n]_d - \dim I_d = 0$
  for $d \ge k(n + 1)$.
\end{proof}
