\chapter{Oct.~28 --- The Blow-Up}

\section{Rational Varieties}

\begin{exercise}
  Show that two irreducible varieties
  $X$ and $Y$
  are birational if and only if
  $K(X) \cong K(Y)$ as $k$-algebras.
\end{exercise}

\begin{example}
  Let $C = V(x^2 - y^3)$, and define
  rational maps
  \[
    f : \Affine^1 \dashrightarrow C,\
    t \longmapsto (t^3, t^2)
    \quad \text{and} \quad
    g : C \dashrightarrow \Affine^1,\
    (x, y) \longmapsto x / y.
  \]
  For $t \ne 0$, we have
  $g(f(t)) = t^3 / t^2$, so
  $g \circ f = \id_{\Affine^1}$. For
  $(a, b) \in C$,
  \[
    (f \circ g)(a, b)
    = (a^3 / b^3, a^2 / b^2)
    = (a^3 / a^2, b^3 / b^2)
    = (a, b),
  \]
  where the second equality follows
  since $a^2 = b^3$ on $C$. Thus
  $f \circ g = \id_C$, so
  $f$ is birational.
\end{example}

\begin{theorem}
  Any irreducible algebraic variety
  is birational to a hypersurface in
  $\PP^n$.
\end{theorem}

\begin{definition}
  An irreducible algebraic variety is
  \emph{rational} if it is birational to
  $\PP^n$.
\end{definition}

\begin{example}
  Rational varieties include
  $\PP^n$, $\Affine^n$, $G(d, n)$,
  $\PP^m \times \PP^n$, etc.
\end{example}

\begin{remark}
  Which hypersurfaces in $\PP^n$ are
  rational? This is an open question. It
  is known that if $X \subseteq \PP^n$
  is a smooth hypersurface over a
  field of characteristic $0$ and
  degree $d > n + 1$, then $X$ is \emph{not}
  rational.
\end{remark}

\section{The Blow-Up}

\begin{example}
  Consider the following birational map:
  \begin{align*}
    \Affine^2
    &\dashrightarrow \PP^1 \\
    (x_1, x_2) &\longmapsto [x_1 : x_2].
  \end{align*}
  Why does this not extend to a morphism
  $\Affine^2 \to \PP^1$?
  The issue is where $(0, 0)$ would go.
  Let us try to make it extend. Consider
  $U = \Affine^2 \setminus \{0\}$
  and the map
  \begin{align*}
    f : U &\longrightarrow \PP^1 \\
    (x_1, x_2) &\longmapsto [x_1 : x_2].
  \end{align*}
  Consider the graph $\Gamma_f = \{(x, y) \in U \times \PP^1 : x_1 y_2 = x_2 y_1\}$,
  and the \emph{blow-up}
  \[
    B_0 \Affine^2 := \overline{\Gamma}_f \subseteq \Affine^2 \times \PP^1,
  \]
  with projection maps
  $\pi : B_0 \Affine^2 \to \Affine^2$
  and $B_0 \Affine^2 \to \PP^1$. Then
  \begin{enumerate}
    \item $B_0 \Affine^2 = \{(x, y) \in \Affine^2 \times \PP^1 : x_1 y_2 - x_2 y_1 = 0\}$.

      Let $Y$ be the right-hand side.
      Since $\Gamma_f \subseteq Y$, it is
      immediate that $B_0 \Affine^2 = \overline{\Gamma}_f \subseteq \overline{Y} = Y$.
      For the reverse inclusion, it suffices
      to show $Y$ is irreducible
      and $\Gamma_f$ is open in $Y$.
      By the equation for $Y$,
      \[
        Y \setminus V(x_1, x_2)
        = \Gamma_f,
      \]
      which shows $\Gamma_f$ is open in $Y$.
      For irreducibility, set
      $U_i = \{(x, y) \in Y : y_i \ne 0\}$.
      Then $U_i \cong \Affine^2$:
      \begin{align*}
        U_1 &\longrightarrow \Affine^2 \\
        (x, y) &\longmapsto (x_1, x_2 / y_1) \\
        ((x_1, x_1 x_2), (1 : x_2)) &\mathrel{\reflectbox{\ensuremath{\longmapsto}}} (x_1, x_2).
      \end{align*}
      So each $U_i$ is irreducible.
      As $U_1 \cap U_2 \ne \varnothing$ 
      and $Y = U_1 \cap U_2$,
      $Y$ is irreducible.
      Thus $B_0 \Affine^2 \supseteq Y$.
    \item $\pi^{-1}(\Affine^2 \setminus \{0\}) \to \Affine^2 \setminus \{0\}$
      is an isomorphism.

      For this, we can give an inverse
      $\Affine^2 \setminus \{0\} \to \pi^{-1}(\Affine^2 \setminus \{0\})$
      by $(x_1, x_2) \mapsto ((x_1, x_2), [x_1 : x_2])$.
    \item $\pi^{-1}(\{0\}) = \{0\} \times \PP^1 \cong \PP^1$.

      This follows from (1).
  \end{enumerate}
\end{example}

\begin{remark}
  An alternative perspective is
  to think of $B_0\Affine^2$ as
  \[B_0 \Affine^2 = \{(p, \ell) \in \Affine^2 \times \PP^1 : p \in \ell\}.\]
  where we think of $\PP^1$ as lines
  in $\Affine^2$ through $0$.
\end{remark}

\begin{example}
  In higher dimensions, we define
  $B_0 \Affine^n = \overline{\Gamma}_f \subseteq \Affine^n \times \PP^{n-1}$,
  where $f : \Affine^n \setminus \{0\} \to \PP^{n - 1}$
  is given by $x \mapsto [x]$.
  Let $\pi : B_0 \Affine^n \to \Affine^n$
  be the projection. Then the same
  argument as before shows:
  \begin{enumerate}
    \item $B_0 \Affine^n = \{(x, y) \in \Affine^n \times \PP^{n - 1} : x_i y_i - x_j y_j = 0 \text{ for all } 1 \le i, j \le n\}$.
    \item $\pi^{-1}(\Affine^n \setminus \{0\}) \to \Affine^n \setminus \{0\}$
      is an isomorphism.
    \item $\pi^{-1}(\{0\}) = \{0\} \times \PP^{n - 1} \cong \PP^{n - 1}$. This
      is called the \emph{exceptional locus}.
  \end{enumerate}
\end{example}

\begin{remark}
  Define charts $U_i = \{(x, y) \in B_0 \Affine^n : y_i \ne 0\}$.
  Then we have
  \begin{center}
    \begin{tikzcd}
      \Affine^n \ar[d, "\cong", swap]
      & x \ar[d, mapsto] \\
      U_1 \ar[d] & ((x_1, x_1 x_2, \dots, x_1 x_n), [1 : x_2 : \cdots : x_n]) \ar[d, mapsto] \\
      \Affine^n & (x_1, x_1 x_2, \dots, x_1 x_n)
    \end{tikzcd}
  \end{center}
  Check that $\Affine^n \to U_1$
  is in fact an isomorphism.
\end{remark}

\begin{example}
  Let $n = 2$ and $\pi : B_0 \Affine^2 \to \Affine^2$.
  What happens to
  $C := V(y^2 - x^3) \subseteq \Affine^2$
  under the blow-up?
  We have $\pi^{-1}(C) = (\{0\} \times \PP^1) \cup \overline{\pi^{-1}(C \setminus \{0\})}$.
  We can compute this explicitly
  on charts. Write
  \begin{center}
    \begin{tikzcd}
      \Affine^2 \ar[r] \ar[dr, "\pi_1", swap] & U_1 \subseteq B_0 \Affine^2 \ar[d, "\pi"] \ar[r, equal] & \{((x, y), [s : t]) \in \Affine^2 \times \PP^1 : xt - ys = 0\} \\
    &= \Affine^2
    \end{tikzcd}
  \end{center}
  where $\Affine^2 \to U_1$ is given by
  $(x, y) \mapsto ((x, x y), [1 : y])$
  and $\pi_1$ is given by
  $(x, y) \mapsto (x, x y)$. Then
  \[
    \pi_1^{-1}(C)
    = V((x y)^2 - x^3)
    = V(x^2 (y^2 - x))
    = V(x) \cup V(y^2 - x)
  \]
  In particular, we see that the
  piece $\overline{\pi^{-1}(C \setminus \{0\})} = V(y^2 - x)$
  no longer has a singularity at $0$.
\end{example}

\begin{definition}
  Let $X \subseteq \Affine^n$ be an
  affine variety, $f_1, \dots, f_r \in A(X)$,
  and $U = X \setminus V(f_1, \dots, f_r)$.
  We define the \emph{blow-up} of $X$ along
  $f_1, \dots, f_r$ to be
  \[
    B_{f_1, \dots, f_r} X
    := \widetilde{X}
    := \text{closure of } \Gamma_f \text{ in } X \times \PP^{n - 1}
  \]
  where $f : U \to \PP^{r - 1}$
  is given by $x \mapsto [f_1(x) : \cdots : f_r(x)]$,
\end{definition}

\begin{remark}
  We have the following:
  \begin{enumerate}
    \item[0.] When $X = \Affine^n$ and
      $f_1, \dots, f_r = x_1, \dots, x_n$,
      then $B_{x_1, \dots, x_n} \Affine^n = B_0 \Affine^n$.
    \item $B_{f_1, \dots, f_r} X \subseteq \{(x, y) \in X \times \PP^{n - 1} : f_i(x) y_j - f_j(x) y_i = 0\}$.
      Equality does not necessarily hold.
    \item $\pi$ is an isomorphism
      over $U$.

      To see this, use that
      $\Gamma_f$ is closed in
      $U \times \PP^{r - 1}$ and
      the projection $\pr_1 : \Gamma_f \to U$
      is an isomorphism to get
      that $\overline{\Gamma}_f \cap (U \times \PP^{r - 1}) = \Gamma_f$.
      So $\pi^{-1}(U) = \Gamma_f \cong U$.
    \item If $X$ is irreducible and
      $U \ne \varnothing$, then $\pi$ is birational.

      For this, use $(1)$ to get
      $B_{f_1, \dots, f_r}$ is irreducible
      and $\pi$ being an isomorphism
      over $U$.
    \item If $(f_1, \dots, f_r) = (f_1', \dots, f_s')$
      as ideals in $A(X)$, then there is
      an isomorphism
      \[
        B_{f_1, \dots, f_r} X
        \overset{\cong}{\longrightarrow}
        B_{f_1', \dots, f_s'} X.
      \]
      Hint: Write $f_j' = \sum_{i = 1}^r g_{i, j} f_i$.
      Use the $g_{i, j}$ to get a
      morphism $\Gamma_f \to \Gamma_{f'}$,
      $(x, y) \mapsto (x, \big[\sum_{i = 1}^r g_{i, 1}(x) y_i : \cdots : \sum_{i = 1}^r g_{i, s}(x) y_i\big])$,
      then use this to get a
      morphism $B_{f_1, \dots, f_r} X \to B_{f_1', \dots, f_s'} X$.
      We can obtain an inverse similarly
      by swapping
      the roles of $f_i$ and $f_j'$.
      See Gathmann 9.16.
    \item (Local) If $W \subseteq X$ is
      open and $W \setminus V(f_1, \dots, f_r)$
      is dense in $W$, then
      \[
        B_{f_1, \dots, f_r} X|_{\pi^{-1}(W)}
        \cong B_{f_1|_W, \dots, f_r|_W} W.
      \]
      (Use the properties of the graph.)
  \end{enumerate}
\end{remark}

\begin{remark}
  We can further generalize the blow-up
  as follows:
  \begin{enumerate}
    \item (Affine case) Let
      $X$ be affine, $I \le A(X)$, and
      $Z \subseteq X$ closed. Then we
      can define $B_I X := B_{f_1, \dots, f_r} X$
      with $f_1, \dots, f_r$ generators of
      $I$ (the isomorphism class is independent of the choice
      of generators), and
      $B_Z X = B_{I(Z)} X$
      with $I(Z) \le A(X)$ the ideal
      consisting of regular functions
      vanishing on $Z$.
    \item Let $X$ be a variety and $Z \subseteq X$ 
      a closed subvariety. Choose an affine
      open cover
      $X = \bigcup_i U_i$, and define
      $B_Z X$ to be the $B_{Z \cap U_i} U_i$
      glued together along
      $B_{Z \cap U_i \cap U_j} (U_i \cap U_j)$.
      (Use that the blow-up can be computed
      locally and glue these together to
      get $B_Z X$.)
  \end{enumerate}
\end{remark}
