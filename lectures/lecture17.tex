\chapter{Oct.~21 --- The Grassmannian}

\section{Pl\"ucker Coordinates}

\begin{remark}
  Consider coordinates on $\PP^{\binom{n}{d} - 1}$
  with respect to
  \[
    \{
      e_I = e_{i_1} \wedge \cdots \wedge e_{i_d}
      : I = \{i_1 < \cdots < i_d\} \subseteq \{1, \ldots, n\}
    \}.
  \]
  So when $d = 2$, $n = 3$, and
  $W = \Span\{e_1 + e_2, e_1 + e_3\}$, then
  \[
    p_{n, d}(W)
    = k(e_1 \wedge e_3 - e_1 \wedge e_2 + e_2 \wedge e_3)
    = [1 : -1 : 1] \in \PP^2.
  \]
  Note that $p_{n, d}(W)$ encodes the
  determinants of the $2 \times 2$ minors
  of
  \[
    \begin{bmatrix}
      1 & 1 & 0 \\
      1 & 0 & 1
    \end{bmatrix}.
  \]
\end{remark}

\begin{remark}
  We will now try to describe the image
  of $p_{n, d} : G(d, n) \to \PP^{\binom{n}{d} - 1}$.
  Note that $0 \ne \omega \in \Lambda^d V$
  is in the image if and only if $\omega$
  is \emph{pure}, i.e. $\omega = v_1 \wedge \cdots \wedge v_d$
  for some $v_1, \ldots, v_d \in V$.
\end{remark}

\begin{prop}\label{prop:rank-wedge}
  For $0 \ne \omega \in \Lambda^d V$, the map
  \begin{align*}
    g : V &\longrightarrow \Lambda^{d + 1} V \\
    v &\longmapsto w \wedge v
  \end{align*}
  has rank $\ge n - d$. Furthermore,
  $\omega$ is pure if and only if the rank
  is exactly $n - d$.
\end{prop}

\begin{proof}
  After a change of basis, we may
assume that $\ker g = \Span\{e_1, \dots, e_r\}$
with $1 \le r \le n$. Write
\[
  \omega = \sum_{I \subseteq \{1, \dots, n\}} a_I e_I.
\]
Note that $e_i \wedge \omega = 0$ for
$1 \le i \le r$, so $a_I = 0$ if
there exists $1 \le i \le r$ with $i \notin I$. So
\[
  \omega = \sum_{\substack{\{1, \dots, r\} \subseteq I \subseteq \{1, \dots, n\}\\|I| = d}} a_I e_I.
  \]
  As $\omega \ne 0$, we have $r \le d$.
  So $\rank g = n - r \ge n - d$.
  Furthermore, if $r = d$, then
  $\omega = a_{\{1, \dots, d\}} e_1 \wedge \cdots \wedge e_d$.
  Conversely, if $\omega$ is pure, a
  similar computation shows that
  $\rank g = n - d$.
\end{proof}

\begin{example}
  Let $n = 3$, $d = 2$, $\omega \in \Lambda^2 V$.
  Consider a map
  \begin{align*}
    g : V &\longrightarrow \Lambda^3 V \\
    v &\longmapsto \omega \wedge v.
  \end{align*}
  Then $\rank g \le 1$. By Proposition
  \ref{prop:rank-wedge}, $\rank g = 1$ and
  $\omega$ is pure.
\end{example}

\begin{example}
  Let $n = 4$, $d = 2$, and
  $\omega = e_1 \wedge e_2$. Then
  \[
    g(a_1 e_1 + a_2 e_2 + a_3 e_3 + a_4 e_4)
    = a_3 e_1 \wedge e_2 \wedge e_3
    + a_4 e_1 \wedge e_2 \wedge e_4.
  \]
  So we have $\rank g = 2 = 4 - 2$.
\end{example}

\begin{corollary}\label{cor:plucker-closed}
  The image of
  $p_{n, d} : G(d, n) \to \PP^{\binom{n}{d} - 1}$
  is a closed set. In particular, this
  endows $G(d, n)$ with the structure of
  an algebraic variety.
\end{corollary}

\begin{proof}
  Let $[\omega] \in \PP^{\binom{n}{d} - 1}$
  where $0 \ne \omega = \sum_{|I| = d} b_I e_I$.
  Then $[\omega]$ is in $\im p_{n, d}$
  if and only if $\omega \in \Lambda^d V$
  is pure, which happens if and only if
  the map
  \begin{align*}
    g_\omega : V &\longrightarrow \Lambda^{d + 1} V \\
    \sum a_i e_i &\longmapsto \sum_{|I| = d} b_I e_I \wedge \sum a_i e_i
  \end{align*}
  has rank $n - d$. Note that $g_\omega$
  is given by some matrix in the $b_I$.
  This matrix has rank $n - d$ if and only
  if the $n - d + 1$ minors of this matrix
  vanish (we already know the rank
  is at least $n - d$ by Proposition
  \ref{prop:rank-wedge}).
  These give equations cutting out
  $\im p_{n, d}$ in $\PP^{\binom{n}{d} - 1}$,
  so $\im p_{n, d}$ is closed in
  $\PP^{\binom{n}{d} - 1}$.
\end{proof}

\begin{example}
  Let $n = 3$, $d = 2$. As all
  $\omega \in \Lambda^2 V$ are pure,
  $G(2, 3) \to \PP^{\binom{n}{d} - 1}$
  is an isomorphism.
\end{example}

\section{Charts of the Grassmannian}

\begin{remark}
  For $I \subseteq \{1, \dots, n\}$
  with $|I| = d$, let
  \[
    U_I \subseteq G(d, n) \subseteq \PP^{\binom{n}{d} - 1}
  \]
  denote the open set on which the
  $e_I$ coordinate does not vanish.
\end{remark}

\begin{prop}
  $U_I \cong \Affine^{d(n - d)}$.
\end{prop}

\begin{proof}
  After a change of coordinates, let
  $I = \{n - d + 1, n - d + 2, \dots, n\}$.
  Then define a map
  \begin{align*}
    h_I : \Affine^{d(n - d)} &\longrightarrow U_I \subseteq \PP^{\binom{n}{d} - 1} \\
    A = (a_{i, j}) &\longmapsto
    p_{n, d}([A \mid I_d]),
  \end{align*}
  where $A$ is a $d \times (n - d)$ matrix
  and $I_d$ is the $d \times d$ identity matrix.
  This map is a morphism, and $h_I$ has inverse
  given by
  \[
    h_I^{-1}
    \Big(
      \Big[\sum_{|J| = d} b_J e_J\Big]
    \Big)
    = (\pm b_{(I \setminus \{i\}) \cup \{j\}} / b_I)_{\substack{1 \le i \le d \\ 1 \le j \le n - d}}.
  \]
  We can see that this is also a morphism,
  so $h_I$ is an isomorphism
  $\Affine^{d(n - d)} \to U_I$.
\end{proof}

\begin{remark}
  Note that $G(d, n) = \bigcup_{|I| = d} U_I$.
  This gives us the following:
  \begin{enumerate}
    \item $\dim G(d, n) = d(n - d)$.
    \item $G(d, n)$ is irreducible,
      as $U_I$ is irreducible and
      $U_I \cap U_J \ne \varnothing$
      (for the latter, it suffices to find
      some $d \times n$ matrix $A$ whose
      $I$ and $J$ minors are both nonzero).
  \end{enumerate}
\end{remark}

\begin{example}
  We have
  $G(2, 4) \hookrightarrow \PP^{\binom{4}{2} - 1} = \PP^5$.
  Note that $\dim G(2, 4) = 4$ and
  $\dim \PP^5 = 5$. One can additionally
  show that $G(2, 4)$ is cut out
  by a single equation via the
  \emph{Pl\"ucker relations}.
\end{example}

\section{Birational Maps}

\begin{remark}
  Many non-isomorphic varieties are
  isomorphic on an open set. Consider:
  \begin{enumerate}
    \item $\PP^{n + m}$ and
      $\PP^n \times \PP^m$ both contain
      $\Affine^{n + m}$ as a dense open
      set.
    \item $G(d, n)$ and
      $\PP^{d(n - d)}$
      both contain $\Affine^{d(n - d)}$
      as a dense open set.
    \item Consider the curve defined by
      \begin{align*}
        f : \Affine^1 &\longrightarrow
        C = V(y^2 - x^3) \subseteq \Affine^2_{x, y} \\
        t &\longmapsto (t^2, t^3).
      \end{align*}
      Note that $f$ is injective but not an
      isomorphism, as
      \[
        f^{-1}(x, y) =
        \begin{cases}
          y / x & \text{if } x \ne 0, \\
          0 & \text{otherwise},
        \end{cases}
      \]
      which is not regular at the origin.
      But this does tell us that
      $\Affine^1 \setminus \{0\} \cong C \setminus \{(0, 0)\}$.
  \end{enumerate}
  So we want to study maps defined on an
  open set.
\end{remark}

\begin{definition}
  Let $X$ and $Y$ be irreducible varieties.
  Then a \emph{rational map}
  $f : X \dashrightarrow Y$ is a
  morphism $f : U \to Y$ with $U \subseteq Y$
  a nonempty open set, up to the
  equivalence
  $f_1 : U_1 \to Y \sim f_2 : U_2 \to Y$
  if $f_1, f_2$ agree on a
  nonempty open set
  $W \subseteq U_1 \cap U_2$.
\end{definition}

\begin{remark}
  Note that if
  $f_1 : U_1 \to Y \sim f_2 : U_2 \to Y$,
  then they agree on $U_1 \cap U_2$
  by the identity principle. So
  $f_1, \dots, f_2$ are equivalent
  to a morphism $U_1 \cap U_2 \to Y$.
  So there exists a maximal nonempty
  open $U \subseteq X$ on which $f$ is a
  morphism.
\end{remark}

\begin{remark}
  If $f, g \in \OO_X(X)$, then we will see
  that the map
  $X \dashrightarrow \Affine^1$
  given by $x \mapsto f(x) / g(x)$
  is rational.
\end{remark}
