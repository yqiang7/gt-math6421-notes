\chapter{Sept.~9 --- Germs and Sheaves}

\section{More on Localization}
\begin{prop}
  If $X$ is an affine variety and
  $f \in A(X)$ is nonzero, then
  $\OO_X(D(f)) \cong A(X)_f$.
\end{prop}

\begin{proof}
  We define a ring homomorphism as follows:
  \begin{align*}
    A(X)_f &\longrightarrow \OO_X(D(f)) \\
    \frac{g}{f^m} &\longmapsto \left( x \mapsto \frac{g(x)}{f^m(x)} \right).
  \end{align*}
  To check that this is well-defined,
  assume $g / f^m \sim h / f^n$ in
  $A(X)_f$. So there exists $k \ge 0$
  such that
  \[
    f^k (g f^n - h f^m) = 0
    \quad \text{in } A(X).
  \]
  So $gf^n - hf^m = 0$ as functions
  $D(f) \to k$, so $g / f^m = h / f^n$
  as functions $D(f) \to k$. Thus
  their images agree in $\OO_X(D(f))$, so
  the map is well-defined.

  Surjectivity follows from the argument
  from last time. For injectivity, assume
  $g / f^m = 0$ as functions
  $D(f) \to k$ with $g \in A(X)$.
  Then $fg = 0$ in $A(X)$, so
$g / f^m \sim 0 / 1$ in $A(X)_f$.
\end{proof}

\section{Germs of Functions}

\begin{definition}
  Let $p \in X$ be a point on an
  affine variety.
  \begin{enumerate}
    \item A \emph{germ} of a regular
      function of $X$ at $p$ is a pair
      $(U, f)$ such that $x \in U \subseteq X$
      is open and $f$ is a regular
      function $U \to k$, up to the
      equivalence relation
      $(U, \varphi) \sim (V, \psi)$
      if there exists an open
      set $x \in W \subseteq U \cap V$
      such that $\varphi|_W = \psi|_W$.
    \item Define
      $\OO_{X, p} = \{\text{germs of regular functions of $X$ at $p$}\}$.
  \end{enumerate}
\end{definition}

\begin{exercise}
  Check that $\OO_{X, p}$ is a ring
  with operations
  \begin{align*}
    (U, \varphi) \cdot (V, \psi)
    &= (U \cap V, \varphi|_{U \cap V} \cdot \psi|_{U \cap V}), \\
    (U, \varphi) + (V, \psi)
    &= (U \cap V, \varphi|_{U \cap V} + \psi|_{U \cap V}),
  \end{align*}
  with the zero function as the
  zero element and the constant $1$
  function as the unit element.
\end{exercise}

\begin{lemma}
  $\OO_{X, p}$ is a local
  ring with unique maximal ideal
  $\m_p = \{(U, \varphi) \in \OO_{X, p} : \varphi(p) = 0\}$.
\end{lemma}

\begin{proof}
  It suffices to show that
  the units of $\OO_{X, p}$ are precisely
  $\OO_{X, p} \setminus \m_p$. To
  see the reverse inclusion, fix
  $(U, \varphi) \in \OO_{X, p}$ with
  $\varphi(p) \ne 0$. So there
  exists an open neighborhood
  $p \in W \subseteq U$ such that
  $\varphi|_W$ never vanishes. Then
  \[
    (U, \varphi) \cdot (W, 1 / \varphi|_W)
    = (W, \varphi|_W) \cdot (W, 1 / \varphi|_W))
    = (W, 1),
  \]
  so $(U, \varphi)$ is a unit
  in $\OO_{X, p}$. The forward inclusion
  is similar.
\end{proof}

\begin{prop}
  With the above setup, there is an
  isomorphism
  \begin{align*}
    A(X)_{I(p)}
    &\longrightarrow \OO_{X, p} \\
    \frac{f}{g}
    &\longmapsto
    \left(D(g), x \mapsto \frac{f(x)}{g(x)}\right)
  \end{align*}
  with $I(p) = \{f \in A(X) : f(p) = 0\}$.
\end{prop}

\begin{proof}
  To see that this is well-defined,
  let $f / g \sim f' / g' \in A(X)_{I(p)}$.
  Then $h (f g' - f' g) = 0$ for some
  $h \in A(X)$ with $h(p) \ne 0$. So
  $f / g = f' / g'$ as functions
  $D(h) \cap D(g) \to k$, which means
  that
  $f / g = f' / g'$ as elements in
  $\OO_{X, p}$.
  Thus the map is well-defined.

  Injectivity is similar to before.
  For surjectivity, choose
  $(U, \varphi) \in \OO_{X, p}$.
  Since $\varphi : U \to k$ is a
  regular function, there exists
  an open set $p \in U_p \subseteq U$
  and $f, g \in A(X)$ such that
  $g$ does not vanish on $U_p$
  and $\varphi(x) = f(x) / g(x)$
  for all $x \in U_p$. So
  $(U, \varphi) \sim (D(g), f / g)$ in
  $\OO_{X, p}$, i.e. $(U, \varphi)$ is 
  in the image.
\end{proof}

\begin{example}
  If $X = \Affine^n_k$ and $p = 0$, then
  \[
    \OO_{\Affine_k^n, 0}
    \cong k[x_1, \ldots, x_n]_{(x_1, \ldots, x_n)}
    =
    \left\{ \frac{f}{g} : f \in k[x_1, \dots, x_n], g \in k[x_1, \ldots, x_n] \setminus (x_1, \dots, x_n) \right\}.
  \]
\end{example}

\begin{remark}
  We will relate the
  local properties of $X$ at $p$
  to properties of
  $\OO_{X, p}$. We will use the following
  statements from commutative algebra:
  Let $A$ be a ring and
  $\p \subseteq A$ a prime ideal. Then
  \begin{enumerate}
    \item $A_\p$ is a local ring with
      unique maximal ideal $\p A_\p$.
    \item There is a bijection from the
      prime ideals of $A_\p$ to
      the prime ideals of $A$ contained
      in $\p$.
    \item $\height_A \p = \dim A_\p$
      (this follows from (2)).
  \end{enumerate}
  This has the following
  consequence: If $X$ is an affine
  variety and $p \in X$, then
  \[\codim_X \{p\} = \height_{A(X)} I(p) = \dim A(X)_{I(p)} = \dim \OO_{X, p}.\]
\end{remark}

\section{Sheaves}

\begin{remark}
  We will now formalize the structures
  $\OO_X(U)$ and $\OO_{X, p}$ that
  we have seen before.
\end{remark}

\begin{definition}
  A \emph{presheaf (of rings)}
  $\mathcal{F}$ on a topological space
  $X$ is the data of
  \begin{enumerate}
    \item for every open
      set $U \subseteq X$, a ring
      $\mathcal{F}(U)$;
    \item for every inclusion of
      open sets $U \subseteq V \subseteq X$,
      a ring homomorphism
      $\rho_{V, U} : \mathcal{F}(V) \to \mathcal{F}(U)$
  \end{enumerate}
  satisfying the following properties:
  \begin{enumerate}
    \item $\mathcal{F}(\varnothing) = 0$;
    \item $\rho_{U, U}$ is the identity
      map;
    \item for inclusions of
      open sets $U \subseteq V \subseteq W \subseteq X$,
      we have
      $\rho_{W, U} = \rho_{V, U} \circ \rho_{W, V}$.
  \end{enumerate}
\end{definition}

\begin{example}
  If $X$ is an affine variety, then
  $\OO_X$ gives a presheaf of rings with
  \begin{enumerate}
    \item for $U \subseteq X$,
      the ring is
      $\OO_X(U) = \{\text{regular functions } \varphi : U \to k\}$;
    \item for $U \subseteq V \subseteq X$,
      the map
      $\OO_X(V) \to \OO_X(U)$
      is given by
      $\varphi \mapsto \varphi|_U$.
  \end{enumerate}
\end{example}

\begin{remark}
  We often call $s \in \mathcal{F}(U)$
  a \emph{section}, and for
  $U \subseteq V$, we call
  $s|_U = \rho_{V, U}(s)$
  the \emph{restriction}.
\end{remark}

\begin{remark}
  A presheaf is the same thing as a functor
  $\Open_X^{\mathrm{op}} \to \Rings$,
  where $\Open_X$ is the category with
  objects the nonempty open sets
  of $X$ and morphisms corresponding
  to the inclusions $U \subseteq V$.
\end{remark}

\begin{definition}
  A presheaf $\mathcal{F}$ on $X$ is a
  \emph{sheaf} if it satisfies
  the \emph{gluing property}: For
  any $U \subseteq X$ open,
  an open cover $\{U_i\}_{i \in I}$ of $U$, and
  $\varphi_i \in \mathcal{F}(U_i)$
  with $\varphi_i|_{U_i \cap U_j} = \varphi_j|_{U_i \cap U_j}$
  for all $i, j \in I$,
  there exists a unique
  $\varphi \in \mathcal{F}(U)$
  such that $\varphi|_{U_i} = \varphi_i$
  for all $i \in I$.
\end{definition}

\begin{example}
  We have the following:
  \begin{enumerate}
    \item If $X$ is an affine variety,
      then $\OO_X$ is a sheaf (if we
      take $\varphi_i \in \OO_X(U_i)$
      that agree on the overlaps, then we
      get $\varphi : U \to k$, which
      is regular since regularity is
      a local property).
    \item If $M$ is a smooth manifold, then
      we can define a sheaf
      (on open subsets $U \subseteq M$) by
      \[
        U \longmapsto \mathcal{F}^{\mathrm{sm}}(U)
        = \{\text{smooth functions } U \to \R\}.
      \]
      We may also consider
      $\mathcal{F}^{\mathrm{cont}}$,
      $\mathcal{F}^{\mathrm{diff}}$,
      $\mathcal{F}^{\mathrm{loc}, \mathrm{const}}$, etc.
      However, $\mathcal{F}^{\mathrm{const}}$
      is a presheaf, but not a sheaf in
      general: We can take
      $U = U_1 \cup U_2$ with
      $U_1 \cap U_2 = \varnothing$, and
      we will only get
      a locally constant function.
      Similarly, $\mathcal{F}^{\mathrm{bounded}}$
      is only a presheaf but not a sheaf.
    \item If $\mathcal{F}$ is a sheaf
      on a topological space $X$ and
      $U \subseteq X$ is open, then we get
      a sheaf $\mathcal{F}|_U$ on $U$
      defined by
      $\mathcal{F}|_U(V) = \mathcal{F}(V)$
      for $V \subseteq U$ open.
  \end{enumerate}
\end{example}

\begin{definition}
  The \emph{stalk} of a sheaf
  $\mathcal{F}$ on a topological space
  $X$ at $x \in X$ is
  \[
    \mathcal{F}_x
    = \{(U, \varphi) : U \subseteq X \text{ open and } \varphi \in \mathcal{F}(U)\} / {\sim},
  \]
  where $(U, \varphi) \sim (V, \psi)$
  if there exists an open set
  $x \in W \subseteq U \cap V$
  such that $\varphi|_W = \psi|_W$.
\end{definition}

\begin{example}
  If $X$ is an affine variety and
  $p \in X$, then
  $\OO_{X, p} \cong (\OO_X)_p$.
\end{example}

\begin{remark}
  As before with $\OO_{X, p}$, one
  can check that
  $\mathcal{F}_x$ naturally
  has the structure of a ring.
\end{remark}

\begin{remark}
  An alternative perspective
  is to define the stalk as a direct limit:
  \[
    \mathcal{F}_x
    = \varinjlim_{U \ni x} \mathcal{F}(U),
  \]
  where the limit is taken over all
  open $x \in U \subseteq X$
  with respect to the ordering
  $U \le V$ if $V \subseteq U$.
\end{remark}
