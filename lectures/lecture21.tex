\chapter{Nov.~11 --- Tangent Spaces}

\section{Tangent Cones}

\begin{definition}
  Let $a \in X \subseteq \Affine^n$,
  where $X$ is an affine variety. Define
  \begin{align*}
    f : X \setminus \{a\} &\longrightarrow \PP^{n-1} \\
    x &\longmapsto [x_1 - a_1 : \cdots : x_n - a_n],
  \end{align*}
  then the \emph{blowup} of $X$
  at $a$ is $B_a X = \overline{\Gamma}_f \subseteq X \times \PP^{n - 1}$.
  Let $\pi : B_a X \to X$ be the first
  projection. Then the \emph{tangent cone}
  of $X$ at $a$ is
  $C_a X = C(\pi^{-1}(\{a\})) \subseteq \Affine^n$,
  where $\pi^{-1}(\{a\}) \subseteq \{a\} \times \PP^{n - 1} \cong \PP^{n - 1}$.
\end{definition}

\begin{example}
  When $a = 0$ and $X = \Affine^n$, we have
  $\pi^{-1}(\{0\}) = \{0\} \times \PP^{n - 1}$
  and $C_0 \Affine^n = \Affine^n$.
\end{example}

\begin{example}
  Let $X = \Affine^2$ and $a = 0$. Then
  we have:
  \begin{center}
    \begin{tabular}{c|ccc}
      $X$ & $V(y - x^2)$ & $V(y^2 - x^3)$ & $V(y^2 - x^2 - x^3)$ \\
      \hline
      $C_0 X$ & $V(y)$ & $V(y^2)$ & $V(y^2 - x^2)$
    \end{tabular}
  \end{center}
\end{example}

\begin{theorem}
  If $a \in X$ is a point on a variety,
  then $\dim C_a X = \codim_X \{a\}$.\footnote{Note that $\codim_X\{a\} = \dim X$ when
  $X$ is irreducible.}
\end{theorem}

\begin{proof}
  Since both sides are local, we may assume
  $X$ is affine. If $X = X_1 \cup \cdots \cup X_r$
  are the irreducible components, then
  we can write
  $B_a X = \bigcup_{i = 1}^r B_a X_i$ (where we take $B_a X_i = X_i$ if $a \not\in X_i$).
  Thus we may assume that $X$ is
  irreducible.
  Now let $\pi : B_a X \to X$ be the first projection, then
  \[
    \dim C_a X
    = \dim C(\pi^{-1}(\{a\}))
    = \dim \pi^{-1}(\{a\}) + 1
    = \dim X,
  \]
  where the last equality is by
  Proposition \ref{prop:blowup-codimension}.
\end{proof}

\begin{prop}\label{prop:blowup-codimension}
  Assume $X \subseteq \Affine^n$ is an
  irreducible affine variety and
  $f_1, \dots, f_r \in A(X)$ such that
  $U = X \setminus V(f_1, \dots, f_r) \ne \varnothing$.
  Then every irreducible component
  of $\pi^{-1}(V(f_1, \dots, f_r))$
  in $B_{f_1, \dots, f_r} X$ is of
  codimension $1$.
\end{prop}

\begin{proof}
  Since $X$ is irreducible, so is
  $U$. As $U \cong \Gamma_f$,
  we know $\Gamma_f$ is irreducible, so
  $B_{f_1, \dots, f_r} X$ is irreducible.
  Now observe that
  \[
    B_{f_1, \dots, f_r} X
    \subseteq \{(x, y) \in X \times \PP^{r - 1} : f_i(x) y_j - f_j(x) y_i = 0\}
  \]
  as this containment holds on $\Gamma_f$
  and the right-hand side is closed.
  Consider
  \[
    U_i = \{(x, y) \in B_{f_1, \dots, f_r}(X) : y_i \ne 0\}.
  \]
  If $(x, y) \in U_i$ and
  $f_i(x) = 0$, then $f_j(x) = 0$ for all
  $j$. So
  $\pi^{-1}(V(f_1, \dots, f_r)) \cap U_i$
  is cut out on $U_i$ by $f_i = 0$, so
  $\pi^{-1}(V(f_1, \dots, f_r))$ is
  codimension $1$ in $B_{f_1, \dots, f_r} X$
  (it is codimension $1$ on each chart).
\end{proof}

\begin{example}
  Let $X = \Affine^n$
  and $f_1, \dots, f_r = x_1, \dots, x_n$.
  Then
  \[
    \pi^{-1}(V(x_1, \dots, x_n))
    = \pi^{-1}(\{0\})
    = \{0\} \times \PP^{n - 1}.
  \]
\end{example}

\section{Tangent Spaces}

\begin{remark}
  Fix $f \in \C[x_1, \dots, x_n]$ with
  $0 \in X = V(f) \subseteq \C^n$. If
  \[
    \left[\frac{\partial f}{\partial x_1}(0), \cdots, \frac{\partial f}{\partial x_n}(0)\right]
    \ne \vec{0},
  \] 
  then $X$ is smooth at $0$ with
  tangent plane $T_0 X = V(f_1) \subseteq \C^n$, where
  $f_1$ is the linear part of $f$.
\end{remark}

\begin{definition}
  Let $X \subseteq \Affine^n$ be an affine variety
  with $0 \in X$. The
  \emph{tangent space} of $X$ at $0$ is
  \[
    T_0 X = V(f_1 : f \in I(X)) \subseteq \Affine^n,
  \]
  where $f_1$ is the homogeneous
  degree $1$ part of $f$
  (note that $T_0 X$ is a linear subspace
  of $k^n$).
\end{definition}

\begin{remark}
  Note the following:
  \begin{enumerate}
    \item We will give an intrinsic
      definition of the tangent space
      later (i.e. without embedding $X$ in $\Affine^n$).
    \item If $f, g \in I(X)$
      and $h \in k[x_1, \dots, x_n]$,
      then
      $[f + g]_1 = f_1 + g_1$ and
      $[fh]_1 = f_1 h(0) + h_1 f(0) = f_1 h(0)$,
      where $f(0) = 0$ since
      $0 \in X$.
      So if $I(X) = (f^{(1)}, \dots, f^{(r)})$,
      then $T_0 X = V(f^{(1)}_1, \dots, f^{(r)}_1)$.
    \item Recall that
      $C_0 X = V(f^{\mathrm{init}} : f \in I(X))$
      and $T_0 X = V(f_1 : f \in I(X))$.
      So $C_0 X \subseteq T_0 X$.
  \end{enumerate}
\end{remark}

\begin{example}
  $T_0 \Affine^n = V(0_1) = \Affine^n$.
\end{example}

\begin{example}
  We have the following table in $\Affine^2$:
  \begin{center}
    \begin{tabular}{c|ccc}
      $X$ & $V(y - x^2)$ & $V(y^2 - x^3)$ & $V(y^2 - x^2 - x^3)$ \\
      \hline
      $T_0 X$ & $V(y)$ & $\Affine^2$ & $\Affine^2$
    \end{tabular}
  \end{center}
\end{example}

\begin{remark}
  To define a tangent space at
  $a \in X$, we can just translate
  $a$ to $0$.
\end{remark}

\begin{prop}
  Let $X \subseteq \Affine^n$ be an
  affine variety with $0 \in X$. Write
  \[
    I(0)
    = (\overline{x}_1, \dots, \overline{x}_n)
    \le A(X) \cong k[x_1, \dots, x_n]/I(X).
  \]
  Then there is a natural vector space
  isomorphism
  \[
    I(0) / I(0)^2
    \overset{\cong}{\longrightarrow}
    \Hom_k(T_0 X, k).
  \]
  (Note that $I(0) / I(0)^2$ is a
  module over $A(X) / I(0) \cong k$,
  so it is a $k$-vector space.)
\end{prop}

\begin{proof}
  Consider the $k$-linear map
  \begin{align*}
    \varphi : I(0) &\longrightarrow \Hom_k(T_0 X, k) \\
    \overline{f} &\longmapsto f_1|_{T_0 X}
  \end{align*}
  where we view $f \in (x_1, \dots, x_n) \le k[x_1, \dots, x_n]$.
  We first check that this is well-defined.
  Assume we have $\overline{f} = \overline{g} \in I(0)$,
  so $f - g = I(X)$.
  So $[f_1 - g_1] = [f - g]_1 \in (h_1 : h \in I(X))$.
  This is the defining ideal for $T_0 X$,
  so $(f_1 - g_1)|_{T_0 X} = 0$.
  So $\varphi(\overline{f}) = \varphi(\overline{g})$, i.e. $\varphi$ is well-defined.

  Now we check that $\varphi$ is surjective.
  For this, note that the right-hand side
  is generated by the coordinate functions,
  and we have $I(0) = (\overline{x}_1, \dots, \overline{x}_n)$.

  Finally, it suffices to
  check that $\ker \varphi = I(0)^2$.
  For the reverse containment, if
  $\overline{f}, \overline{g} \in I(0)$,
  then
  \[
    (fg)_1
    = f_1 g(0) + g_1 f(0)
    = 0
  \]
  since $f(0) = g(0) = 0$. So
  $\varphi(\overline{f} \overline{g}) = 0$,
  i.e. $I(0)^2 \subseteq \varphi$.
  For the forward containment,
  if $f \in \ker \varphi$, then
  $f_1|_{T_0 X} = 0$,
  so there exists $g \in I(X)$ such that
  $f_1 = g_1$. Then
  $f - g$ has no constant or
  degree $1$ term, so
  $\overline{f} = \overline{f - g} \in I(0)^2$.
  Thus $\ker \varphi \subseteq I(0)^2$ as
  well, so we have $\ker \varphi = I(0)^2$.
\end{proof}

\begin{example}
  If $X = \Affine^n$ and $a = 0$, then
  $I(0) = (x_1, \dots, x_n) \le k[x_1, \dots, x_n]$, and
  \[
    I(0) / I(0)^2
    = k \overline{x}_1 \oplus \cdots \oplus k \overline{x}_n.
  \]
\end{example}

\begin{remark}
  We now want a representation of
  $T_0 X$, independent of its embedding
  in $\Affine^n$.
\end{remark}

\begin{prop}\label{prop:ideal-quotient-isomorphism}
  For an affine variety $X \subseteq \Affine^n$
  with $a \in X$, let
  \[
    I(a) = (f \in A(X) : f(a) = 0)
    \quad \text{and} \quad
    I_a = (f \in \OO_{X, a} : f(a) = 0).
  \]
  Then $I(a) / I(a)^2 \cong I_a / I_a^2$
  as $k$-vector spaces.
\end{prop}

\begin{definition}[Redefinition of tangent space]
  For a variety $X$ and $a \in X$,
  the \emph{tangent space} of $X$ at $a$ is
  the $k$-vector space
  \[
    T_a X
    = \Hom_k(I_a / I_a^2, k).
  \]
\end{definition}

\begin{remark}
  Note that $I_a / I_a^2$ is a module over
  $\OO_{X, a} / I_a \cong k$.
\end{remark}

\begin{lemma}\label{lem:localization-quotient}
  Given a ring $A$,
  $S \subseteq A$ a multiplicative system,
  and $A$-modules $N \subseteq M$, then
  \[
    S^{-1}(M / N) \cong
    S^{-1} M / S^{-1} N.
  \]
\end{lemma}

\begin{proof}
  Consider the short exact sequence
  \begin{center}
    \begin{tikzcd}
      0 \arrow{r} & N \arrow{r} & M \arrow{r} & M / N \arrow{r} & 0.
    \end{tikzcd}
  \end{center}
  Using that localization is an exact
  functor $\mathrm{Mod}_A \to \mathrm{Mod}_{S^{-1} A}$,
  we get a short exact sequence
  \[
    \begin{tikzcd}
      0 \arrow{r} & S^{-1} N \arrow{r} & S^{-1} M \arrow{r} & S^{-1}(M / N) \arrow{r} & 0.
    \end{tikzcd}
  \]
  So we get that
  $S^{-1}(M / N) \cong S^{-1} M / S^{-1} N$.
\end{proof}

\begin{remark}
  Previously, we had
  $\OO_{X, a} = S^{-1} A(X)$ where
  $S = A(X) \setminus I(a)$, and
  $I_a = S^{-1} I(a)$. So
  \[
    \OO_{X, a} / I_a
    = S^{-1} (A(X) / I(a))
    = S^{-1} k \cong k
  \]
  by Lemma \ref{lem:localization-quotient}
  (note that any localization of $k$ is
  still $k$).
\end{remark}

\begin{proof}[Proof of Proposition \ref{prop:ideal-quotient-isomorphism}]
  Observe that we have
  \[
    I_a / I_a^2
    = S^{-1} I(a) / S^{-1} I(a)^2
    \cong S^{-1}(I(a) / I(a)^2).
  \]
  Now we claim that the map
  \begin{align*}
    I(a) / I(a)^2 &\longrightarrow S^{-1}(I(a) / I(a)^2) \\
    \overline{f} &\longmapsto \overline{f} / 1
  \end{align*}
  is an isomorphism. The hard part is
  surjectivity. The key idea is to note
  that for $g \in S$, we can write
  $\overline{f} / g = (1 / g(a)) \overline{f} / 1$
  (this is just a computation).
\end{proof}
