\chapter{Sept.~30 --- Projective Varieties, Part 2}

\section{More on Projective Varieties}

\begin{example}
  Consider $X = V(y^2 z - x^3 - z x^2 - z^3) \subseteq \PP^2$
  and $H_z = V(z)$. Let
  \begin{align*}
    U_z = \PP^2 \setminus H_z
    &\underset{\text{bij}}{\overset{f}{\longrightarrow}} \Affine^2 \\
    {[x:y:1]} &\longmapsto (x, y).
  \end{align*}
  Then $f(X \cap U_z) = V(y^2 - x^3 - x^2 - 1)$,
  and
  \[
    X \cap U_z
    = V(y^2 z - x^3 - z x^2 - z^3, z)
    = V(x^3, z)
    = \{[0 : 1 : 0]\}.
  \]
\end{example}

\begin{example}
  Let $I = (x_0, \dots, x_n)$
  be the irrelevant ideal.
  Then $I$ is radical, but
  \[I_p(V_p(I)) = I_p(\varnothing) = (1) \ne \sqrt{I}.\]
\end{example}

\section{Cones}
\begin{definition}
  A subset $C \subseteq \Affine^{n + 1}$
  is a \emph{cone} if $0 \in C$ and
  $\lambda x \in C$ whenever
  $x \in C$ and $\lambda \in k$.
\end{definition}

\begin{example}
  If $X \subseteq \PP^n$ is a projective
  variety, then we can set $C(X) = \pi^{-1}(X) \{0\}$, where
  \begin{align*}
    \pi :
    \Affine^{n + 1} \setminus \{0\}
    &\longrightarrow \PP^n \\
    x &\longmapsto [x].
  \end{align*}
\end{example}

\begin{prop}
  If $C \subseteq \Affine^{n + 1}$
  is a cone, then $I_a(C) \le k[x_0, \dots, x_n]$
  is homogeneous.
\end{prop}

\begin{proof}
  Fix $f \in I_a(C)$. Then we can
  write $f = \sum_{i = 0}^d f_i$
  with $f_i$ homogeneous of degree $i$.
  We want to show that
  $f_i \in I_a(C)$ for each $i$.
  Fix $x \in C$. For any
  $\lambda \in k$,
  \[
    0 = f(\lambda x)
    = \sum_{i = 0}^d \lambda^i f_i(x).
  \]
  Viewing this as a polynomial in
  $\lambda$ (with $x$ fixed),
  we must have each $f_i(x) = 0$.
  Thus $f_i \in I_a(C)$.
\end{proof}

\section{Projective Nullstellensatz}

\begin{theorem}[Projective Hilbert's Nullstellensatz]
  We have the following:
  \begin{enumerate}
    \item For a projective variety
      $X \subseteq \PP^n$,
      $V_p(I_p(X)) = X$.
    \item For a homogeneous ideal
      $J \le k[x_0, \dots, x_n]$
      with $\sqrt{J} \ne (x_0, \dots, x_n)$,
      $I_p(V_p(J)) = \sqrt{J}$.
  \end{enumerate}
  As a consequence, there is a bijection
  between projective varieties and
  radical homogeneous ideals
  of $k[x_0, \dots, x_n]$
  which are not equal to
  $(x_0, \dots, x_n)$, given by
  $X \mapsto I_p(X)$ with inverse
  $J \mapsto V_p(J)$.
\end{theorem}

\begin{proof}
  (1) This is similar to the affine case.

  (2) Fix
  a homogeneous ideal
  $(1) \ne J \le k[x_0, \dots, x_n]$
  such that $\sqrt{J} \ne (x_0, \dots, x_n)$
  (the theorem is clearly true for
  the unit ideal). Then observe
  that we can write
  \begin{align*}
    I_p(V_p(J))
    &= (f \in k[x_0, \dots, x_n] \text{ homogeneous} : f(x) = 0 \text{ for all } [x] \in V_p(J)) \\
    &= (f \in k[x_0, \dots, x_n] \text{ homogeneous} : f(x) = 0 \text{ for all } x \in V_a(J) \setminus \{0\}) \\
    &= (f \in k[x_0, \dots, x_n] : f(x) = 0 \text{ for all } x \in \overline{V_a(J) \setminus \{0\}}) \\
    &=
    \begin{cases}
      I_a(V_a(J)) & \text{if } V_a(J) \supsetneq \{0\}, \quad\quad \text{(A)} \\
      I_a(\varnothing) & \text{if } V_a(J) = \{0\}, \quad\quad \text{(B)}
    \end{cases}
  \end{align*}
  In Case A, we get that
  $I_p(V_p(J)) = I_a(V_a(J)) = \sqrt{J}$
  by the affine Nullstellensatz.
  In Case B, we have
  $V_a(J) = \{0\}$, so
  $\sqrt{J} = (x_0, \dots, x_n)$,
  which we assumed was not the case.
\end{proof}

\section{The Zariski Topology on \texorpdfstring{$\PP^n$}{Pn}}

\begin{remark}
  We have the following properties
  of $I_p$ and $V_p$:
  \begin{enumerate}
    \item For homogeneous
      ideals $J_i \le k[x_0, \dots, x_n]$
      for $i \in I$, we have
      $V_p(\sum_{i \in I} J_i) = \bigcap_{i \in I} V_p(J_i)$;

      If $I = \{1, 2\}$, then we have
      $V_p(J_1 J_2) = V_p(J_1) \cup V_p(J_2)$.
    \item If $X_1, X_2 \subseteq \PP^n$
      are projective varieties, then
      \[
        I_p(X_1 \cup X_2)
        = I_p(X_1) \cap I_p(X_2)
        \quad \text{and} \quad
        I_p(X_1 \cap X_2)
        = \sqrt{I_p(X_1) + I_p(X_2)},
      \]
      where we assume in the second
      equality that
      $X_1 \cap X_2 \ne \varnothing$.
  \end{enumerate}
  The proofs are similar to the
  affine case.
\end{remark}

\begin{example}
  Let $X_1 = V(x) \subseteq \PP^2$
  and $X_2 = V(y, z) \subseteq \PP^2$.
  Then $I(X_1 \cap X_1) = I(\varnothing) = (1)$,
  but we have
  $I(X_1) + I(X_2) = (x, y, z)$, which
  is already radical.
\end{example}

\begin{definition}
  The \emph{Zariski topology} on
  $\PP^n$ is the topology whose
  closed sets are projective
  varieties $X \subseteq \PP^n$
  (equivalently, the vanishing loci
  of homogeneous ideals).
\end{definition}

\begin{remark}
  This is a topology by the above
  properties of $I_p$ and $V_p$.
  We now want to relate this to
  the topology on our charts.
  Let $H_0 = V(x_0)$ and consider
  the bijection
  \begin{align*}
    \Affine^n
    &\overset{\rho_0}{\longrightarrow} \PP^n \setminus H_0 \\
    (x_1, \dots, x_n)
    &\longmapsto [1 : x_1 : \dots : x_n].
  \end{align*}
  We want to show that $\rho_0$ is a
  homeomorphism. Write
  $\Affine^n \subseteq \PP^n$.
  Consider the ring homomorphism
  \begin{align*}
    k[x_0, \dots, x_n]
    &\overset{\Phi}{\longrightarrow} k[x_1, \dots, x_n] \\
    f(x_0, \dots, x_n)
    &\longmapsto f(1, x_1, \dots, x_n) =: f^i
  \end{align*}
  We call $f^i$ the
  \emph{dehomogenization} of $f$.
\end{remark}

\begin{example}
  Let $f(x) = x_0 x_2^2 - x_1^3 - x_0 x_1^2 - x_0^3$,
  then
  $f^i(x) = x_2^2 - x_1^3 - x_1^2 - 1$.
\end{example}

\begin{definition}
  If $J \le k[x_0, \dots, x_n]$
  is homogeneous, then define
  its \emph{dehomogenization} to be
  \[
    J^i = (f^i : f \in J)
    = \Phi(J).
  \]
\end{definition}

\begin{prop}
  For $J \le k[x_0, \dots, x_n]$
  homogeneous,
  $V_p(J) \cap \Affine^n = V_a(J^i)$.
\end{prop}

\begin{proof}
  The idea is to
  use that for
  $[1 : x_1 : \dots : x_n] \in \PP^n$
  and $f \in k[x_0, \dots, x_n]$
  homogeneous, we have
  $f([1 : x]) = 0$ if and only if
  $f^i(x) = 0$. Fill in the details
  as an exercise.
\end{proof}

\begin{definition}
  If $f \in k[x_1, \dots, x_n]$
  with $\deg f  = d$, then define
  its \emph{homogenization} to be
  \[
    f^h = x_0^d f(x_1 / x_0, \dots, x_n / x_0) \in k[x_0, x_1, \dots, x_n],
  \]
  which is homogeneous of degree $d$.
\end{definition}

\begin{example}
  Let $f = x_2^2 - x_1^3 - x_1^2 - 1$.
  Then we have
  \[
    f^h = x_0^3 ((x_2 / x_0)^2 - (x_1 / x_0)^3 - (x_1 / x_0)^2 - 1)
    = x_0 x_2^2 - x_1^3 - x_0 x_1^2 - x_0^3.
  \]
\end{example}

\begin{remark}
  While $f^h g^h = (fg)^h$, note that
  $(f + g)^h \ne f^h + g^h$ in general.
\end{remark}

\begin{definition}
  For $J \le k[x_1, \dots, x_n]$ an ideal,
  define its \emph{homogenization} to be
  \[
    J^h = (f^h : f \in J).
  \]
\end{definition}

\begin{prop}
  For $J \le k[x_1, \dots, x_n]$ an ideal,
  $V_a(J) = V_p(J^h) \cap \Affine^n$.
\end{prop}

\begin{proof}
  Left as an exercise,
  use that
  $f(a_1, \dots, a_n) = 0$ if and only
  if $f^h(1, a_1, \dots, a_n) = 0$.
\end{proof}

\begin{remark}
  The above results imply that
  $\rho_0 : \Affine^n \to \PP^n \setminus H_0$
  is a homeomorphism.
\end{remark}
