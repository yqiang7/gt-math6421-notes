\chapter{Oct.~23 --- Birational Maps}

\section{Birational Maps, Continued}

\begin{example}
  The following are examples of
  birational maps:
  \begin{enumerate}
    \item The \emph{cuspidal curve}
      $C = V(x^2 - y^3) \dashrightarrow \Affine^1$
      given by $(x, y) \mapsto x / y$ is
      a morphism on $C \setminus \{(0, 0)\}$.
    \item The projection
      $\PP^n \dashrightarrow \PP^{n - 1}$,
      $[x_0 : \dots : x_n] \mapsto [x_0 : \dots : x_{n - 1}]$
      is a morphism on
      $\PP^n \setminus \{[0 : \dots : 0 : 1]\}$.
    \item Consider a map
      \begin{align*}
        \Affine^n &\dasharrow \Affine^n \\
        x &\longmapsto
        \left(
          \frac{f_1(x)}{g_1(x)},
          \dots,
          \frac{f_n(x)}{g_n(x)}
        \right)
      \end{align*}
      for some nonzero $f_i, g_j \in k[x_1, \dots, x_n]$.
      This is a morphism on
      $\Affine^n \setminus V(g_1 \cdots g_n)$.
  \end{enumerate}
\end{example}

\begin{definition}
  We define the following:
  \begin{enumerate}
    \item A map $f : X \dashrightarrow Y$
      is \emph{dominant} if the image
      of $f$ contains a nonempty open set.
    \item If $f : X \dashrightarrow Y$
      and $g : Y \dashrightarrow Z$
      are dominant rational maps, i.e.
      \begin{center}
        \begin{tikzcd}
          X \arrow[dashed]{r}{f} & Y \arrow[dashed]{r}{g} & Z \\
          U \ar[u, hook] \ar[ur, "f'", swap] & V \ar[u, hook] \ar[ur, "g'", swap]
        \end{tikzcd}
      \end{center}
      then we can define the composition
      $g \circ f : X \dashrightarrow Z$
      by
      \[
        (f')^{-1}(V) \overset{f'}{\longrightarrow} V
        \overset{g'}{\longrightarrow} Z.
      \]
  \end{enumerate}
\end{definition}

\begin{definition}
  A rational map $f : X \dashrightarrow Y$
  is \emph{birational} if it is dominant
  and there exists a dominant rational map
  $g : Y \dashrightarrow X$ such that
  $g \circ f = \id_X$ and $f \circ g = \id_Y$
  (as rational maps). In this case,
  we say that $X$ and $Y$ are
  are \emph{birational}.
\end{definition}

\begin{example}
  We have the following:
  \begin{enumerate}
    \item Let $X$ be an irreducible variety
      and $U \subseteq X$ a nonempty open
      set. Then the inclusion morphism
      $f : U \hookrightarrow X$ is
      birational with
      $g : X \dashrightarrow U$
      the identity map on $U$.
    \item Define a morphism
      \begin{align*}
        \Affine^n &\overset{f}{\longrightarrow} \Affine^n \\
        x &\longmapsto
        (x_1, x_1 x_2, \dots, x_1 x_n).
      \end{align*}
      Then we can define an inverse
      rational map
      \begin{align*}
        \Affine^n &\overset{g}{\dashrightarrow} \Affine^n \\
        y &\longmapsto
        (x_1, x_2 / x_1, \dots, x_n / x_1).
      \end{align*}
      So $f$ is a birational map.
  \end{enumerate}
\end{example}

\begin{prop}
  Two irreducible varieties are birational
  if and only if they contain isomorphic
  nonempty open sets.
\end{prop}

\begin{proof}
  $(\Leftarrow)$ Assume we have open
  sets $U \subseteq X$, $V \subseteq Y$
  with an isomorphism $f : U \overset{\cong}{\longrightarrow} V$.
  Then we get rational maps
  $f : X \dashrightarrow Y$ and
  $f^{-1} : Y \dashrightarrow X$ which
  compose to the identity.

  $(\Rightarrow)$ Assume we have
  \begin{center}
  \begin{tikzcd}
    X \arrow[dashed]{r}{f} & Y & Y \arrow[dashed]{r}{f} & X \\
    U \ar[u, hook] \ar[ur, "f'", swap] & & V \ar[u, hook] \ar[ur, "g'", swap]
  \end{tikzcd}
  \end{center}
  satisfying $g \circ f = \id_X$
  and $f \circ g = \id_Y$.
  Then the composition
  \[
    (g')^{-1}(U) \overset{g'}{\longrightarrow} U \overset{f'}{\longrightarrow} Y
  \]
  is the inclusion map by the identity
  principle, so $g'((g')^{-1}(U)) = (f')^{-1}(g^{-1}(U)) \subseteq (f')^{-1}(V)$.
  Similarly, we can also get the inclusion
  $f'((f')^{-1}(V)) \subseteq (g')^{-1}(U)$.
  So we get morphisms
  \begin{center}
    \begin{tikzcd}
      (f')^{-1}(U) \arrow[bend left]{r}{f'} & (g')^{-1}(V) \arrow[bend left]{l}{g'}
    \end{tikzcd}
  \end{center}
  that compose to the identity on
  some open sets, so they compose to
  $\id$ by the identity principle.
\end{proof}

\begin{example}
  Consider the following:
  \begin{enumerate}
    \item  $\PP^{n + m}$
      and $\PP^n \times \PP^m$ both
      contain $\Affine^{n + m}$,
      so they are birational.
    \item $G(d, n)$ and $\PP^{d(n - d)}$
      both contain $\Affine^{d(n - d)}$,
      so they are birational.
  \end{enumerate}
\end{example}

\section{Rational Functions}

\begin{definition}
  A \emph{rational function} on a variety
  $X$ is a rational map
  $X \dashrightarrow \Affine^1$.
  This is equivalent to the data
  $(U, \varphi)$ for $U \subseteq X$ a
  nonempty open set and
  $\varphi \in \OO_X(U)$, up to the
  equivalence $(U, \varphi) \sim (U', \varphi')$
  if and only if $\varphi|_W = \varphi'|_W$
  for some nonempty open $W \subseteq U \cap U'$.
\end{definition}

\pagebreak
\begin{example}
  If $f, g \in k[x_1, \dots, x_n]$
  with $g \ne 0$, then $f / g$ gives a
  rational function on $\Affine^n$. The
  two ways of thinking about this are:
  \begin{enumerate}
    \item The rational map
      $\varphi : \Affine^n \dashrightarrow \Affine^1$
      given by $x \mapsto f(x) / g(x)$.
    \item The equivalence class
      of $(D(g), f(x) / g(x))$.
  \end{enumerate}
\end{example}

\begin{definition}
  Let $X$ be an irreducible variety $X$.
  Then the \emph{function field} of $X$ is
  \[
    K(X) = \{\text{rational functions on } X\}
  \]
\end{definition}

\begin{remark}
  Note that $K(X)$ is in fact a field:
  \begin{itemize}
    \item Multiplication is
      defined by
      $(U, \varphi) \cdot (V, \psi) = (U \cap V, \varphi|_{U \cap V} \psi|_{U \cap V})$
      and similarly for addition.
    \item One can check that
      the above operations turn
      $K(X)$ into a ring.
    \item For $0 \ne (U, \varphi) \in K(X)$,
      we have
      $(U, \varphi)
        \cdot (U \setminus \{\varphi = 0\}, 1 / \varphi)
        = (X, 1)$.
  \end{itemize}
\end{remark}

\begin{remark}
  We have the following properties of
  $K(X)$:
  \begin{enumerate}
    \item For a nonempty open set $U \subseteq X$, the
      restriction map
      $K(X) \to K(U)$
      is an isomorphism.
    \item For an irreducible affine variety
      $X$, the map
      \begin{align*}
        \Frac(A(X)) &\overset{\alpha}{\longrightarrow} K(X) \\
        \varphi / \psi
        &\longmapsto (D(\psi), \varphi / \psi)
      \end{align*}
      is an isomorphism. One can check
      that it is a well-defined
      homomorphism, and it is injective
      since it is a homomorphism out of a
      field. To see surjectivity, let
      $(U, \varphi) \in K(X)$. By shrinking
      $U$, we may assume that $U = D(g)$
      for some $0 \ne g \in A(X)$.
      Since
      \[
        \OO_X(D(g)) = A(X)_g,
      \]
      we can write $\varphi = f / g^n$
      for some $n \ge 0$ and $f \in A(X)$.
      So $\alpha(f / g^n) = (U, \varphi)$.
  \end{enumerate}
\end{remark}

\begin{example}
  We have
  $K(\Affine^n) = \Frac(A(\Affine^n)) = \Frac(k[x_1, \dots, x_n]) = k(x_1, \dots, x_n)$.
\end{example}

\begin{remark}
  If $f : X \dashrightarrow Y$ is a
  dominant morphism of irreducible
  varieties, then we get a $k$-algebra
  homomorphism on their function fields by
  the pullback:
  \begin{align*}
    f^* : K(Y) &\longrightarrow K(X) \\
    (V, \varphi) &\longmapsto
    (f^{-1}(V), (f')^*(\varphi)).
  \end{align*}
\end{remark}

\begin{exercise}
  Let $X$ and $Y$ be irreducible varieties.
  Then
  \begin{enumerate}
    \item If $f : X \dashrightarrow Y$ is
      birational, then $f^* : K(Y) \to K(X)$
      is an isomorphism.
    \item If $K(X) \cong K(Y)$ as
      $k$-algebras, then $X$ is birational
      to $Y$.
  \end{enumerate}
\end{exercise}

\begin{example}
  Since $G(d, n) \sim_{\mathrm{bir}} \PP^{d(n - d)}$,
  we have $K(G(d, n)) \cong K(\PP^{d(n - d)})$.
\end{example}

\begin{corollary}
  Any irreducible variety $X$ is birational
  to a hypersurface in $\PP^n$.
\end{corollary}

\begin{proof}
  By Noether normalization, there exists a
  transcendence basis 
  $x_1, \dots, x_n$ for $K(X)$ over $k$
  (this is finite as we may replace $X$ by
  an affine open set).
  Assuming $k$ is characteristic $0$ (or
  working harder in nonzero characteristic),
  $k(x, \dots, x_n) \subseteq K(X)$
  is separable. By the definition of a
  transcendence basis, this extension is
  finite. So by the primitive element
  theorem, there exists $y \in K(X)$ such
  that
  \[K(X) = k(x_1, \dots, x_n, y).\] Let
  $f \in k(x_1, \dots, x_n)[t]$ denote
  the minimal polynomial of $y$.
  Clearing denominators, we may assume
  $f \in k[x_1, \dots, x_n][t]$.
  By Gauss's lemma, $f$ is still
  irreducible in $k[x_1, \dots, x_n, t]$.
  Now set
  \[
    Y = V(f(x_1, \dots, x_n, x_{n + 1}))
    \subseteq \Affine^{n + 1},
  \]
  which is irreducible. Now we have
  \[
    K(Y) =
    \Frac
    \frac{(k[x_1, \dots, x_{n + 1}])}
    {(f(x_1, \dots, x_{n + 1}))}
    \cong \frac{k(x_1, \dots, x_n)[y]}{(f)}
    \cong K(X),
  \]
  so $X \sim_{\mathrm{bir}} Y$.
  We get the desired statement after
  taking projective closures.
\end{proof}
