\chapter{Dec.~2 --- B\'ezout's Theorem}

TODO: Fill in notes from beginning of class.

\section{B\'ezout's Theorem}

\begin{prop}
  If $X \subseteq \PP^n$ is a projective
  variety and $f \in k[x_0, \dots, x_n]$ 
  is a homogeneous polynomial not
  vanishing on any irreducible component
  of $X$, then
  \[
    \deg(I(X) + (f))
    = (\deg X)(\deg f).
  \]
\end{prop}

\begin{proof}
  Set $e = \deg f$. We first show the
  $(*)$ condition from last lecture holds:
  If $g \in k[x_0, \dots, x_n]$ is
  homogeneous with $fg \in I(X)$, then
  we want to show that  $g \in I(X)$.
  As $f$ does not vanish on any irreducible
  component of $X$, $g$ must vanish
  on every irreducible component.
  So $g \in I(X)$ as desired.
  
  Now we analyze the Hilbert functions.
  Write
  \[
    h_X(d) = \frac{\deg X}{m!} d^m + ad^{m - 1} + \text{lower order terms}
  \]
  for some $a \in \Q$. As
  $(*)$ holds, for $d \gg 0$ we have
  \begin{align*}
    h_{I(X) + (f)}(d)
    &= h_X(d) - h_X(d - e) \\
    &= \left(\frac{\deg X}{m!} d^m + a d^{m - 1}\right)
    - \left(\frac{\deg X}{m!} (d - e)^m + a (d - e)^{m - 1}\right)
    + \text{lower order terms} \\
    &= \frac{e \deg X}{(m - 1)!} d^{m - 1}
    + \text{lower order terms}.
  \end{align*}
  Thus we see that
  $\deg(I(X) + (f)) = e(\deg X) = (\deg f)(\deg X)$.
\end{proof}

\begin{example}
  If $X = \PP^n$ and
  $f \in k[x_0, \dots, x_n]_e$ is
  irreducible, then
  \[
    \deg V_p(f)
    = \deg((f))
    = \deg(I(X) + (f))
    = (\deg X)(\deg f)
    = 1 \cdot \deg f
    = \deg f,
  \]
  which reproves that the degree of a
  degree $d$ hypersurface is $d$.
\end{example}

\begin{theorem}[B\'ezout's theorem]
  For any two plane curves $X, Y \in \PP^2$
  (i.e. a closed projective variety
  in $\PP^2$ of pure dimension $1$) without
  common irreducible components,
  \[
    \#(X \cap Y)
    \le (\deg X)(\deg Y).
  \]
\end{theorem}

\begin{proof}
  Let $e = \deg Y$. As $Y \subseteq \PP^2$ 
  is a hypersurface, there exists
  $f \in k[x_0, \dots, x_n]_e$ with
  $I(Y) = (f)$. So
  \[
    (\deg X)(\deg Y)
    = (\deg X)(\deg f)
    = \deg(I(X) + (f))
    = \deg(I(X) + I(Y)).
  \]
  Note that $I(X \cap Y) = \sqrt{I(X) + I(Y)} \supseteq I(X) + I(Y)$, so we get
  \[
    (\deg X)(\deg Y)
    = \deg(I(X) + I(Y))
    \ge \deg(I(X \cap Y)).
  \]
  Since $I(X \cap Y)$ is zero-dimensional,
  we have $\deg(I(X \cap Y)) = \deg(X \cap Y) = \#(X \cap Y)$.
\end{proof}

\begin{remark}
  We can make the following
  extensions to B\'ezout's theorem:
  \begin{enumerate}
    \item Multiplicities:
      We can define $\mult_a(X, Y) \in \Z_{> 0}$
      for $a \in X \cap Y$. Then
      \[
        \sum_{a \in X \cap Y} \mult_a(X, Y)
        = (\deg X)(\deg Y).
      \]
    \item For a general $Y \subseteq \PP^2$
      (relative to $X$), we have
      equality
      $(\deg X)(\deg Y) = \#(X \cap Y)$.
    \item One can also get a similar
      statement in higher dimensions.
  \end{enumerate}
\end{remark}

\section{Divisors on Curves}

\begin{remark}
  Let $C$ be a smooth projective curve,
  e.g. $C \cong \PP^1$ or $C \subseteq \PP^2$
  a general degree $d$ hypersurface.
\end{remark}

\begin{definition}
  A \emph{divisor} on $C$ is a formal
  $\Z$-linear combination of distinct points,
  i.e.
  \[
    D = a_1 p_1 + \dots + a_r p_r,
    \quad a_i \in \Z,\, p_i \in C.
  \]
  Let $\Div C = \bigoplus_{p \in C} \Z p = \{D : \text{$D$ is a divisor on $C$}\}$,
  which is a free abelian group.
  For
  $D = \sum a_i p_i \in \Div C$,
  its \emph{degree} is
  $\deg D = \sum a_i$ (think of this as the
  number of points, counting multiplicity).
\end{definition}

\begin{example}
  Given $f \in K(C)$, we want to
  get a divisor
  $\divv f \in \Div(C)$, called the
  \emph{divisor of zeros and poles}.
  For example, if
  \[
    f = \frac{x_0 x_1}{(x_0 - x_1)^2}
    \in K(\PP^1),
  \]
  then we want $\divv f = [1 : 0] + [0 : 1] - 2[1 : 1]$.

  To define this precisely, for
  $p \in C$ recall
  $\OO_{C, p}$ is a regular local
  Noetherian ring of dimension $1$. Thus
  commutative algebra implies that
  $\OO_{C, p}$ is a discrete valuation
  ring (DVR).
\end{example}

\begin{definition}
  A \emph{discrete valuation ring (DVR)}
  is a UFD with unique irreducible
  element (up to multiplication by units).
\end{definition}

\begin{remark}
  Assume $R$ is a DVR with $\pi \in R$
  irreducible (such an element
  $\pi$ is called a
  \emph{uniformizer}). Then for $0 \ne r \in R$,
  we can write
  \[
    r = u \pi^m, \quad
    u \in R^\times,\, m \ge \N.
  \]
\end{remark}

\begin{example}
  Consider $0 \in \Affine^1$, so
  $\OO_{\Affine^1, 0} = k[x]_{(x)}$.
  For $f \in \OO_{\Affine^1, 0}$, we can
  write
  \[
    f = x^m \frac{q(x)}{r(x)}
  \]
  with $m \in \N$ and
  $q, r \in k[x]$ such that
  $q(0), r(0) \ne 0$.
\end{example}

\begin{definition}
  For $p \in C$, we get a function
  (in fact a \emph{valuation})
  \[
    \ord_p : K(C)^\times \longrightarrow \Z
  \]
  where $\ord_p(f)$ is the unique
  integer such that $f = u \pi^{\ord_p(f)}$
  with $u \in \OO_{C, p}^\times$ and
  $\pi \in \OO_{C, p}$ a uniformizer.
\end{definition}

\begin{definition}
  For $0 \ne f \in K(C)$, the
  \emph{divisor of zeros and poles} of $f$
  is
  \[
    \divv(f) := \sum_{p \in C} \ord_p(f) p.
  \]
  Divisors of this form are called
  \emph{principal}.
\end{definition}

\begin{exercise}
  Show that $\divv f$ is a finite sum, so
  $\divv f$ is well-defined.
\end{exercise}

\begin{example}
  For $C = \PP^1$ and
  $f = x_0 x_1 / (x_0 - x_1)^2$ as before,
  $\divv f = [1 : 0] + [0 : 1] - 2[1 : 1]$.
\end{example}

\begin{remark}
  One can check that
  $\divv(fg) = \divv(f) + \divv(g)$, i.e.
  $\divv$
  is a group homomorphism. So
  \[
    \PDiv(C)
    = \{\divv f : f \in K(C)^\times\}
  \]
  is a subgroup of $\Div C$.
\end{remark}

\begin{definition}
  We say $D, D' \in \Div C$ are \emph{linearly equivalent}
  if $D = D' + \divv f$
  for some $f \in K(C)^\times$, and
  write $D \sim D'$. The \emph{Picard group}
  of $C$ is
  \[
    \Pic := (\Div C) / (\PDiv C)
    = \text{divisors on $C$ up to linear equivalence}.
  \]
\end{definition}

\begin{example}
  We have $\Pic \PP^1 \cong \Z$.

  We give a sketch of the proof. There
  is a surjective group homomorphism
  \begin{align*}
    \varphi : \Div \PP^1 &\longrightarrow \Z \\
    D &\longmapsto \deg D.
  \end{align*}
  Then we want to show that
  $\ker \varphi = \PDiv(\PP^1)$.
  First fix $D \in \ker \varphi$, so
  $D = \sum_{i = 1}^r a_i p_i$ with $\sum_{i = 1}^r a_i = 0$.
  Write $p_i = [s_i : t_i]$,
  and consider $f = \prod_{i = 1}^r (x_0 t_i - x_1 s_i)^{a_i}$.
  Since $\sum_{i = 1}^r a_i = 0$,
  we have $f \in K(\PP^1)$ and
  $\divv f = D$, so
  $D \in \PDiv(\PP^1)$.
  The reverse inclusion is similar.

  Note that we always have
  $\PDiv(\PP^1) \subseteq \ker \varphi$,
  but the other inclusion may
  fail for other curves.
\end{example}

\begin{remark}
  Using B\'ezout's theorem (the one which
  counts multiplicities),
  one can show that
  \[
    \deg(\divv f) = 0
  \]
  for any $f \in K(C)^\times$.
\end{remark}

\begin{example}
  Let $C \subseteq \PP^2$ be a smooth
  cubic plane curve (i.e. an \emph{elliptic curve}), and fix a point
  $p_0 \in C$.
  Up to a change of coordinates, we may
  assume that $p_0 = [1 : 0 : 0]$. Then
  \begin{enumerate}
    \item $\Pic C$ is \emph{not}
      finitely generated.
    \item The map
      $C \to \Pic^0(C)$ by
      $p \mapsto p - p_0$ is a bijection,
      where $\Pic^0(C) = \{D \in \Pic C : \deg D = 0\}$.
      As $\Pic^0(C)$ is a group, this
      bijection
      gives a group structure on $C$
      with identity $p_0$.
  \end{enumerate}
  One can also describe this more
  geometrically. If $p = [1 : 0 : 0]$, then
  up to a change of coordinates, one can
  assume $C$ is given by
  \[
    y^2 = x^3 + ax + b
  \]
  on a chart $\Affine_{x, y}^2 \subseteq \PP^2_{x : y : z}$.
  The usual group law on $C$ is given as
  follows: For $p, q \in C$, let
  $r$ be the unique third intersection
  point of the line through $p, q$ with
  $C$. Then we define $p + q$ to be the
  reflection $s$ of $r$ across the $x$-axis.

  To see this agrees with the
  group structure defined from
  $\Pic^0(C)$, we need to check
  \[(p - p_0) + (q - p_0) \sim s - p_0\]
  Let the line through $p, q$ be
  $\ell = V(L(x, y, z))$ and the line
  through $r, s$ be
  $\ell' = V(L'(x, y, z))$.
  Then
  \[
    0 \sim \divv \frac{L(x, y, z)}{z}
    = p + q + r -3p_0
    \quad \text{and} \quad
    0 \sim \divv \frac{L'(x, y, z)}{z}
    = r + s + p_0 - 3p_0,
  \]
  which shows that the two definitions
  for the group law agree.
\end{example}
