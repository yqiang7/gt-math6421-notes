\chapter{Oct.~16 --- Completeness and Embeddings}

\section{More on Completeness}

\begin{example}
  Recall from before that we have:
  \begin{enumerate}
    \item $\Affine^n$ is not complete for
      $n \ge 1$.
    \item $\PP^n$ is complete.
    \item Any projective variety is
      complete.
  \end{enumerate}
\end{example}

\begin{remark}
  Note the following:
  \begin{enumerate}
    \item If $k = \C$, then a variety is
      complete if and only if $X^{\mathrm{an}}$ is compact
      in the analytic topology.
    \item \emph{Nagata's compactification theorem}:
      Any variety $X$ admits an open
      embedding $X \hookrightarrow \overline{X}$
      with $\overline{X}$ complete.
    \item In dimension $1$, completeness
      is equivalent to being projective.
      In dimension $\ge 2$, being projective
      implies completeness, but the converse
      may fail.
  \end{enumerate}
\end{remark}

\begin{prop}
  If $f : X \to Y$ is a morphism of varieties
  with $X$ complete, then
  \begin{enumerate}
    \item $f(X)$ is closed in $Y$.
    \item $f(X)$ is complete.
  \end{enumerate}
\end{prop}

\begin{proof}
  (1) Consider the projection
  $\pr_2 : X \times Y \to Y$. As
  $Y$ is separated, the graph
  $\Gamma_f$ of $f$ is closed in $X \times Y$.
  Now $f(X) = \pr_2(\Gamma_f)$, which is
  closed in $Y$ as $X$ is complete.

  (2) Fix any variety $Z$. Now consider
  the projection
  $\pi' : f(X) \times Z \to Z$ and
  $W \subseteq f(X) \times Z$ closed. Now
  consider the projection $\pi : X \times Z \to Z$.
  Then $\pi'(W) = \pi((f, \id)^{-1}(W))$.
  The set $(f, \id)^{-1}(W)$ is closed in
  $X \times Z$ by continuity, and
  $\pi'(W)$ is closed in $Z$ as $X$ is complete.
  So $f(X)$ is complete.
\end{proof}

\begin{corollary}
  If $X$ is a complete variety that is
  connected, then any $\varphi \in \OO_X(X)$
  is constant. In particular,
  $\OO_X(X) \cong k$.
\end{corollary}

\begin{proof}
  Any $\varphi \in \OO_X(X)$ induces a
  morphism $f : X \to \PP^1_{s : t}$ by
  $x \mapsto [1 : \varphi(x)]$.
  By construction, we have
  $f(X) \subseteq \{s \ne 0\} \cong \Affine^1$.
  As $X$ is complete, $f(X)$ is closed,
  and as $X$ is connected, $f(X)$ is
  connected. Since the only proper closed
  subsets of $\PP^1$ are points, $f(X)$
  be must a single point $[1 : a]$ since
  $f(X)$ is connected.
  So $\varphi(x) = a$ for every
  $x \in X$.
\end{proof}

\begin{corollary}
  The only complete affine varieties are
  finite point sets.
\end{corollary}

\begin{proof}
  Assume $X = V(I) \subseteq \Affine^n$ is
  complete. Using the decomposition
  of $X$ into connected components (there
  finitely many
  since $X$ is Noetherian), we may reduce
  to the case that $X$ is connected.
  Then $A(X) = \OO_X(X) \cong k$, but this happens
  if and only if $X$ is a single point
  (see Homework 1).
\end{proof}

\section{The Veronese Embedding}

\begin{remark}
  If $X \subseteq \PP^n$ is a projective
  variety, which open sets in $X$ are
  affine?
  \begin{itemize}
    \item $X_i = X \cap (\PP^n \setminus
      V(x_i))$ is affine.
    \item $X \setminus (\PP^n \setminus H)$
      with $H \subseteq \PP^n$ a
      hyperplane is affine.
  \end{itemize}
  We will see that
  $X \cap (\PP^n \setminus V(g))$ is affine
  with $g \in k[x_0, \ldots, x_n]$
  homogeneous of degree $d > 0$.
\end{remark}

\begin{definition}[Veronese embedding, $d$-tuple embedding]
  Fix $n , d > 0$. Let
  $f_0, \dots, f_N \in k[x_0, \ldots, x_n]$
  denote the monomials of degree $d$, where
  $N = \binom{n + d}{d} - 1$. The
  \emph{Veronese embedding} $\nu_{n, d}$
  is the map
  \begin{align*}
    \nu_{n, d} : \PP^n
    &\longrightarrow \PP^N \\
    x &\longmapsto [f_0(x) : \cdots : f_N(x)].
  \end{align*}
\end{definition}

\begin{example}\label{ex:veronese}
  Let $n = 1$, $d = 3$. Then the degree-$3$
  Veronese embedding is given by
  \begin{align*}
    f : \PP^1 &\longmapsto \PP^3_{s : t : u : v} \\
    [x : y] &\longmapsto [x^3 : x^2y : xy^2 : y^3].
  \end{align*}
  Then $X = f(\PP^1) = V(sv - tu, t^2 - su, u^2 - vt)$.
  We can define an inverse
  \begin{align*}
    X &\longrightarrow \PP^1 \\
    [s : t : u : v] &\longmapsto
    \begin{cases}
      [1 : t / s] & \text{if } s \ne 0, \\
      [u / v : 1] & \text{if } v \ne 0.
    \end{cases}
  \end{align*}
  Note that $ut = sv$ on $X$ when
  $sv \ne 0$, so this is well-defined.
\end{example}

\begin{prop}
  $\nu_{n, d} : \PP^n \to \PP^N$ is a
  closed embedding.
\end{prop}

\begin{proof}
  As $\PP^n$ is complete,
  $X = \nu_{n, d}(\PP^n)$ is closed in
  $\PP^N$. A similar computation as
  in Example \ref{ex:veronese} shows that
  $\nu_{n, d} : \PP^n \to X$ has
  an inverse. So $\nu_{n, d}$ is a
  closed embedding.
\end{proof}

\begin{remark}
  Note the following:
  \begin{enumerate}
    \item With some work, one can show that
      $\nu_{n, d}(\PP^n)$ can be cut out
      by quadratic equations
      \[
        \{
          z_i z_j - z_k z_\ell :
          f_i f_j = f_k f_\ell
          \text{ as monomials}
        \}.
      \]
    \item If $g \in k[x_0, \ldots, x_n]$
      is homogeneous of degree $d > 0$, then
      \[
        \PP^n \supseteq
        V(g)
        = \nu_{n, d}^{-1}(H)
      \]
      for some hyperplane $H \subseteq \PP^N$.
  \end{enumerate}
\end{remark}

\begin{theorem}
  If $X \subseteq \PP^n$ is a projective
  variety, then for any
  $g \in k[x_0, \ldots, x_n]$ homogeneous of
  degree $d$, the variety
  $X \setminus V(g)$ is affine.
\end{theorem}

\begin{proof}
  Consider the Veronese
  embedding $\nu_{n, d} : \PP^n \to \PP^N$.
  Then $V(g) = \nu_{n, d}^{-1}(H)$ for some
  hyperplane $H \subseteq \PP^N$. So
  $X \setminus V(g) \cong \nu_{n, d}(X) \setminus H$,
  which is affine.
\end{proof}

\section{The Grassmannian}

\begin{definition}
  For $0 \le d \le n$, define the
  \emph{Grassmannian}
  \[
    G(d, n) :=
    \{
      \text{$d$-dimensional subspaces of }
      k^n
    \}.
  \]
\end{definition}

\begin{example}
  We have the following:
  \begin{itemize}
    \item $G(0, n)$ and $G(n, n)$ are
      just points.
    \item $G(1, n) = \PP^{n - 1}$
      set-theoretically.
    \item $G(d, n) \cong G(n - d, n)$
      as dimension $d$ subspaces of
      $V = k^n$ are in bijection with
      dimension $n - d$ subspaces of
      $V^*$, where $W \subseteq V$
      corresponds to
      $\ker(V^* \to W^*)$.
    \item $G(2, n) = \{\text{lines in } \PP^{n - 1}\}$.
  \end{itemize}
\end{example}

\begin{theorem}
  The Grassmannian
  $G(d, n)$ can be endowed with the
  structure of a (projective) variety of
  dimension $d(n - d)$.
\end{theorem}

\begin{proof}
  One strategy is to let
  $V = k^n = \Span\{e_1, \dots, e_n\}$.
  We sketch this idea.
  Observe that
  \begin{enumerate}
    \item A $d$-dimensional subspace
      $W \subseteq V$ can be represented by
      a $d \times n$ matrix $A$ of
      rank $d$ (choose a basis of $W$ and
      write the coordinates with respect
      to the basis $\{e_1, \dots, e_n\}$
      of $V$).
      Note that $A$ is unique up to
      the action by
      $\GL_d(k)$, so we get a point
      $[A] \in G(d, n)$.
    \item For $I = \{1, \dots, n\}$ with
      $|I| = d$, define the set
      \[
        U_I = \{
          [A] \in G(d, n)
          : \det(A_I) \ne 0
        \},
      \]
      where $A_I$ denotes the $I$-th
      $d \times d$ minor of $A$.
      One can check that the condition
      $\det(A_I) \ne 0$
      is well-defined
      (i.e. $\det((BA)_I) = \det(B) \det(A_I)$ for $B \in \GL_d(k)$).

    Then we have a bijection
    $\Affine^{d(n - d)} \to U_I$ given as
    follows. When $I = \{1, \dots, d\}$,
    define
    \begin{align*}
      \Affine^{d(n - d)} &\longrightarrow U_I \\
      C &\longmapsto
      [ I_d \mid C ].
    \end{align*}
    One can make a similar definition for
    other $I$.
    Note that $G(d, n) = \bigcup_I U_I$.
  \item Show that the $U_I$ glue to give
    $G(d, n)$ the structure of a variety.
  \end{enumerate}
  The second strategy is to use the
  wedge product $\Lambda^d V$.
  Recall that
  \begin{itemize}
    \item $\Lambda^d V$ has basis given
      by $e_I := e_{i_1} \wedge \cdots \wedge e_{i_d}$
      with $I = \{i_1 < \cdots < i_d\} \subseteq \{1, \dots, n\}$.
    \item For $v_1, \dots, v_d \in V$
      with $v_i = \sum_{j = 1}^d a_{i, j} e_j$,
      we have
      $v_1 \wedge \cdots \wedge v_d = \sum_{I \subseteq \{1, \dots, n\}} \det(a_I) e_I$.
  \end{itemize}
  The following are two linear algebra
  lemmas we will use:
  \begin{quote}
    \vspace{-2em}
    \begin{lemma}
      Let $v_1, \dots, v_d \in V$. Then
      $v_1, \dots, v_d$ are linearly independent
      if and only if $v_1 \wedge \cdots \wedge v_d \ne 0$.
    \end{lemma}

    \begin{proof}
      Use the determinant formula.
    \end{proof}

    \begin{lemma}\label{lem:span-wedge}
      For linearly independent
      sets $\{v_1, \dots, v_d\}, \{w_1, \dots, w_d\} \subseteq V$,
      then $v_1 \wedge \cdots \wedge v_d$
      and $w_1 \wedge \cdots \wedge w_d$
      are linearly dependent in
      $\Lambda^d V$ if and only if
      $\Span\{v_1, \dots, v_d\} = \Span\{w_1, \dots, w_d\}$.
    \end{lemma}

    \begin{proof}
      $(\Leftarrow)$ It suffices to show
      that $k v_1 \wedge \cdots \wedge v_d \subseteq \Lambda^d V$
      is preserved under change of basis operations.
      Check this as an exercise.

      $(\Rightarrow)$
      Assume that
      $v_1 \wedge \cdots \wedge v_d = \lambda w_1 \wedge \cdots \wedge w_d$
      for $\lambda \ne 0$.
      Then
      \[
        w_i \wedge v_1 \wedge \cdots \wedge v_d = 0,
      \]
      so $w_i \in \Span\{v_1, \dots, v_d\}$.
      So we have
      $\Span\{w_1, \dots, w_d\} \subseteq
      \Span\{v_1, \dots, v_d\}$,
      and by symmetry, the reverse inclusion
      holds as well.
    \end{proof}
  \end{quote}
  So given a $d$-dimensional vector subspace
  $\Span\{v_1, \dots, v_d\} = W \subseteq V$,
  we get a $1$-dimensional subspace
  $k_1 v_1 \wedge \cdots \wedge v_d = \Lambda^d W \subseteq \Lambda^d V$.
  By Lemma \ref{lem:span-wedge}, there is
  an injection
  \begin{align*}
    p_{n, d} : G(d, n)
    &\lhook\joinrel\longrightarrow \PP^{\binom{n}{d} - 1} \\
    W = \Span\{v_1, \dots, v_d\}
    &\longmapsto
    k v_1 \wedge \cdots \wedge v_d
  \end{align*}
  This is called the \emph{Pl\"ucker embedding}.
  It remains to show $p_{n, d}$
  is closed, which is Corollary
  \ref{cor:plucker-closed}.
\end{proof}
